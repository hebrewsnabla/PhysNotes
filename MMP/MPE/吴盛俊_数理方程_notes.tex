\documentclass[UTF8]{ctexart} % use larger type; default would be 10pt

\usepackage[utf8]{inputenc} % set input encoding (not needed with XeLaTeX)

%%% Examples of Article customizations
% These packages are optional, depending whether you want the features they provide.
% See the LaTeX Companion or other references for full information.

%%% PAGE DIMENSIONS
\usepackage{geometry} % to change the page dimensions
\geometry{a4paper} % or letterpaper (US) or a5paper or....
% \geometry{margin=2in} % for example, change the margins to 2 inches all round
% \geometry{landscape} % set up the page for landscape
%   read geometry.pdf for detailed page layout information

\usepackage{graphicx} % support the \includegraphics command and options

% \usepackage[parfill]{parskip} % Activate to begin paragraphs with an empty line rather than an indent

%%% PACKAGES
\usepackage{booktabs} % for much better looking tables
\usepackage{array} % for better arrays (eg matrices) in maths
\usepackage{paralist} % very flexible & customisable lists (eg. enumerate/itemize, etc.)
\usepackage{verbatim} % adds environment for commenting out blocks of text & for better verbatim
\usepackage{subfig} % make it possible to include more than one captioned figure/table in a single float
% These packages are all incorporated in the memoir class to one degree or another...

%%% HEADERS & FOOTERS
\usepackage{fancyhdr} % This should be set AFTER setting up the page geometry
\pagestyle{fancy} % options: empty , plain , fancy
\renewcommand{\headrulewidth}{0pt} % customise the layout...
\lhead{}\chead{}\rhead{}
\lfoot{}\cfoot{\thepage}\rfoot{}

%%% SECTION TITLE APPEARANCE
\usepackage{sectsty}
\allsectionsfont{\sffamily\mdseries\upshape} % (See the fntguide.pdf for font help)
% (This matches ConTeXt defaults)

%%% ToC (table of contents) APPEARANCE
\usepackage[nottoc,notlof,notlot]{tocbibind} % Put the bibliography in the ToC
\usepackage[titles,subfigure]{tocloft} % Alter the style of the Table of Contents
\renewcommand{\cftsecfont}{\rmfamily\mdseries\upshape}
\renewcommand{\cftsecpagefont}{\rmfamily\mdseries\upshape} % No bold!

%%% END Article customizations

%%% The "real" document content comes below...

\setlength{\parindent}{0pt}
\usepackage{physics}
\usepackage{amsmath}
%\usepackage{symbols}
\usepackage{AMSFonts}
\usepackage{bm}
%\usepackage{eucal}
\usepackage{mathrsfs}
\usepackage{amssymb}
\usepackage{float}
\usepackage{multicol}
\usepackage{abstract}
\usepackage{empheq}
\usepackage{extarrows}
\usepackage{textcomp}


\DeclareMathOperator{\p}{\prime}
\DeclareMathOperator{\ti}{\times}
\DeclareMathOperator{\s}{\sum_{n=1}^{\infty}}
\DeclareMathOperator{\intinf}{\int_0^\infty}
\DeclareMathOperator{\intdinf}{\int_{-\infty}^\infty}
\DeclareMathOperator{\suminf}{\sum_{n=0}^\infty}
\DeclareMathOperator{\e}{\mathrm{e}}
\renewcommand{\I}{\mathrm{i}}
\DeclareMathOperator{\Arg}{\mathrm{Arg}}
\DeclareMathOperator{\ra}{\rightarrow}
\DeclareMathOperator{\llra}{\longleftrightarrow}
\DeclareMathOperator{\lra}{\longrightarrow}
\DeclareMathOperator{\dlra}{\Leftrightarrow}
\DeclareMathOperator{\dra}{\Rightarrow}
\newcommand{\dis}{\displaystyle}
\numberwithin{equation}{section}

\title{Notes of WU Shengjun MMP\\
Part II: Mathematical Physical Equation}
\author{hebrewsnabla}
%\date{} % Activate to display a given date or no date (if empty),
         % otherwise the current date is printed 

\begin{document}
% \boldmath
\maketitle
\setcounter{section}{6}
\section{}
\subsection{Math Model}
\subsubsection{Oscillation of String}
\begin{equation}\label{key}
\pdv[2]{u}{t} - \dfrac{T}{\rho}\pdv[2]{u}{x} = \dfrac{F(x,t)}{\rho}
\end{equation}
\subsection{D'Alembert Eq.}
\subsubsection{Infinite String}
\begin{equation}\label{1}
\qty(\pdv[2]{t} - a^2 \pdv[2]{x})u(x,t) = 0
\end{equation}
Denote
\begin{equation}\label{key}
\xi = x + at \quad \eta = x - at
\end{equation}
thus
\begin{equation}\label{key}
\pdv{x} = \pdv{\xi} + \pdv{\eta}
\end{equation}
\begin{equation}\label{key}
\pdv[2]{x} = \pdv[2]{\xi} + 2\pdv[2]{}{\xi}{\eta} + \pdv[2]{\eta}
\end{equation}
\begin{equation}\label{key}
\pdv[2]{t} = a^2 \qty(\pdv[2]{\xi} - 2\pdv[2]{}{\xi}{\eta} + \pdv[2]{\eta})
\end{equation}
$\eqref{1}$ can be rewritten as
\begin{equation}\label{key}
-4a^2 \pdv{}{\xi}{\eta} u(\xi,\eta) = 0
\end{equation}
i.e.
\begin{equation}\label{key}
\pdv{}{\xi}{\eta} u(\xi,\eta) = 0
\end{equation}

initial condition:
\begin{equation}\label{key}
u|_{t=0} = \phi(x)\quad \pdv{u}{t}|_{t=0} = \psi(x)
\end{equation}
suppose 
\begin{equation}\label{key}
u(x,t) = f_1(x+at) + f_2(x-at)
\end{equation}
thus
\begin{equation}\label{key}
\left\{\mqty{\phi(x) = f_1(x) + f_2(x)\\
	\psi(x) = a(f_1^{\p}(x) - f_2^{\p}(x))}
\right.
\end{equation}
\begin{equation}\label{key}
f_1(x) - f_2(x) = \dfrac{1}{a}\int_{x_0}^x \psi(\xi)\dd\xi + f_1(x_0) - f_2(x_0)
\end{equation}
thus
\begin{equation}\label{key}
\left\{\mqty{f_1(x) = \dfrac{1}{2}\phi(x) + \dfrac{1}{2a}\int_{x_0}^x \psi(\xi)\dd\xi + \dfrac{1}{2}[f_1(x_0) - f_2(x_0)]~\vspace{10pt}\\
f_2(x) = \dfrac{1}{2}\phi(x) - \dfrac{1}{2a}\int_{x_0}^x \psi(\xi)\dd\xi - \dfrac{1}{2}[f_1(x_0) - f_2(x_0)]}\right.
\end{equation}
\begin{equation}\label{key}
\left\{\mqty{f_1(x+at) = \dfrac{1}{2}\phi(x+at) + \dfrac{1}{2a}\int_{x_0}^{x+at} \psi(\xi)\dd\xi + \dfrac{1}{2}[f_1(x_0) - f_2(x_0)]~\vspace{10pt}\\
f_2(x-at) = \dfrac{1}{2}\phi(x-at) - \dfrac{1}{2a}\int_{x_0}^{x-at} \psi(\xi)\dd\xi - \dfrac{1}{2}[f_1(x_0) - f_2(x_0)]}\right.
\end{equation}
General solution (D'Alembert Eq.):
\begin{equation}\label{key}
\begin{aligned}
u(x,t) &= f_1(x+at) + f_2(x-at)\\
&=\dfrac{1}{2}[\phi(x+at) + \phi(x-at)] + \dfrac{1}{2a}\int_{x-at}^{x+at} \psi(\xi)\dd\xi
\end{aligned}
\end{equation}

\subsubsection{Half infinite string}
\paragraph{Odd Continuation}
Boundary condition: $x\geq 0,\; u|_{x=0} = 0$.\\
Odd continuation: $\phi{x} \ra \Phi(x),\; \psi(x) \ra \Psi(x)$
\begin{equation}\label{key}
\Phi(x) = \left\{\begin{aligned}
&\phi(x) \quad & x\geq 0\\
&-\phi(-x)\quad &x\leq 0
\end{aligned}\right.
\qquad\quad
\Psi(x) = \left\{\begin{aligned}
&\psi(x) \quad & x\geq 0\\
&-\psi(-x)\quad &x\leq 0
\end{aligned}\right.
\end{equation}
thus
\begin{equation}\label{key}
\begin{aligned}
u(x,t) &=\dfrac{1}{2}[\Phi(x+at) + \Phi(x-at)] + \dfrac{1}{2a}\int_{x-at}^{x+at} \Psi(\xi)\dd\xi\\
&=\left\{\mqty{\dfrac{1}{2}[\phi(x+at) + \phi(x-at)] + \dfrac{1}{2a}\int_{x-at}^{x+at} \psi(\xi)\dd\xi \quad t\leq\dfrac{x}{a}~\vspace{7pt}\\
\dfrac{1}{2}[\phi(x+at) - \phi(-x+at)] + \dfrac{1}{2a}\int_{-x+at}^{x+at} \psi(\xi)\dd\xi \quad t\geq\dfrac{x}{a}}
\right.
\end{aligned}
\end{equation}
\paragraph{Even Continuation}
...\\

\section{Separation of Variables}
Boundary conditions:\\
Type I (Dirichlet): $u(x,t)|_s = f_1$\\
Type II (Neumann): $\dis\pdv{u}{x}\rvert_s = f_2$\\
Type III (Robin): $~$

\subsection{}
String oscillation
\begin{equation}\label{key}
\left\{
\begin{aligned}
&\pdv[2]{u}{t} - a^2\pdv[2]{u}{x} = 0 \quad (0 < x <\ell, t>0)\\
&u|_{x=0} = 0,\; u|_{x=\ell} = 0 \quad (t>0)\\
&u|_{t=0} = \phi(x),\; \pdv{u}{t}|_{t=0} = \psi(x) \quad(0\leq x\leq \ell)
\end{aligned}\right.
\end{equation}
Suppose
\begin{equation}\label{key}
u(x,t) = X(x)T(t)
\end{equation}
thus
\begin{equation}\label{key}
X(x)T''(t) = a^2 X''(x)T(t)
\end{equation}


\subsection{非齐次}
\subsubsection{Fourier}

\subsubsection{冲量定理法}
受迫振动
\begin{equation}\label{key}
\pdv[2]{u}{t} - a^2\pdv[2]{u}{x} = f(x,t)\\
u|_{x=0} = 0,\; u|_{x=\ell} = 0\\
u|_{t=0} = \phi(x),\; \pdv{u}{t}|_{t=0} = \psi(x)
\end{equation}
decomposition:
\begin{equation}\label{key}
u(x,t) = u_I(x,t) + u_{II}(x,t)
\end{equation}
which satisfies
\begin{equation}\label{key}
\pdv[2]{u_I}{t} - a^2\pdv[2]{u_I}{x} = 0\\
u_I|_{x=0} = 0,\; u_I|_{x=\ell} = 0\\
u_I|_{t=0} = \phi(x),\; \pdv{u_I}{t}|_{t=0} = \psi(x)
\end{equation}
\begin{equation}\label{key}
\pdv[2]{u_{II}}{t} - a^2\pdv[2]{u_{II}}{x} = f(x,t)\\
u_{II}|_{x=0} = 0,\; u_{II}|_{x=\ell} = 0\\
u_{II}|_{t=0} = 0,\; \pdv{u_{II}}{t}|_{t=0} = 0
\end{equation}
$u_I$ goto $8.1$\\
$u_{II}$ (冲量定理法)\\



\subsection{(非齐次) Boundary Condition}
\begin{equation}\label{key}
\begin{aligned}
&\pdv[2]{u}{t} - a^2\pdv[2]{u}{x} = f(x,t)\\
&u|_{x=0} = \mu(t),\; u|_{x=\ell} = \nu(t)\\
&u|_{t=0} = \phi(x),\; \pdv{u}{t}|_{t=0} = \psi(x)
\end{aligned}
\end{equation}
Choose 

\subsection{Poisson Equation}
\begin{equation}\label{key}
\left\{\begin{aligned}
\nabla^2 u = f(\vb{r})\\
u|_\Sigma = \phi(M)
\end{aligned}\right.
\end{equation}
choose $v(\vb{r})$, s.t.
\begin{equation}\label{key}
\nabla^2 v = f(\vb{r})
\end{equation}
Let $u = v + w$
\begin{equation}\label{key}
\left\{\begin{aligned}
\nabla^2 w = 0\\
w|_\Sigma = \phi(M) - v|_\Sigma
\end{aligned}\right.
\end{equation}

\section{Solving 2\textdegree ODE with Series}
\subsection{}
spheric coord
\begin{equation}\label{key}
\nabla^2_r = \dfrac{1}{r^2}\pdv{r}\qty(r^2\pdv{r})
\end{equation}

\subsubsection{Laplace Equation}
\begin{equation}\label{key}
\nabla^2 u = 0
\end{equation}
\paragraph{Spheric Coords}
\begin{equation}\label{key}
\dfrac{1}{r^2}\pdv{r}\qty(r^2\pdv{u}{r}) + \dfrac{1}{r^2\sin\theta}\pdv{\theta}\qty(\sin\theta\pdv{u}{\theta}) + \dfrac{1}{r^2\sin^2\theta}\pdv[2]{u}{\phi} = 0
\end{equation}
Let
\begin{equation}\label{key}
u(r,\theta,\phi) = R(r)Y(\theta,\phi)
\end{equation}

\paragraph{Cylindric Coords}

\subsubsection{波动方程}

\subsubsection{输运方程}

\subsubsection{Helmholtz Equation}
\paragraph{Spheric Coords}
\paragraph{Cylindric Coords}

\subsection{常点邻域上的级数解法}
\begin{equation}\label{key}
\dv[2]{w}{z} + p(z)\dv{w}{z} + q(z)w = 0
\end{equation}
\subsubsection{(常点) and Singularity}

\subsubsection{常点邻域上的级数解}

\subsubsection{Solving Legendre Eq. with Series}

\subsection{Serial Solution in Neighborhood of Canonical Singularity}
\subsubsection{}
\subsubsection{}

\subsubsection{Bessel Equation}

\subsubsection{Imaginary Bessel Equation}

\subsection{Sturm–Liouville Eigenvalue Problem}
\subsubsection{Sturm–Liouville Eigenvalue Problem}
Sturm–Liouville Equation
\begin{equation}\label{key}
\dv{x}\qty[k(x)\dv{y}{x}] - q(x)y + \lambda\rho(x)y = 0 \quad (a\leq x\leq b)
\end{equation}

\subsubsection{Eigenvalue Problem}

\subsubsection{Generalized Fourier Series}





\section{Spheric Function}
\begin{equation}\label{key}
\dfrac{1}{\sin\theta}\pdv{\theta}\qty(\sin\theta\pdv{Y}{\theta}) + \dfrac{1}{\sin^2 \theta}\pdv[2]{Y}{\phi} + \ell(\ell + 1)Y = 0
\end{equation}
\subsection{axis-symmetric}
\begin{equation}\label{key}
\dfrac{1}{\sin\theta}\pdv{\theta}\qty(\sin\theta\pdv{\Theta}{\theta})  + \ell(\ell + 1)\Theta = 0
\end{equation}
Let $x=\cos\theta$
\begin{equation}\label{key}
(1-x^2)\dv[2]{\Theta}{x} - 2x\dv{\Theta}{x} + \ell(\ell +1)\Theta = 0
\end{equation}
\subsubsection{Legendre Polynomial}
\begin{equation}\label{key}
P_\ell(x) = \sum_{k=0}^{[\ell/2]}(-1)^k \dfrac{(2\ell - 2k)!}{k!2^\ell (\ell - k)!(\ell - 2k)!}x^{\ell - 2k}
\end{equation}
\paragraph{Differential representation}~\\
...\\
\paragraph{Integral representation}~\\
...\\

\subsubsection{Second Legendre Function}

\subsubsection{Orthogonality}

\subsubsection{Normality}

\subsubsection{Generalized Fourier Series}
\begin{equation}\label{key}
f(x) = \sum_{\ell=0}^\infty f_\ell P_\ell(x)
\end{equation}

\subsubsection{Axis-symmetric Solution of Laplace Function}


\subsection{Associated Legendre Polynomial}
Associated Legendre Equation
\begin{equation}\label{key}
(1 - x^2)\dv[2]{\Theta}{x} - 2x\dv{\Theta}{x} + \qty[\ell(\ell+1) - \dfrac{m^2}{1-x^2}]\Theta = 0
\end{equation}

\subsubsection{Associated Legendre Function}
\paragraph{Expression}
...\\
\begin{equation}\label{key}
P_\ell^m(x) = (1 - x^2)^{m/2} P_\ell^{[m]}(x)
\end{equation}
where
\begin{equation}\label{key}
P_\ell^{[m]}(x) = \dv[m]{P_\ell(x)}{x}
\end{equation}
and
\begin{equation}\label{key}
m = 0,1,2,\cdots,\ell
\end{equation}

\paragraph{Differential}

\paragraph{Integral}

\subsubsection{Orthogonality}
\begin{equation}\label{key}
\int_{-1}^1 P_k^m(x) P_\ell^m(x)\dd x = 0 \quad (k\neq\ell)
\end{equation}
or
\begin{equation}\label{key}
\int_0^\pi P_k^m(\cos\theta) P_\ell^m(\cos\theta)\sin\theta\dd\theta = 0
\end{equation}

\subsubsection{Module}
\begin{equation}\label{key}
(N_\ell^m)^2 = \int_{-1}^1[P_\ell^m(x)]^2\dd x = \sqrt{\dfrac{(\ell + m)!\cdot 2}{(\ell - m)!(2\ell + 1)}}
\end{equation}

\subsubsection{Generalized Fourier Series}
\begin{equation}\label{key}
\left\{
\begin{aligned}
& f(x) = \sum_{\ell=m}^\infty f_\ell P_\ell^m(x)\\
& f_\ell = \dfrac{(\ell - m)!(2\ell + 1)}{(\ell + m)!\cdot 2}\int_{-1}^1 f(x) P_\ell^m(x)\dd x
\end{aligned}\right.
\end{equation}

\subsubsection{递推}


\subsection{General Spheric Function}
\subsubsection{Spheric Function}
\paragraph{Expression}
\begin{equation}\label{key}
Y_\ell^m(\theta,\phi) = P_\ell^m(\cos\theta)\left\{\mqty{\sin m\phi\\ \cos m\phi}\right\}
\end{equation}
where $\{ \}$ means both upper and lower expression is ok.

\paragraph{Complex Form}

\end{document}
