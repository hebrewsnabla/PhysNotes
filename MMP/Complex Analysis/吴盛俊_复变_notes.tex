\documentclass[UTF8]{ctexart} % use larger type; default would be 10pt

\usepackage[utf8]{inputenc} % set input encoding (not needed with XeLaTeX)

%%% Examples of Article customizations
% These packages are optional, depending whether you want the features they provide.
% See the LaTeX Companion or other references for full information.

%%% PAGE DIMENSIONS
\usepackage{geometry} % to change the page dimensions
\geometry{a4paper} % or letterpaper (US) or a5paper or....
% \geometry{margin=2in} % for example, change the margins to 2 inches all round
% \geometry{landscape} % set up the page for landscape
%   read geometry.pdf for detailed page layout information

\usepackage{graphicx} % support the \includegraphics command and options

% \usepackage[parfill]{parskip} % Activate to begin paragraphs with an empty line rather than an indent

%%% PACKAGES
\usepackage{booktabs} % for much better looking tables
\usepackage{array} % for better arrays (eg matrices) in maths
\usepackage{paralist} % very flexible & customisable lists (eg. enumerate/itemize, etc.)
\usepackage{verbatim} % adds environment for commenting out blocks of text & for better verbatim
\usepackage{subfig} % make it possible to include more than one captioned figure/table in a single float
% These packages are all incorporated in the memoir class to one degree or another...

%%% HEADERS & FOOTERS
\usepackage{fancyhdr} % This should be set AFTER setting up the page geometry
\pagestyle{fancy} % options: empty , plain , fancy
\renewcommand{\headrulewidth}{0pt} % customise the layout...
\lhead{}\chead{}\rhead{}
\lfoot{}\cfoot{\thepage}\rfoot{}

%%% SECTION TITLE APPEARANCE
\usepackage{sectsty}
\allsectionsfont{\sffamily\mdseries\upshape} % (See the fntguide.pdf for font help)
% (This matches ConTeXt defaults)

%%% ToC (table of contents) APPEARANCE
\usepackage[nottoc,notlof,notlot]{tocbibind} % Put the bibliography in the ToC
\usepackage[titles,subfigure]{tocloft} % Alter the style of the Table of Contents
\renewcommand{\cftsecfont}{\rmfamily\mdseries\upshape}
\renewcommand{\cftsecpagefont}{\rmfamily\mdseries\upshape} % No bold!

%%% END Article customizations

%%% The "real" document content comes below...

\setlength{\parindent}{0pt}
\usepackage{physics}
\usepackage{amsmath}
%\usepackage{symbols}
\usepackage{AMSFonts}
\usepackage{bm}
%\usepackage{eucal}
\usepackage{mathrsfs}
\usepackage{amssymb}
\usepackage{float}
\usepackage{multicol}
\usepackage{abstract}
\usepackage{empheq}
\usepackage{extarrows}


\DeclareMathOperator{\p}{\prime}
\DeclareMathOperator{\ti}{\times}
\DeclareMathOperator{\s}{\sum_{n=1}^{\infty}}
\DeclareMathOperator{\intinf}{\int_0^\infty}
\DeclareMathOperator{\intdinf}{\int_{-\infty}^\infty}
\DeclareMathOperator{\suminf}{\sum_{n=0}^\infty}
\DeclareMathOperator{\e}{\mathrm{e}}
\renewcommand{\I}{\mathrm{i}}
\DeclareMathOperator{\Arg}{\mathrm{Arg}}
\DeclareMathOperator{\ra}{\rightarrow}
\DeclareMathOperator{\llra}{\longleftrightarrow}
\DeclareMathOperator{\lra}{\longrightarrow}
\DeclareMathOperator{\dlra}{\Leftrightarrow}
\DeclareMathOperator{\dra}{\Rightarrow}
\newcommand{\dis}{\displaystyle}
\numberwithin{equation}{section}

\title{Notes of WU Shengjun MMP\\
Part I: Complex Analysis}
\author{hebrewsnabla}
%\date{} % Activate to display a given date or no date (if empty),
         % otherwise the current date is printed 

\begin{document}
% \boldmath
\maketitle

\section{Complex Function}
\subsection{Complex Number}
\begin{equation}\label{key}
z=\rho\e^{\I\phi}
\end{equation}
$\rho$ is the modulus of $z$. $\phi$ is the argument of $z$, namely $\Arg z$.\\
Def: $\arg z \in \{\Arg z\}$, $0 \leq \arg z < 2\pi$\\
thus $\arg z$ is called the principle value of $Arg z$.

\begin{equation}\label{key}
\sqrt[n]{z}=\sqrt[n]{\rho}\e^{\I\tfrac{\phi+2k\pi}{n}}
\end{equation}
$\sqrt[n]{z}$ have $n$ different values\\
Logarithm and exponential of complex numbers are defined as
\begin{equation}\label{key}
\ln z=\ln (\rho\e^{\I\phi})=\ln\rho+\I(\phi+2k\pi)
\end{equation}
\begin{equation}\label{key}
z^s=\e^{s\ln z}=\e^{s(\ln\rho+\I(\phi+2k\pi))}
\end{equation}
Specially
\begin{equation}\label{key}
\ln\I=\I(\dfrac{\pi}{2}+2k\pi)
\end{equation}
\begin{equation}\label{key}
\I^\I=\e^{\I\cdot\I(\dfrac{\pi}{2}+2k\pi)}=-\dfrac{\pi}{2}+2k\pi
\end{equation}

\subsection{Complex Function}
\subsubsection{Definition}
If $\forall z \in E \subseteq \mathbb{C},\; \exists$ one or more complex number $\omega$ corresponds to $z$, we call $\omega$ as a complex function of $z$, namely
\begin{equation}\label{key}
\omega = f(z)
\end{equation}
\subsubsection{Domain}
Definitions:
\paragraph{Neighbourhood}~\\
Neighbourhood of $z_0$ is a disc of the form $\{z\in\mathbb{C}:\abs{z - z_0}<\varepsilon\}$
\paragraph{Interior Point}~\\
A point $z_0 \in S$ is said to be an interior point of $S$
if there exists a neighbourhood of $z_0$ which is contained in $S$.
\paragraph{Open Set}~\\
The set $S$ is said to be open if every point of $S$ is an interior point of $S$.
\paragraph{Exterior Point}~\\
A point $z_0$ is said to be an exterior point of $S$
if $z_0$ and all neighbourhood of $z_0$ are not contained in $S$.
\paragraph{Boundary Point}
\paragraph{Connectivity}
\paragraph{Domain}

\subsubsection{Examples}
\begin{equation}\label{key}
\sin z=\dfrac{1}{2\I}(\e^{\I z}-\e^{-\I z})
\end{equation}
\begin{equation}\label{key}
\cos z=\dfrac{1}{2}(\e^{\I z}+\e^{-\I z})
\end{equation}
\begin{equation}\label{key}
\begin{aligned}
\sin z&=\dfrac{\e^{\I x-y}-\e^{-\I x+y}}{2\I}\\
&=\dfrac{\e^{-y}(\cos x+\I\sin x)-\e^y(\cos x-\I\sin x)}{2\I}\\
&=\dfrac{\e^{-y}(-\I\cos x+\sin x)+\e^y(\I\cos x+\sin x)}{2}\\
&=\dfrac{\e^y+\e^{-y}}{2}\sin x + \I\dfrac{\e^y-\e^{-y}}{2}\cos x
\end{aligned}
\end{equation}
\begin{equation}\label{key}
\abs{\sin z}=\dfrac{1}{2}\sqrt{\e^{2y}+\e^{-2y}+2(\sin^2 x-\cos^2 x)}
\end{equation}
$\abs{\sin z}$ and $\abs{\cos z}$ can $>1$.\\
$\sin z$ and $\cos z$ have period $2\pi$.\\
\begin{equation}\label{key}
\sinh z=\dfrac{1}{2}(\e^z-\e^{-z})
\end{equation}
\begin{equation}\label{key}
\cosh x=\dfrac{1}{2}(\e^z+\e^{-z})
\end{equation}
$\e^z,\;\sinh z\;\cosh z$ have period $2\pi\I$.
\begin{equation}\label{key}
\ln z = \ln\abs{z} + \I\Arg z
\end{equation}
\subsubsection{Derivatives}
\begin{equation}\label{key}
\dv{f}{z}=\lim_{\Delta x\rightarrow 0}\dfrac{f(z+\Delta z)-f(z)}{\Delta z}
\end{equation}
suppose $\Delta y = 0,\;\Delta z=\Delta x \ra 0$
\begin{equation}\label{key}
\lim_{\Delta z \ra 0} \dfrac{f(z+\Delta z)-f(z)}{\Delta z}=\pdv{u}{x}+\I\pdv{v}{x}
\end{equation}
suppose $\Delta x = 0,\;\Delta z=\I\Delta y \ra 0$
\begin{equation}\label{key}
\lim_{\Delta z \ra 0} \dfrac{f(z+\Delta z)-f(z)}{\Delta z} = \pdv{v}{y} - \I\pdv{u}{y}
\end{equation}
Cauchy-Riemann (C-R) condition: $\dis\pdv{u}{x}+\I\dis\pdv{v}{x} = \dis\pdv{v}{y} - \I\dis\pdv{u}{y}$, i.e.
\begin{equation}\label{key}
\left\{\begin{aligned}
\pdv{u}{x} &= \pdv{v}{y}\\
\pdv{v}{x} &= -\pdv{u}{y}
\end{aligned}\right.
\end{equation}

which is the necessary condition for $f(z)$ being differentiable at $z$.\\
The sufficient condition: $\exists$ continuous $\dis\pdv{u}{x},\dis\pdv{v}{x},\dis\pdv{v}{y},\dis\pdv{u}{y}$, which satisfy C-R condition. \\
C-R cond $\dlra\; \pdv{u}{z^*}=0$\\
\paragraph{C-R Eq. in Polar Coordinates}
\begin{equation}\label{key}
\begin{aligned}
\pdv{u}{\rho} &= \dfrac{1}{\rho}\pdv{v}{\phi}\\
\dfrac{1}{\rho}\pdv{u}{\phi} &= -\pdv{v}{\rho}
\end{aligned}
\end{equation}
\section{Integral}
\subsection{Introduction}

\subsection{Cauchy Theorem}
单连通
\begin{equation}\label{key}
\oint_l f(z)\dd z=0
\end{equation}
复连通
\begin{equation}\label{key}
\oint_l f(z)\dd z + \sum_{i=1}^n\oint_{l_i}f(z)\dd z=0
\end{equation}
\subsection{不定积分}
Complex Newton-Lebniz
\begin{equation}\label{key}
F(z)=\int_{z_0}^z f(\zeta)\dd\zeta,\quad F^{\p}(z)=f(z)
\end{equation}
Consider integral
\begin{equation}\label{key}
I=\int_a^b z^n\dd z,\;n\in\mathbb{Z}
\end{equation}
1) $n\neq -1$\\
2) $n=-1$
\begin{equation}\label{key}
I=\ln b-\ln a=\ln\abs{\dfrac{b}{a}}+\I(\Arg b-\Arg a)
\end{equation}
What about
\begin{equation}\label{key}
I=\oint_l (z-\alpha)^n\dd z
\end{equation}
1) $\alpha$ is external, $I=0$\\
2) $\alpha$ is internal\\
2.1) $n\geqslant 0$\\
2.2) $n<0$, let $z-\alpha=R\e^{\I n\phi}$
\begin{equation}\label{key}
\begin{aligned}
I&=\int_0^{2\pi}R^n\e^{\I n\phi}\dd(\alpha+R\e^{\I n\phi})\\
&=\I R^{n+1}\int_0^{2\pi}\e^{\I(n+1)\phi}\dd\phi
\end{aligned}
\end{equation}
2.2.1) $n\neq -1$
2.2.2) $n=-1$
~\\
\begin{equation}\label{key}
I=\int_C (\abs{z}-\e^z\sin z)\dd z
\end{equation}
where $C$ is 

\subsection{Cauchy Equation}
$f(z)$在闭单联通区域$B$上解析, $l$is boundary of $B$, $\alpha\in B$.
\begin{equation}\label{key}
f(\alpha)=\dfrac{1}{2\pi\I}\oint_l \dfrac{f(z)}{z-\alpha}\dd z
\end{equation}
Discussion\\
1)
\begin{equation}\label{key}
f(z)=\dfrac{1}{2\pi\I}\oint_l\dfrac{}{}
\end{equation}
2) 无界推广\\
3) derivatives
\begin{equation}\label{key}
f^{\p}(z)=\dfrac{1}{2\pi\I}\oint_l\dfrac{f(\zeta)}{(\zeta-z)^2}\dd\zeta
\end{equation}
\begin{equation}\label{key}
f^{(n)}(z)=\dfrac{n!}{2\pi\I}\oint_l\dfrac{f(\zeta)}{(\zeta-z)^{n+1}}\dd\zeta
\end{equation}

\section{Power Series}
\subsection{Complex Series}
\subsubsection{Introduction}
\begin{equation}\label{key}
\sum_{k=1}^\infty w_k = \sum_{k=1}^\infty u_k + \I\sum_{k=1}^\infty v_k
\end{equation}
\subsubsection{Convergence Test}
\paragraph{Cauchy's Convergence Test}~\\
$\forall \varepsilon>0,\;\exists N,$ s.t. when $n>N,\;\forall p\in \mathbb{N}$
\begin{equation}\label{key}
\abs{\sum_{k=n+1}^{n+p} w_k}<\varepsilon
\end{equation}
\paragraph{Absolute Convergence}
\subsubsection{Function Series}
\paragraph{Convergence Test}
\subparagraph{Cauchy's}
\subsubsection{Uniform Convergence}

\subsection{Power Series}
\subsubsection{Definition}
\subsubsection{Convergence and Divergence Test}
\paragraph{D'Alembert's Test}
\paragraph{Root}
\paragraph{Convergence Circle}
\subsubsection{Analytical Features}

\subsection{Taylor Expansion}

\subsection{解析延拓}

\subsection{Laurent Expansion}
\subsubsection{Bilateral Power Series}
\begin{equation}\label{key}
\cdots+a_{-2}(z-z_0)^{-2} + a_{-1}(z-z_0)^{-1} + a_0 + a_1(z-z_0) + a_2(z-z_0)^2 + \cdots
\end{equation}
Positive part: convergence radius = $R_1$\\
Negative: denote $\zeta=\dfrac{1}{z-z_0}$\\
conv radius = $\dfrac{1}{R_2}$
thus, bilateral power series is abs and uniform conv when
\begin{equation}\label{key}
R_2 < \abs{z-z_0} < R_1
\end{equation}
which is called convergence ring.
\subsubsection{Laurent Expansion Th.}
~\\
~\\
pos part:\\
aka canonical part\\
neg part:\\
aka 主部\\
~\\
Laurent Expansion is unique. Proof omitted.\\


\paragraph{Attention}
1) $z=z_0$ may be a singularity or not.\\
2) Although Laurent Expansion looks the same as Taylor Expansion,
\begin{equation}\label{key}
a_k \neq \dfrac{F^{(k)}(z_0)}{k!}
\end{equation}
no matter whether $z_0$ is singularity.\\
3) \\

\subsection{Isolated Singularity}

\section{留数定理}
\subsection{}
\subsection{计算实变函数定积分}
\paragraph{type I}
\paragraph{type II}~\\
Suppose
\begin{equation}\label{key}
I = \lim_{\tiny\mqty{R_1\ra\infty\\R_2\ra\infty}} \int_{-R_1}{R_2} f(x)\dd{x}
\end{equation}
exists\\
when $R_1 = R_2 \ra \infty$, I is called principle (主值) of the integral above, namely
\begin{equation}\label{key}
\mathscr{P}\intdinf f(x)\dd{x}
\end{equation}
Th\\
\begin{equation}\label{key}
\mathscr{P}\intdinf f(x)\dd{x} = 2\pi\I\sum_{k} \Res f(b_k)\;\;\text{(upper semi-plane)}
\end{equation}
\paragraph{type III}
\begin{equation}\label{key}
\int_0^\infty F(x)\cos mx\dd x = \dfrac{1}{2}\intdinf F(x)\e^{\I mx}\dd x
\end{equation}
\begin{equation}\label{key}
\int_0^\infty G(x)\sin mx\dd x = \dfrac{1}{2\I}\intdinf G(x)\e^{\I mx}\dd x
\end{equation}
\paragraph{Jordan's Lemma (约当引理)}
\begin{equation}\label{key}
\lim_{R\ra\infty}\int_{C_R} F_z\e^{\I mz}\dd z = 0
\end{equation}
when $m>0$, $C_R$ is a semi-circle on upper semi-plane,\\
or $m<0$, $C_R$ is a semi-circle on lower semi-plane.

\end{document}
