\documentclass[UTF8]{ctexart} % use larger type; default would be 10pt

\usepackage[utf8]{inputenc} % set input encoding (not needed with XeLaTeX)

%%% Examples of Article customizations
% These packages are optional, depending whether you want the features they provide.
% See the LaTeX Companion or other references for full information.

%%% PAGE DIMENSIONS
\usepackage{geometry} % to change the page dimensions
\geometry{a4paper} % or letterpaper (US) or a5paper or....
% \geometry{margin=2in} % for example, change the margins to 2 inches all round
% \geometry{landscape} % set up the page for landscape
%   read geometry.pdf for detailed page layout information

\usepackage{graphicx} % support the \includegraphics command and options

% \usepackage[parfill]{parskip} % Activate to begin paragraphs with an empty line rather than an indent

%%% PACKAGES
\usepackage{booktabs} % for much better looking tables
\usepackage{array} % for better arrays (eg matrices) in maths
\usepackage{paralist} % very flexible & customisable lists (eg. enumerate/itemize, etc.)
\usepackage{verbatim} % adds environment for commenting out blocks of text & for better verbatim
\usepackage{subfig} % make it possible to include more than one captioned figure/table in a single float
% These packages are all incorporated in the memoir class to one degree or another...

%%% HEADERS & FOOTERS
\usepackage{fancyhdr} % This should be set AFTER setting up the page geometry
\pagestyle{fancy} % options: empty , plain , fancy
\renewcommand{\headrulewidth}{0pt} % customise the layout...
\lhead{}\chead{}\rhead{}
\lfoot{}\cfoot{\thepage}\rfoot{}

%%% SECTION TITLE APPEARANCE
\usepackage{sectsty}
\allsectionsfont{\sffamily\mdseries\upshape} % (See the fntguide.pdf for font help)
% (This matches ConTeXt defaults)

%%% ToC (table of contents) APPEARANCE
\usepackage[nottoc,notlof,notlot]{tocbibind} % Put the bibliography in the ToC
\usepackage[titles,subfigure]{tocloft} % Alter the style of the Table of Contents
\renewcommand{\cftsecfont}{\rmfamily\mdseries\upshape}
\renewcommand{\cftsecpagefont}{\rmfamily\mdseries\upshape} % No bold!

%%% END Article customizations

%%% The "real" document content comes below...

\setlength{\parindent}{0pt}
\usepackage{physics}
\usepackage{amsmath}
%\usepackage{symbols}
\usepackage{AMSFonts}
\usepackage{bm}
%\usepackage{eucal}
\usepackage{mathrsfs}
\usepackage{amssymb}
\usepackage{float}
\usepackage{multicol}
\usepackage{abstract}
\usepackage{empheq}
\usepackage{extarrows}
\usepackage{textcomp}


\DeclareMathOperator{\p}{\prime}
\DeclareMathOperator{\ti}{\times}
\DeclareMathOperator{\s}{\sum_{n=1}^{\infty}}
\DeclareMathOperator{\intinf}{\int_0^\infty}
\DeclareMathOperator{\intdinf}{\int_{-\infty}^\infty}
\DeclareMathOperator{\suminf}{\sum_{n=0}^\infty}
\DeclareMathOperator{\e}{\mathrm{e}}
\renewcommand{\I}{\mathrm{i}}
\DeclareMathOperator{\Arg}{\mathrm{Arg}}
\DeclareMathOperator{\ra}{\rightarrow}
\DeclareMathOperator{\llra}{\longleftrightarrow}
\DeclareMathOperator{\lra}{\longrightarrow}
\DeclareMathOperator{\dlra}{\Leftrightarrow}
\DeclareMathOperator{\dra}{\Rightarrow}
\newcommand{\dis}{\displaystyle}

\DeclareMathOperator{\psis}{\psi^\ast}
\DeclareMathOperator{\Psis}{\Psi^\ast}
\DeclareMathOperator{\hi}{\hat{\vb{i}}}
\DeclareMathOperator{\hj}{\hat{\vb{j}}}
\DeclareMathOperator{\hk}{\hat{\vb{k}}}
\DeclareMathOperator{\hr}{\hat{\vb{r}}}
\DeclareMathOperator{\hT}{\hat{\vb{T}}}
\DeclareMathOperator{\hH}{\hat{\vb{H}}}
\DeclareMathOperator{\hL}{\hat{\vb{L}}}
\DeclareMathOperator{\hp}{\hat{\vb{p}}}
\DeclareMathOperator{\hx}{\hat{\vb{x}}}
\DeclareMathOperator{\ha}{\hat{\vb{a}}}


\DeclareMathOperator{\Tdv}{-\dfrac{\hbar^2}{2m}\dv[2]{x}}
\DeclareMathOperator{\Tna}{-\dfrac{\hbar^2}{2m}\nabla^2}

\numberwithin{equation}{subsection}

\title{Notes of 137A\\
Quantum Mechanics}
\author{hebrewsnabla}
%\date{} % Activate to display a given date or no date (if empty),
         % otherwise the current date is printed 

\begin{document}
% \boldmath
\maketitle

\tableofcontents

\newpage

Prof Mina Aganagic\\
bcourses.berkeley.edu\\
office -- 407 Old LeConte\\
office hour -- Th 2:00 - 3:00 p.m.\\
HW 30\%, Mid 30\%, Final 40\% \\
GSI: Christian Schmid\\
Mid -- Th Oct 18 in class, Final -- Tu Dec 12\\
~\\
Text\\
Griffths, Intro QM\\
Feynman\\
Morrison, Understanding Quantum Mechanics\\
~\\

\section{The Wavefunction}
\setcounter{subsection}{5}
\subsection{The Uncertainty Principle}
Uncertainty
\begin{equation}\label{key}
\sigma_Q = \sqrt{\langle Q^2\rangle - \langle Q\rangle^2}
\end{equation}
Uncertainty Principle
\begin{equation}\label{key}
\sigma_x\sigma_p \geq \dfrac{\hbar}{2}
\end{equation}

\section{Time-independent Schr\"odinger Eq.}
\subsection{Stationary State}
\begin{equation}\label{key}
\I\hbar\pdv{\Psi}{t} = \hat{\vb{H}}\Psi
\end{equation}
Assume
\begin{equation}\label{key}
V = V(x)
\end{equation}
\begin{equation}\label{key}
\Psi(x,t) = \phi(t)\psi(x)
\end{equation}
thus
\begin{equation}\label{key}
\pdv{t}\Psi(x,t) = \dv{\phi(t)}{t}\psi(x)  \quad \pdv[2]{x}\Psi(x,t) = \phi(t)\pdv[2]{\psi(x)}{x}
\end{equation}
\begin{equation}\label{key}
\I\hbar\dv{\phi(t)}{t}\psi(x) = -\dfrac{\hbar^2}{2m}\phi(t)\pdv[2]{\psi(x)}{x} + V(x)\phi(t)\psi(x)
\end{equation}
\begin{equation}\label{key}
\dfrac{\I\hbar}{\phi(t)}\dv{\phi(t)}{t} = -\dfrac{\hbar^2}{2m\psi(x)}\pdv[2]{\psi(x)}{x} + V(x)
\end{equation}
thus
\begin{equation}\label{key}
\dfrac{\I\hbar}{\phi(t)}\dv{\phi(t)}{t} = E = -\dfrac{\hbar^2}{2m\psi(x)}\pdv[2]{\psi(x)}{x} + V(x)
\end{equation}
right side -- TISE\\
left side --
\begin{equation}\label{key}
\I\hbar\dv{\phi(t)}{t} = E\phi(t)
\end{equation}
\begin{equation}\label{key}
\phi(t) = \e^{-\I Et/\hbar}
\end{equation}
No time dependence of physically measurable quantities
\begin{equation}\label{key}
p(x,t) = \abs{\Psi(x,t)}^2 = \psis(x)\e^{\I Et/\hbar}\psi(x)\e^{-\I Et/\hbar}
\end{equation}
\begin{equation}\label{key}
\dv{t}\langle Q(x,-\I\hbar\pdv{x})\rangle = 0
\end{equation}
\begin{equation}\label{key}
\sigma_H = \sqrt{\langle \hat{\vb{H}}^2\rangle - \langle \hat{\vb{H}}\rangle^2} = 0
\end{equation}
~\\
Every solution of TDSE can be written as linear combination of solutions with definite energies (sol of TISE)\\
Assume "discrete spectrum" ...\\

most general sol of SE
\begin{equation}\label{key}
\Psi(x,t) = \sum_n^\infty c_n\psi_n(x)\e^{-\I E_n t/\hbar}
\end{equation}
where $ c_n \in \mathbb{C}$.\\
How to find $ c_n $?\\
Assume initial condition $ \Psi(x,0) $\\
Assume discrete \& non-degenerate spectrum: $ E_n \neq E_m $ if $ n\neq m $\\
thus
\begin{equation}\label{key}
\Psi(x,0) = \sum_n^\infty c_n\psi_n(x)
\end{equation}
\begin{equation}\label{key}
c_n = \int\dd x \Psi(x,0)\psis_n(x)
\end{equation}

\newpage
9/11\\
1-D hamil has no-degenerate eigenvalues\\
if so
\begin{equation}\label{key}
-\dfrac{\hbar^2}{2m}\dv[2]{\psi_1}{x} + V(x)\psi_1 = E\psi_1
\end{equation}
\begin{equation}\label{key}
-\dfrac{\hbar^2}{2m}\dv[2]{\psi_2}{x} + V(x)\psi_2 = E\psi_2
\end{equation}
\begin{equation}\label{key}
-\dfrac{\hbar^2}{2m}\dv[2]{\psi_1}{x}\dfrac{1}{\psi_1} = -\dfrac{\hbar^2}{2m}\dv[2]{\psi_2}{x}\dfrac{1}{\psi_2}
\end{equation}
\begin{equation}\label{key}
\psi_1\psi_2'' - \psi_2\psi_1'' = 0
\end{equation}
\begin{equation}\label{key}
\dd{x}(\psi_2\psi_1' - \psi_1\psi_2') = 0
\end{equation}
\begin{equation}\label{key}
\psi_2\psi_1' - \psi_1\psi_2' = Cons.
\end{equation}
since when $ x\ra\infty $, $ \psi_2\psi_1' - \psi_1\psi_2' \ra 0 $
\begin{equation}\label{key}
\psi_2\psi_1' - \psi_1\psi_2' = 0
\end{equation}
\begin{equation}\label{key}
\dfrac{\psi_1'}{\psi_1} = \dfrac{\psi_2'}{\psi_2}
\end{equation}
\begin{equation}\label{key}
\ln\psi_1 = \ln\psi_2 + Cons.
\end{equation}
\begin{equation}\label{key}
\psi_1 = \psi_2\cdot Cons.
\end{equation}

superposition state is not 定态
\begin{equation}\label{key}
\Psi(x,t) = \dfrac{1}{\sqrt{2}}\qty(\psi_1(x)\e^{-\I E_1 t/\hbar} + \psi_2(x)\e^{-\I E_2 t/\hbar})
\end{equation}
...
\begin{equation}\label{key}
\langle x\rangle = \dfrac{1}{2}\qty(\bra{\psi_1}x\ket{\psi_1} + \bra{\psi_2}x\ket{\psi_2}) + \bra{\psi_1}x\ket{\psi_2}\cos\dfrac{(E_2 - E_1)t}{\hbar}
\end{equation}

\subsection{The Infinite Square Well}

\subsection{SH Oscillator}
\begin{equation}\label{key}
V(x) = \dfrac{1}{2}m\omega^2 x^2
\end{equation}
Expand near a minimum
\begin{equation}\label{key}
V(x) = V(x_0) + \dfrac{1}{2}\dv[2]{V(x)}{x} (x - x_0)^2 + \cdots
\end{equation}
def $ k_{eff} = \dv[2]{x}V(x)$

TISE for SHO
\begin{equation}\label{key}
-\dfrac{\hbar^2}{2m}\dv[2]{x}\Psi(x) + \dfrac{1}{2}m\omega^2 x^2\Psi(x) = E\Psi(x)
\end{equation}
\begin{equation}\label{key}
\hH = \dfrac{\hp^2}{2m} + \dfrac{1}{2}m\omega^2 x^2
\end{equation}
def
\begin{equation}\label{key}
\hH = \dfrac{1}{2}\hbar\omega (\ha_+ \ha_- + \ha_-\ha_+)
\end{equation}
or
\begin{equation}\label{key}
\hH = \hbar\omega(\ha_+\ha_- + \dfrac{1}{2})
\end{equation}
where
\begin{equation}\label{key}
\ha_\pm = \dfrac{1}{\sqrt{2\hbar m\omega}}(\mp\I \hp + m\omega x)
\end{equation}
\begin{equation}\label{key}
[\ha_-, \ha_+] = 1
\end{equation}
Noticing
\begin{equation}\label{key}
\begin{aligned}
[\hH,\ha_+] &= \hbar\omega[\ha_+\ha_- + \dfrac{1}{2}, \ha_+]\\
&= \hbar\omega\ha_+
\end{aligned}
\end{equation}
we have
\begin{equation}\label{key}
\hH(\ha_+\psi(x)) = (E+\hbar\omega)(\ha_+\psi(x))
\end{equation}
Similarly
\begin{equation}\label{key}
\begin{aligned}
[\hH,\ha_-] &= \hbar\omega[\ha_+\ha_- + \dfrac{1}{2}, \ha_-]\\
&= \hbar\omega\ha_-
\end{aligned}
\end{equation}
\begin{equation}\label{key}
\hH(\ha_\psi(x)) = (E - \hbar\omega)(\ha_-\psi(x))
\end{equation}


\newpage
9/18\\
Ex.
\begin{equation}\label{key}
\Psi(x,0) = A\qty(3\e^{\I\theta_0}\sin\dfrac{\pi x}{a} + 2\cos\dfrac{\pi x}{2a})
\end{equation}
Calc $ A $, $ \Psi(x,t) $.\\
Sol:
\begin{equation}\label{key}
\braket{\Psi}{\Psi} = \abs{A}^2\int_{-a}^a \qty(3\e^{-\I\theta_0}\sin\dfrac{\pi x}{a} + 2\cos\dfrac{\pi x}{2a})\qty(3\e^{\I\theta_0}\sin\dfrac{\pi x}{a} + 2\cos\dfrac{\pi x}{2a})\dd x = 1
\end{equation}

\newpage
9/20\\
suppose
\begin{equation}\label{key}
\ha_+\ha_- \psi(x) = u\psi(x)
\end{equation}
\begin{equation}\label{key}
\bra{\psi}\ha_+\ha_-\ket{\psi} = u
\end{equation}
\begin{equation}\label{key}
\bra{\psi}\ha_+\ha_-\ket{\psi} = \braket{\ha_-\psi}{\ha_-\psi}
\end{equation}
thus $ \ha_-\psi(x) = 0 $ iff $ u = 0 $.\\
??\\

\subsection{Free particle}
$ \psi_k(x,t) $ is not a quantum state of a free particle
\begin{equation}\label{key}
\psi(x,t) = \sum_k a_k\psi_k(x,t)
= \dfrac{1}{\sqrt{2\pi}}\intdinf \dd k \phi(k)\e^{\I k(x - \hbar kt/2m)}
\end{equation}

\newpage
10/9\\
\begin{equation}\label{key}
\expval{Q^2} = \expval{Q}^2
\end{equation}
\begin{equation}\label{key}
\bra{Q\psi}\ket{Q\psi} \bra{\psi}\ket{\psi} = \abs{\bra{\psi}\ket{Q\psi}}^2
\end{equation}

\section{Formalism}
\subsection{}
\subsection{}
\subsection{Eigenfunctions of a Hermitian Operator}
\subsubsection{Discrete Spectra}
Let $ \{\ket{f_{q,A}}\}_A $ be a collection of orthogonal eigenvectors of eigenvalue $ q $.
\begin{equation}\label{key}
\bra{f_{q,A}}\ket{f_{q',B}} = \delta_{qq'}\delta_{AB}
\end{equation}

\subsubsection{Continuous Spectra}



\subsection{Generalized Statistical Interpretation}
If
\begin{equation}\label{key}
\ket{\Psi} = \sum_{q,A} c_{q,A}\ket{f_{q,A}}
\end{equation}
thus 
\begin{equation}\label{key}
c_{q,A} = \braket{f_{q,A}}{\Psi}
\end{equation}

\subsection{}

\subsection{Dirac Notation}
Identity operator
\begin{equation}\label{key}
\hat{I} = \sum_{q,A}\ket{f_{q,A}}\bra{f_{q,A}}
\end{equation}

\section{}
\begin{equation}\label{key}
[\hL, \hH] = 0
\end{equation}
\begin{equation}\label{key}
[\hL_x, \hL_y] = \I\hbar\hL_z
\end{equation}
\begin{equation}\label{key}
[\hL^2, \hL_x] = 0
\end{equation}

\begin{equation}\label{key}
\qty[\Tna + \dfrac{\ell(\ell+1)}{2mr^2}\hbar^2]R(r) = E R(r)
\end{equation}

\begin{equation}\label{key}
\hL_\pm = \hL_x \pm \I\hL_y
\end{equation}
\begin{equation}\label{key}
[\hL_z, \hL_\pm] = \pm\hbar\hL_\pm
\end{equation}
If
\begin{equation}\label{key}
\hL_z f = \mu f
\end{equation}
\begin{equation}\label{key}
\hL_z(\hL_\pm f) = (\mu + \hbar)(\hL_\pm f)
\end{equation}






\end{document}
