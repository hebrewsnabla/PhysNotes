\documentclass[UTF8]{ctexart} % use larger type; default would be 10pt

\usepackage[utf8]{inputenc} % set input encoding (not needed with XeLaTeX)

%%% Examples of Article customizations
% These packages are optional, depending whether you want the features they provide.
% See the LaTeX Companion or other references for full information.

%%% PAGE DIMENSIONS
\usepackage{geometry} % to change the page dimensions
\geometry{a4paper} % or letterpaper (US) or a5paper or....
% \geometry{margin=2in} % for example, change the margins to 2 inches all round
% \geometry{landscape} % set up the page for landscape
%   read geometry.pdf for detailed page layout information

\usepackage{graphicx} % support the \includegraphics command and options

% \usepackage[parfill]{parskip} % Activate to begin paragraphs with an empty line rather than an indent

%%% PACKAGES
\usepackage{booktabs} % for much better looking tables
\usepackage{array} % for better arrays (eg matrices) in maths
\usepackage{paralist} % very flexible & customisable lists (eg. enumerate/itemize, etc.)
\usepackage{verbatim} % adds environment for commenting out blocks of text & for better verbatim
\usepackage{subfig} % make it possible to include more than one captioned figure/table in a single float
% These packages are all incorporated in the memoir class to one degree or another...

%%% HEADERS & FOOTERS
\usepackage{fancyhdr} % This should be set AFTER setting up the page geometry
\pagestyle{fancy} % options: empty , plain , fancy
\renewcommand{\headrulewidth}{0pt} % customise the layout...
\lhead{}\chead{}\rhead{}
\lfoot{}\cfoot{\thepage}\rfoot{}

%%% SECTION TITLE APPEARANCE
\usepackage{sectsty}
\allsectionsfont{\sffamily\mdseries\upshape} % (See the fntguide.pdf for font help)
% (This matches ConTeXt defaults)

%%% ToC (table of contents) APPEARANCE
\usepackage[nottoc,notlof,notlot]{tocbibind} % Put the bibliography in the ToC
\usepackage[titles,subfigure]{tocloft} % Alter the style of the Table of Contents
\renewcommand{\cftsecfont}{\rmfamily\mdseries\upshape}
\renewcommand{\cftsecpagefont}{\rmfamily\mdseries\upshape} % No bold!

%%% END Article customizations

%%% The "real" document content comes below...

\setlength{\parindent}{0pt}
\usepackage{physics}
\usepackage{amsmath}
%\usepackage{symbols}
\usepackage{AMSFonts}
\usepackage{bm}
%\usepackage{eucal}
\usepackage{mathrsfs}
\usepackage{amssymb}
\usepackage{float}
\usepackage{multicol}
\usepackage{abstract}
\usepackage{empheq}
\usepackage{extarrows}
\usepackage{fontenc}
\usepackage{textcomp}


\DeclareMathOperator{\p}{\prime}
\DeclareMathOperator{\ti}{\times}
\DeclareMathOperator{\s}{\sum_{n=1}^{\infty}}
\DeclareMathOperator{\intinf}{\int_0^\infty}
\DeclareMathOperator{\intdinf}{\int_{-\infty}^\infty}
\DeclareMathOperator{\suminf}{\sum_{n=0}^\infty}
\DeclareMathOperator{\sumin}{\sum_{i=1}^n}
\DeclareMathOperator{\e}{\mathrm{e}}
\renewcommand{\I}{\mathrm{i}}
\DeclareMathOperator{\llra}{\longleftrightarrow}
\DeclareMathOperator{\lra}{\longrightarrow}
\DeclareMathOperator{\dlra}{\Leftrightarrow}
\DeclareMathOperator{\dra}{\Rightarrow}
\newcommand{\txra}{\textrightarrow}
\newcommand{\dis}{\displaystyle}
\newcommand{\dbar}{{\mathchar'26\mkern-11mu\mathrm{d}}}
\numberwithin{equation}{subsection}

\title{Notes of JU Guoxing TD\&SP}
\author{hebrewsnabla}
%\date{} % Activate to display a given date or no date (if empty),
         % otherwise the current date is printed 

\begin{document}
% \boldmath
\maketitle
\setcounter{section}{18}
\section{Equipartition of Energy}
\subsection{Equipartition Theorem}

\section{Partition Function}
Energy Fluctuation:\\
$ \langle(E - \langle E\rangle)^2\rangle $\\
Relative fluctuation:\\
\begin{equation}\label{key}
\dfrac{\langle(E - \langle E\rangle)^2\rangle}{\langle E\rangle^2} \propto \dfrac{1}{N}
\end{equation}
\subsection{}
\subsection{Obtain functions of State}
\subsubsection{Internal E}
\begin{equation}\label{key}
U = \dfrac{\sum_i E_i \e^{-\beta E_i}}{\sum_i \e^{-\beta E_i}} = \dfrac{\sum_i E_i \e^{-\beta E_i}}{Z}
\end{equation}
while
\begin{equation}\label{key}
-\dv{Z}{\beta} = \sum_i E_i \e^{-\beta E_i}
\end{equation}
thus
\begin{equation}\label{key}
U = -\dfrac{1}{Z}\dv{Z}{\beta} = -\dv{\ln Z}{\beta}
\end{equation}
\subsubsection{Entropy}

E.g. 20.3\\
a) 2-evel system. Energy level $-\Delta/2,\Delta/2$.\\
\begin{equation}\label{key}
Z = 
\end{equation}
Find $U, F, S$\\
Discussion:\\
Def: characteristic temperature (特征温度) $k_B T_{ch} = E$\\
1) High temperature limit: $\beta\Delta = \dfrac{\Delta}{k_B T} << 1$\\
i.e. $T >> T_{ch}$\\
2) Low temperature limit: $\beta\Delta = \dfrac{\Delta}{k_B T} >> 1$\\
\begin{equation}\label{key}
U = -\dfrac{\Delta}{2}
\end{equation}
ground state occupied.\\
特征温度附近,热容量有极大值,称为Schottky反常。\\
~\\
b) simple harmonic oscillator
\begin{equation}\label{key}
Z = 
\end{equation}
Discussion:\\
Def: Einstein characteristic Temp $k_B \theta_E = \hbar\omega$\\
1) $T >> \theta_E$



\subsection{}

\subsection{Combining Partition Functions}

\section{Statistical Mechanics for Ideal Gas}
\subsection{Density of States}
box $V = L\ti L\ti L$\\
wave vector
\begin{equation}\label{key}
\vb{k} = \dfrac{\vb{p}}{\hbar}
\end{equation}
wave fxn
\begin{equation}\label{key}
\psi(x,y,z) = \qty(\dfrac{2}{3})^{3/2} \sin(k_x x)\sin(k_y y)\sin(k_z z)
\end{equation}
PBC:
\begin{equation}\label{key}
\psi(x,y,z) = \psi(x+L, y, z) = ...
\end{equation}
\begin{equation}\label{key}
k_x = \dfrac{2\pi n_x}{L}
\end{equation}
\begin{equation}\label{key}
k_y = ...
\end{equation}
\begin{equation}\label{key}
E = \dfrac{1}{2m}p^2 = \dfrac{2\pi^2\hbar^2}{mL^2}(n_x^2 + n_y^2 + n_z^2)
\end{equation}
in k space, every state occupy a volume $(2\pi/L)^3$, (in n space, vol = 1)\\
thus in momentum space, the volume of a state 
\begin{equation}\label{key}
\qty(\dfrac{2\pi\hbar}{L})^3 = \dfrac{h^3}{V}
\end{equation}
thus density of state
\begin{equation}\label{key}
g(\vb{p})\dd[3]\vb{p} = \dfrac{V}{h^3}\dd[3]\vb{p}
\end{equation}
\begin{equation}\label{key}
g(\vb{k})\dd[3]\vb{k} = \dfrac{V}{(2\pi)^3}\dd[3]\vb{k}
\end{equation}
in spheric coord
\begin{equation}\label{key}
g(p,\theta,\phi)\dd p\dd\theta\dd\phi = \dfrac{Vp^2\sin\theta}{h^3}\dd p\dd\theta\dd\phi
\end{equation}
\begin{equation}\label{key}
g(p)\dd p = \dfrac{Vp^2}{h^3}\dd p\int_0^\pi\sin\theta\dd\theta\int_0^{2\pi}\dd\phi = \dfrac{4\pi V}{h^3}p^2\dd p
\end{equation}
\begin{equation}\label{key}
\begin{aligned}
g(k)\dd k &= g(p)\dd p = g(p)\dv{p}{k}\dd k = \dfrac{4\pi V}{h^3}p^2 \hbar\dd k\\
&= \dfrac{V}{2\pi^2}k^2\dd k
\end{aligned}
\end{equation}
$\because$
\begin{equation}\label{key}
\dd E = \dfrac{p\dd p}{m}
\end{equation}
\begin{equation}\label{key}
\begin{aligned}
g(E)\dd E = g(p)\dd p = \dfrac{4\pi V}{h^3}\sqrt{2mE} m\dd E
\end{aligned}
\end{equation}

\subsection{Quantum Density}
single-particle parti fxn of ideal gas
\begin{equation}\label{key}
\begin{aligned}
Z_1 &= \intinf\e^{-\beta E(k)}g(k)\dd k\\
&= \dfrac{V}{2\pi^2}\intinf \e^{-\beta\hbar^2 k^2/2m}k^2\dd k\\
&= \dfrac{V}{\hbar^3}\qty(\dfrac{m}{2\pi\beta})^{3/2}
\end{aligned}
\end{equation}
def: quantum density
\begin{equation}\label{key}
n_Q = \dfrac{Z_1}{V} = \dfrac{1}{\hbar^3}\qty(\dfrac{m}{2\pi\beta})^{3/2}
\end{equation}
thermal wavelength (热波长)
\begin{equation}\label{key}
\lambda_{th} = \hbar\sqrt{\dfrac{2\pi\beta}{m}} = h\sqrt{\dfrac{\beta}{2\pi m}}
\end{equation}
\begin{equation}\label{key}
Z_1 = \dfrac{V}{\lambda_{th}}
\end{equation}

\subsection{Distinguishability (可分辨性)}
for distinguishable particles
\begin{equation}\label{key}
Z_N = (Z_1)^N
\end{equation}
indistinguishable but non-degenerate (非简并)
\begin{equation}\label{key}
Z_N = \dfrac{Z_1^N}{N!}
\end{equation}
for ideal gas, non-degeneracy requires
\begin{equation}\label{key}
N << \text{number of }E_\ell
\end{equation}
or number density
\begin{equation}\label{key}
n << n_Q
\end{equation}
that's a good approximation in room temp, but not good for electron in metals.

\subsection{State Functions of Ideal Gas}
\begin{equation}\label{key}
\begin{aligned}
\ln Z_N &= N(\ln V - 3\ln\lambda_{th}) - \ln N!\\
&= N\ln V + \dfrac{3}{2}N\ln T + Cons. = N\ln V - \dfrac{3}{2}N\ln\beta + Cons.
\end{aligned}
\end{equation}
\begin{equation}\label{key}
U = -\dv{\ln Z_N}{\beta} = \dfrac{3}{2}\dfrac{N}{\beta}
\end{equation}
\begin{equation}\label{key}
C_V = \dfrac{3}{2}k_B
\end{equation}
\begin{equation}\label{key}
\begin{aligned}
F &= -\dfrac{1}{\beta}\ln Z_N\\
&= -\dfrac{N}{\beta}\ln V - \dfrac{3N}{2\beta}\ln T - \dfrac{Cons.}{\beta}\\
&= -Nk_B T\ln V - \dfrac{3N}{2}k_B T\ln T - Cons.\cdot k_B T
\end{aligned}
\end{equation}
\begin{equation}\label{key}
p = -\qty(\pdv{F}{T})_T = \dfrac{Nk_B T}{V}
\end{equation}
---------------
\begin{equation}\label{key}
\begin{aligned}
\ln Z_N &= N\ln V - 3N\ln\lambda_{th}) - N\ln N + N\\
&=N\qty(1 + \ln\dfrac{V}{N\lambda^3})
\end{aligned}
\end{equation}
\begin{equation}\label{key}
F = ...
\end{equation}
\begin{equation}\label{key}
S = ...
\end{equation}
\begin{equation}\label{key}
G = ...
\end{equation}

\subsection{Gibbs Paradox}

\subsection{Heat Capacity of Diatomic Gas}




\section{Chemical Potential}
\subsection{Definition}
\begin{equation}\label{key}
\dd U = T\dd S - p\dd V + \mu\dd N
\end{equation}
\begin{equation}\label{key}
\mu = \qty(\pdv{U}{N})_{S,V}
\end{equation}

\subsection{Meaning of CP}
\begin{equation}\label{key}
\dd S = \qty(\pdv{S}{U})_{N,V}\dd U + \qty(\pdv{S}{V})_{N,U}\dd V + \qty(\pdv{S}{N})_{U,V}\dd N
\end{equation}
\begin{equation}\label{key}
\dd S = \dfrac{\dd U}{T} + \dfrac{p\dd V}{T} - \dfrac{\mu\dd N}{T}
\end{equation}
$\therefore$
\begin{equation}\label{key}
\qty(\pdv{S}{N})_{U,V} = -\dfrac{\mu}{T}
\end{equation}


\subsection{Grand Partition Function (巨配分函数)}
system $\epsilon, N, V$\\
reservoir $U >> \epsilon, \mathcal{N} >> N$\\
thus, entropy of reservoir
\begin{equation}\label{key}
S(U-\epsilon,\mathcal{N}-N) = S(U,\mathcal{N}) - \dfrac{1}{T}\epsilon + \dfrac{\mu}{T}N
\end{equation}
\begin{equation}\label{key}
\Omega = \e^{S/k_B} = \e^{S(U,\mathcal{N})}
\end{equation}
\begin{equation}\label{key}
\ln\Omega(U-\epsilon,\mathcal{N}-N) = \ln\Omega(U,\mathcal{N}) - \qty(\pdv{\ln\Omega}{N})(-N) + \qty(\pdv{\ln\Omega}{E})(-\epsilon)
\end{equation}
\begin{equation}\label{key}
\alpha = \qty(\pdv{\ln\Omega}{N}) = -\dfrac{\mu}{k_B T}
\end{equation}
\begin{equation}\label{key}
\beta = \qty(\pdv{\ln\Omega}{E}) = \dfrac{1}{k_B T}
\end{equation}
\begin{equation}\label{key}
P = \dfrac{1}{\Xi}\e^{-\alpha N - \beta E_s}
\end{equation}
grand partition function
\begin{equation}\label{key}
\Xi = 
\sum_{N=0}^\infty \sum_s \e^{\beta(\mu N - E_s)}
\end{equation}
or
\begin{equation}\label{key}
\Xi = \sum_{N=0}^\infty \sum_s \e^{\alpha N - \beta E_s}  \quad(\alpha = \beta\mu)
\end{equation}
-------------~\\
\begin{equation}\label{key}
\begin{aligned}
\langle N\rangle &= \dfrac{1}{\Xi}\sum_N\sum_s N \e^{\alpha N - \beta E_s}\\
&= \dfrac{1}{\Xi}\sum_N\sum_s \qty(-\pdv{\alpha}) \e^{\alpha N - \beta E_s} = -\pdv{\alpha}\ln\Xi
\end{aligned}
\end{equation}
\begin{equation}\label{key}
\begin{aligned}
U &= \langle E\rangle = \dfrac{1}{\Xi}\sum_N\sum_s E_s \e^{\alpha N - \beta E_s}\\
&= \dfrac{1}{\Xi}\sum_N\sum_s \qty(-\pdv{\beta}) \e^{\alpha N - \beta E_s} = -\pdv{\beta}\ln\Xi
\end{aligned}
\end{equation}
\begin{equation}\label{key}
X = 
\end{equation}

\subsection{Grand Potential}
\begin{equation}\label{key}
\Phi_G = -k_B T\ln\Xi
\end{equation}
Since
\begin{equation}\label{key}
S = k_B\qty(\ln\Xi - \alpha\pdv{\alpha}\ln\Xi - \beta\pdv{\beta}
\ln\Xi)
\end{equation}
\begin{equation}\label{key}
\langle N\rangle = -\pdv{\alpha}\ln\Xi
\end{equation}
\begin{equation}\label{key}
U = -\pdv{\beta}\ln\Xi
\end{equation}
thus
\begin{equation}\label{key}
S = k_B(\ln\Xi + \alpha\langle N\rangle + \beta U) = k_B\ln\Xi - \dfrac{\mu}{T}\langle N\rangle + \dfrac{1}{T}U
\end{equation}
\begin{equation}\label{key}
\Phi_G = U - TS - \mu\langle N\rangle = F - \mu\langle N\rangle
\end{equation}
\begin{equation}\label{key}
\dd\Phi_G = -S\dd T -p\dd V -\langle N\rangle\dd\mu
\end{equation}

\subsection{}
\begin{equation}\label{key}
S = \dfrac{1}{T}U + \dfrac{p}{T}V - \dfrac{\mu}{T}N
\end{equation}
i.e.
\begin{equation}\label{key}
U -TS + pV = \mu N
\end{equation}
thus
\begin{equation}\label{G = muN}
G = \mu N
\end{equation}
($\mu$ - single particle Gibbs function)
\begin{equation}\label{key}
\Phi_G = F - \mu N = F - G = -pV
\end{equation}
Differentiate $\eqref{G = muN}$
\begin{equation}\label{key}
\dd G = \mu\dd N + N\dd\mu
\end{equation}
while
\begin{equation}\label{key}
\dd G = -S\dd T + V\dd p +\mu\dd N
\end{equation}
thus
\begin{equation}\label{key}
S\dd T - V\dd p + N\dd\mu = 0
\end{equation}
which is Gibbs-Duhem Equation.\\
\paragraph{Fluctuation ...}

\subsection{}

\subsection{Conservation of Number of Particle}


\end{document}
