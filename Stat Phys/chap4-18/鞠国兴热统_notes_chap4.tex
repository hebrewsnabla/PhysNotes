\documentclass[UTF8]{ctexart} % use larger type; default would be 10pt

\usepackage[utf8]{inputenc} % set input encoding (not needed with XeLaTeX)

%%% Examples of Article customizations
% These packages are optional, depending whether you want the features they provide.
% See the LaTeX Companion or other references for full information.

%%% PAGE DIMENSIONS
\usepackage{geometry} % to change the page dimensions
\geometry{a4paper} % or letterpaper (US) or a5paper or....
% \geometry{margin=2in} % for example, change the margins to 2 inches all round
% \geometry{landscape} % set up the page for landscape
%   read geometry.pdf for detailed page layout information

\usepackage{graphicx} % support the \includegraphics command and options

% \usepackage[parfill]{parskip} % Activate to begin paragraphs with an empty line rather than an indent

%%% PACKAGES
\usepackage{booktabs} % for much better looking tables
\usepackage{array} % for better arrays (eg matrices) in maths
\usepackage{paralist} % very flexible & customisable lists (eg. enumerate/itemize, etc.)
\usepackage{verbatim} % adds environment for commenting out blocks of text & for better verbatim
\usepackage{subfig} % make it possible to include more than one captioned figure/table in a single float
% These packages are all incorporated in the memoir class to one degree or another...

%%% HEADERS & FOOTERS
\usepackage{fancyhdr} % This should be set AFTER setting up the page geometry
\pagestyle{fancy} % options: empty , plain , fancy
\renewcommand{\headrulewidth}{0pt} % customise the layout...
\lhead{}\chead{}\rhead{}
\lfoot{}\cfoot{\thepage}\rfoot{}

%%% SECTION TITLE APPEARANCE
\usepackage{sectsty}
\allsectionsfont{\sffamily\mdseries\upshape} % (See the fntguide.pdf for font help)
% (This matches ConTeXt defaults)

%%% ToC (table of contents) APPEARANCE
\usepackage[nottoc,notlof,notlot]{tocbibind} % Put the bibliography in the ToC
\usepackage[titles,subfigure]{tocloft} % Alter the style of the Table of Contents
\renewcommand{\cftsecfont}{\rmfamily\mdseries\upshape}
\renewcommand{\cftsecpagefont}{\rmfamily\mdseries\upshape} % No bold!

%%% END Article customizations

%%% The "real" document content comes below...

\setlength{\parindent}{0pt}
\usepackage{physics}
\usepackage{amsmath}
%\usepackage{symbols}
\usepackage{AMSFonts}
\usepackage{bm}
%\usepackage{eucal}
\usepackage{mathrsfs}
\usepackage{amssymb}
\usepackage{float}
\usepackage{multicol}
\usepackage{abstract}
\usepackage{empheq}
\usepackage{extarrows}
\usepackage{fontenc}
\usepackage{textcomp}


\DeclareMathOperator{\p}{\prime}
\DeclareMathOperator{\ti}{\times}
\DeclareMathOperator{\s}{\sum_{n=1}^{\infty}}
\DeclareMathOperator{\intinf}{\int_0^\infty}
\DeclareMathOperator{\intdinf}{\int_{-\infty}^\infty}
\DeclareMathOperator{\suminf}{\sum_{n=0}^\infty}
\DeclareMathOperator{\sumin}{\sum_{i=1}^n}
\DeclareMathOperator{\e}{\mathrm{e}}
\renewcommand{\I}{\mathrm{i}}
\DeclareMathOperator{\ra}{\rightarrow}
\DeclareMathOperator{\llra}{\longleftrightarrow}
\DeclareMathOperator{\lra}{\longrightarrow}
\DeclareMathOperator{\dlra}{\Leftrightarrow}
\DeclareMathOperator{\dra}{\Rightarrow}
\newcommand{\txra}{\textrightarrow}
\newcommand{\dis}{\displaystyle}
\newcommand{\dbar}{{\mathchar'26\mkern-11mu\mathrm{d}}}
\numberwithin{equation}{subsection}

\title{Notes of JU Guoxing TD\&SP}
\author{hebrewsnabla}
%\date{} % Activate to display a given date or no date (if empty),
         % otherwise the current date is printed 

\begin{document}
% \boldmath
\maketitle
\setcounter{section}{3}
\section{Temperature and Boltzmann Factor}
\subsection{Thermal Equilibrium}
\subsection{温度计}



~\\
compact - 紧致(数学译法) - 致密(物理译法)\\

~\\

\subsection{Microstates and Macrostates}
\subsection{Description of Dynamic states of particles}
\subsubsection{Classic Description}
\subsubsection{Quantum Description}
\subsubsection{Identical Particle System}
\paragraph{Identity Principle}~\\
Identical particles in quantum physics is indistinguishable.
\paragraph{Boson \& Fermion}~\\
光子的自旋简并度为2,不满足$2s+1$\\
光子的静质量为0不是显然的,而是外推的结果。\\
\paragraph{Bose, Fermi \& Boltzmann System}~\\
Fermion - Pauli Exclusion - Fermi-Dirac Distribution\\
Boson - no exclusion - Bose-Einstein Condensation (BEC)\\
就量子本性来说不可分辨,但可以根据位置加以区分 - Localized system / Boltzmann system - Maxwell-Boltzmann Distribution.
\subsubsection{Number of Microstates}
$\varepsilon_l$ energy level\\
$\varepsilon_s$ energy of state\\
$\omega_l$ degeneracy of energy level\\
$a_l$ number of particles of energy level\\
~\\
For systems w/ constant N, V, E
\begin{equation}\label{key}
\sum_l a_l = N\quad \sum_l \varepsilon_l a_l = E
\end{equation}
\paragraph{Boltzmann System}

\subsection{}
\subsection{Ensamble}
\newpage
Mar 27 \\
S.Weinberg 还原论\\
P.W.Anderson "More is different"\\
Wilczek\\
~\\
\subsection{Canonical Ensamble \& Boltzmann Distribution}
\paragraph{Canonical Ensamble}~\\
Constant $N,\;V,\;T\;$\\
与大热源(heat source, heat reservoir, heat bath; with huge heat capacity)仅有能量交换的平衡系统\\
$\varepsilon =$ energy of system\\
$E_r =$ energy of reservoir\\
$\varepsilon + E_r = E $ (Cons.)
~~~~~~~~~~$\varepsilon << E$\\

\newpage
非平衡态统计物理  普利高津  Belscu\\

\setcounter{section}{10}
\section{Energy}
\subsection{}
\subsection{First Law}
\begin{equation}\label{key}
\dd U = \dbar Q + \dbar W
\end{equation}
\begin{equation}\label{key}
\dbar W = -p\dd V \quad \text{(对系统做功为正)}
\end{equation}

\subsection{Capacity}
\begin{equation}\label{key}
\begin{aligned}
\dbar Q &= \dd U + p\dd V\\
&=\qty(\pdv{U}{T})_V \dd T + \qty(\pdv{U}{V})_T\dd V + p\dd V
\end{aligned}
\end{equation}
thus
\begin{equation}\label{key}
C_v \equiv \qty(\pdv{Q}{T})_V = \qty(\pdv{U}{T})_V
\end{equation}
\begin{equation}\label{key}
C_p = \qty(\pdv{Q}{T})_p = \qty(\pdv{U}{T})_V + \qty[\qty(\pdv{U}{V})_T + p]\qty(\pdv{V}{T})_p
\end{equation}
\begin{equation}\label{key}
C_p - C_V = \qty[\qty(\pdv{U}{V})_T + p]\qty(\pdv{V}{T})_p
\end{equation}
---------------------------- \\
For ideal gas (one atom, per mol)
\begin{equation}\label{key}
C_v = \dfrac{3}{2}R
\end{equation}
\begin{equation}\label{key}
C_p = C_V + [0+p]\dfrac{R}{p} = \dfrac{5}{2}R
\end{equation}
---------------------------- \\
Def: adiabatic index
\begin{equation}\label{key}
\gamma = \dfrac{C_p}{C_V}
\end{equation}

\section{Isothermal \& Adiabatic Processes}
\subsection{}
\subsection{Isothermal Process of Ideal Gas}
\paragraph{Attention:}
能够在p-V图上表示出来的,一定是平衡态或近似为平衡态,这样才有确定的p,T,否则画不出来的。\\
\begin{equation}\label{key}
\dd T = 0\;\dra\; \dd U = 0
\end{equation}
\begin{equation}\label{key}
\Delta Q = \int\dbar Q = -\int\dbar W = RT\ln\dfrac{V_2}{V_1}
\end{equation}

\subsection{Adiabatic Process of Ideal Gas}
\begin{equation}\label{key}
\dbar Q = 0
\end{equation}
\begin{equation}\label{key}
\dd U = \dbar W
\end{equation}
\begin{equation}\label{key}
C_V\dd T = -p\dd V = -\dfrac{RT}{V}\dd V
\end{equation}
\begin{equation}\label{key}
\ln\dfrac{T_1}{T_2} = -\dfrac{RT}{C_V}\ln\dfrac{V_2}{V_1}
\end{equation}
while
\begin{equation}\label{key}
\gamma = 1 + \dfrac{R}{C_V}
\end{equation}
thus
\begin{equation}\label{key}
TV^{\gamma-1} = Cons.
\end{equation}
or
\begin{equation}\label{key}
pV^\gamma = Cons.
\end{equation}
\paragraph{Attention:}
$pV = nRT$ is a state fxn, while $pV^\gamma = Cons.$ is a process fxn. (Although state fxn and process fxn may not be differed in Mechanics)

\subsection{Adiabatic Atmosphere}


\section{Heat Engine and the 2nd Law}
\subsection{The 2nd Law}
\subsubsection{Kelvin's Statement}
Discussion:\\
1) \\
2) \\
\subsubsection{Clausius' Statement}
Discussion:\\
1) \\
2) \\
\subsection{The Carnot Cycle}
\paragraph{Cycle Process}~\\
def\\
working substance\\
~\\
Carnot engine runs between two heat source w/ temp $T_h$ and $T_\ell$\\
A\txra B  isothermal
\begin{equation}\label{c1}
Q_h = RT_h\ln\dfrac{V_B}{V_A}
\end{equation}
B\txra C  adiabatic
\begin{equation}\label{c2}
\dfrac{T_h}{T_\ell}  =  \qty(\dfrac{V_C}{V_B})^{\gamma-1}
\end{equation}
C\txra D  isothermal
\begin{equation}\label{c3}
Q_\ell = -RT_\ell \ln\dfrac{V_D}{V_C}
\end{equation}
D\txra A  adiabatic
\begin{equation}\label{c4}
\dfrac{T_\ell}{T_h}  =  \qty(\dfrac{V_A}{V_D})^{\gamma-1}
\end{equation}
with $\eqref{c2},\eqref{c4}$
\begin{equation}\label{key}
\dfrac{V_C}{V_B}\dfrac{V_A}{V_D} = 1
\end{equation}
thus
\begin{equation}\label{key}
\dfrac{V_B}{V_A} = \dfrac{V_C}{V_D}
\end{equation}
with $\eqref{c1},\eqref{c3}$
\begin{equation}\label{key}
\dfrac{Q_h}{Q_\ell} = \dfrac{T_h}{T_\ell}
\end{equation}
%\begin{equation}\label{key}
%W = Q_h - Q_\ell = R\qty(T_h\ln\dfrac{V_B}{V_A} + T_\ell \ln\dfrac{V_D}{V_C}) = R\ln\dfrac{V_B}{V_A}(T_h - T_\ell)
%\end{equation}
\begin{equation}\label{key}
\eta = \dfrac{W}{Q_h} = \dfrac{Q_h - Q_\ell}{Q_h} = 1 - \dfrac{T_\ell}{T_h}
\end{equation}

\subsection{Carnot's Theorem}
\subsection{}
\subsection{}
\subsection{}
\subsection{Clausius' Theorem}
For Carnot Cycle
\begin{equation}\label{key}
\dfrac{Q_h}{Q_\ell} = \dfrac{T_h}{T_\ell}
\end{equation}
thus
\begin{equation}\label{key}
\sum\dfrac{Q_{rev}}{T} = \dfrac{Q_h}{T_h} + \dfrac{-Q_\ell}{T_\ell} = 0
\end{equation}
i.e.
\begin{equation}\label{key}
\oint\dfrac{\dbar Q_{rev}}{T} = 0
\end{equation}
\paragraph{Clausius Inequality}
\begin{equation}\label{key}
\oint\dfrac{\dbar Q}{T} \leq 0
\end{equation}

\section{Entropy}
\subsection{Definition}
Prove $\displaystyle\int_A^B \dfrac{\dbar Q_R}{T}$ is path-independent\\
def:
\begin{equation}\label{key}
\dd S = \dfrac{\dbar Q_R}{T}
\end{equation}
where "R" means reversible.
\begin{equation}\label{key}
S_B - S_A = \int_A^B \dfrac{\dbar Q_R}{T}
\end{equation}
For adiabatic process (aka isentropic process, 等熵过程)
\begin{equation}\label{key}
\dd S = 0
\end{equation}

\subsection{Irreversible Change}
\begin{equation}\label{key}
\oint \dfrac{\dbar Q_R}{T} = 0
\end{equation}
\begin{equation}\label{key}
\oint \dfrac{\dbar Q}{T} \leq 0
\end{equation}
\begin{equation}\label{key}
\int_A^B \dfrac{\dbar Q}{T} + \oint_B^A \dfrac{\dbar Q_R}{T} \leq 0
\end{equation}
\begin{equation}\label{key}
\oint_A^B \dfrac{\dbar Q}{T} \leq \int_A^B \dfrac{\dbar Q_R}{T}
\end{equation}
\begin{equation}\label{key}
\dd S  = \int_A^B \dfrac{\dbar Q_R}{T} \geq \oint_A^B \dfrac{\dbar Q}{T}
\end{equation}
For isolated system, $\dbar Q = 0$, thus
\begin{equation}\label{key}
\dd S \geq 0
\end{equation}

\subsection{Revisit First Law}
For reversible process
\begin{equation}\label{key}
\dd U = T\dd S - p\dd V
\end{equation}
-------------------------------------~
\paragraph{Attention: Also ok for irreversible process}~\\
For irreversible process
\begin{equation}\label{key}
\dbar Q \leq T\dd S \quad \dbar W \geq -p\dd V
\end{equation}
------------------------------------~\\
thus 
\begin{equation}\label{T=pdvU}
T = \qty(\pdv{U}{S})_V
\end{equation}
\begin{equation}\label{key}
p = -\qty(\pdv{U}{V})_S
\end{equation}
\begin{equation}\label{key}
\dfrac{p}{T} = -\qty(\pdv{U}{V})_S\qty(\pdv{S}{U})_V = \qty(\pdv{S}{V})_U
\end{equation}

\subsection{Joule Expansion (Irreversible)}
isol sys, ideal gas. $V_0 \ra 2V_0$
\begin{equation}\label{key}
p_i V_0 = RT_i
\end{equation}
\begin{equation}\label{key}
p_f\cdot 2V_0 = RT_f
\end{equation}
isol \txra
\begin{equation}\label{key}
T_i = T_f
\end{equation}
thus
\begin{equation}\label{key}
p_f = \dfrac{p_i}{2}
\end{equation}
Since $\dd U = 0$, thus
\begin{equation}\label{key}
T\dd S = p\dd V
\end{equation}
\begin{equation}\label{key}
\Delta S = \int_{V_0}^{2V_0}\dfrac{p}{T}\dd V = \int_{V_0}^{2V_0}\dfrac{R}{V}\dd V = R\ln 2
\end{equation}

\subsection{Statistical Basis for Entropy}
with $\eqref{T=pdvU}$, we have
\begin{equation}\label{key}
\dfrac{1}{T} = \qty(\pdv{S}{U})_V
\end{equation}
consider
\begin{equation}\label{key}
\dfrac{1}{k_B T} = \dv{\ln\Omega}{E}
\end{equation}
Def
\begin{equation}\label{key}
S = k_B \ln\Omega
\end{equation}

\subsection{Entropy of Mixing}
gas1, p, T, xV, xN\\
gas2, p ,T, (1-x)V, (1-x)N\\
\txra gas1,2 p, T, V, N
\begin{equation}\label{key}
\begin{aligned}
\Delta S &= \int_{xV}^V xNk_B\dfrac{\dd V}{V} + \int_{(1-x)V}^V (1-x)Nk_B\dfrac{\dd V}{V}\\
&=-Nk_B[x\ln x + (1-x)\ln(1-x)]
\end{aligned}
\end{equation}

\subsection{}
\subsection{Entropy and Probability}
A sys has $N$ micro-states (等概率), divided into some macro-states, each containing $n_i$ micro-states.
\begin{equation}\label{key}
\sum_i n_i = N
\end{equation}
thus
\begin{equation}\label{key}
P_i = \dfrac{n_i}{N}
\end{equation}
Entropy
\begin{equation}\label{key}
S_{tot} = S + S_{micro}
\end{equation}
where $S$ is connected to macro-states, which we can measure\\
$S_{micro}$ is connected to micro-states
\begin{equation}\label{key}
S_{micro} = \sum_i P_i S_i = \sum_i P_i k_B \ln n_i
\end{equation}
thus
\begin{equation}\label{key}
\begin{aligned}
S &= S_{tot} - S_{micro} = k_B \ln N - \sum_i P_i k_B \ln n_i\\
&=k_B\qty(\ln\dfrac{n_i}{P_i} - )
\end{aligned}
\end{equation}

\section{}

\section{Thermodynamic Potentials}
\subsection{Internal Energy}
\begin{equation}\label{key}
\dd U = T\dd S - p\dd V
\end{equation}
for isochoric process (等容过程)
\begin{equation}\label{key}
\dd U = T\dd S
\end{equation}
for reversible isochoric process
\begin{equation}\label{key}
\dd U = \dbar Q = C_V\dd T
\end{equation}

\subsection{Enthalpy}
\begin{equation}\label{key}
H \equiv U + pV
\end{equation}
thus
\begin{equation}\label{key}
\dd H = T\dd S + V\dd p
\end{equation}
For isobaric process (等压过程)
\begin{equation}\label{key}
\dd H = T\dd S
\end{equation}
for reversible isobaric process
\begin{equation}\label{key}
\dd H = \dbar Q_R = C_p\dd T
\end{equation}

\subsection{Helmholtz Function}
\begin{equation}\label{key}
F \equiv U - TS
\end{equation}
thus
\begin{equation}\label{key}
\dd F = -S\dd T -p\dd V
\end{equation}

\subsection{Gibbs Function}
\begin{equation}\label{key}
G \equiv H - TS = U + pV - TS
\end{equation}
thus 
\begin{equation}\label{key}
\dd G = -S\dd T + V\dd p
\end{equation}

---------------------------------~\\
\paragraph{Gibbs-Helmholtz Equations}
\begin{equation}\label{key}
\begin{aligned}
U &= F + TS = F - T\qty(\pdv{F}{T})_V\\
 &= T^2 \dfrac{F - T\qty(\pdv{F}{T})_V}{T^2} = -T^2\qty(\pdv{T}\dfrac{F}{T})_V
\end{aligned}
\end{equation}
\begin{equation}\label{key}
H = -T^2\qty(\pdv{T}\dfrac{G}{T})_p
\end{equation}

\subsection{}

\subsection{Maxwell Relations}
\begin{equation}\label{key}
\qty(\pdv{T}{V})_S = -\qty(\pdv{p}{S})_V
\end{equation}
\begin{equation}\label{key}
\qty(\pdv{T}{p})_S = \qty(\pdv{V}{S})_p
\end{equation}
\begin{equation}\label{key}
\qty(\pdv{S}{V})_T = \qty(\pdv{p}{T})_V
\end{equation}
\begin{equation}\label{key}
\qty(\pdv{S}{p})_T = -\qty(\pdv{V}{T})_p
\end{equation}

E.g. 16.4
\begin{equation}\label{key}
C_V = \qty(\pdv{Q}{T})_V = T\qty(\pdv{S}{T})_V
\end{equation}
\begin{equation}\label{key}
\qty(\pdv{C_V}{V})_T = \qty(\pdv{V}T\qty(\pdv{S}{T})_V)_T = T\qty(\pdv{T}\qty(\pdv{S}{V})_T)_V = T\qty(\pdv[2]{p}{T})_V
\end{equation}
Similarly
\begin{equation}\label{key}
\qty(\pdv{C_p}{p})_T = -T\qty(\pdv[2]{V}{T})_p
\end{equation}

isobaric expansivity (等压膨胀率)
\begin{equation}\label{key}
\beta_p = \dfrac{1}{V}\qty(\pdv{V}{T})_p
\end{equation}
adiabatic expansivity (绝热膨胀率)
\begin{equation}\label{key}
\beta_S = \dfrac{1}{V}\qty(\pdv{V}{T})_S
\end{equation}




\end{document}
