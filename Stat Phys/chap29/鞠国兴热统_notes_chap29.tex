\documentclass[UTF8]{ctexart} % use larger type; default would be 10pt

\usepackage[utf8]{inputenc} % set input encoding (not needed with XeLaTeX)

%%% Examples of Article customizations
% These packages are optional, depending whether you want the features they provide.
% See the LaTeX Companion or other references for full information.

%%% PAGE DIMENSIONS
\usepackage{geometry} % to change the page dimensions
\geometry{a4paper} % or letterpaper (US) or a5paper or....
% \geometry{margin=2in} % for example, change the margins to 2 inches all round
% \geometry{landscape} % set up the page for landscape
%   read geometry.pdf for detailed page layout information

\usepackage{graphicx} % support the \includegraphics command and options

% \usepackage[parfill]{parskip} % Activate to begin paragraphs with an empty line rather than an indent

%%% PACKAGES
\usepackage{booktabs} % for much better looking tables
\usepackage{array} % for better arrays (eg matrices) in maths
\usepackage{paralist} % very flexible & customisable lists (eg. enumerate/itemize, etc.)
\usepackage{verbatim} % adds environment for commenting out blocks of text & for better verbatim
\usepackage{subfig} % make it possible to include more than one captioned figure/table in a single float
% These packages are all incorporated in the memoir class to one degree or another...

%%% HEADERS & FOOTERS
\usepackage{fancyhdr} % This should be set AFTER setting up the page geometry
\pagestyle{fancy} % options: empty , plain , fancy
\renewcommand{\headrulewidth}{0pt} % customise the layout...
\lhead{}\chead{}\rhead{}
\lfoot{}\cfoot{\thepage}\rfoot{}

%%% SECTION TITLE APPEARANCE
\usepackage{sectsty}
\allsectionsfont{\sffamily\mdseries\upshape} % (See the fntguide.pdf for font help)
% (This matches ConTeXt defaults)

%%% ToC (table of contents) APPEARANCE
\usepackage[nottoc,notlof,notlot]{tocbibind} % Put the bibliography in the ToC
\usepackage[titles,subfigure]{tocloft} % Alter the style of the Table of Contents
\renewcommand{\cftsecfont}{\rmfamily\mdseries\upshape}
\renewcommand{\cftsecpagefont}{\rmfamily\mdseries\upshape} % No bold!

%%% END Article customizations

%%% The "real" document content comes below...

\setlength{\parindent}{0pt}
\usepackage{physics}
\usepackage{amsmath}
%\usepackage{symbols}
\usepackage{AMSFonts}
\usepackage{bm}
%\usepackage{eucal}
\usepackage{mathrsfs}
\usepackage{amssymb}
\usepackage{float}
\usepackage{multicol}
\usepackage{abstract}
\usepackage{empheq}
\usepackage{extarrows}
\usepackage{fontenc}
\usepackage{textcomp}


\DeclareMathOperator{\p}{\prime}
\DeclareMathOperator{\ti}{\times}
\DeclareMathOperator{\s}{\sum_{n=1}^{\infty}}
\DeclareMathOperator{\intinf}{\int_0^\infty}
\DeclareMathOperator{\intdinf}{\int_{-\infty}^\infty}
\DeclareMathOperator{\suminf}{\sum_{n=0}^\infty}
\DeclareMathOperator{\sumin}{\sum_{i=1}^n}
\DeclareMathOperator{\e}{\mathrm{e}}
\renewcommand{\I}{\mathrm{i}}
\DeclareMathOperator{\llra}{\longleftrightarrow}
\DeclareMathOperator{\lra}{\longrightarrow}
\DeclareMathOperator{\dlra}{\Leftrightarrow}
\DeclareMathOperator{\dra}{\Rightarrow}
\newcommand{\txra}{\textrightarrow}
\newcommand{\dis}{\displaystyle}
\newcommand{\dbar}{{\mathchar'26\mkern-11mu\mathrm{d}}}
\numberwithin{equation}{subsection}

\title{Notes of JU Guoxing TD\&SP}
\author{hebrewsnabla}
%\date{} % Activate to display a given date or no date (if empty),
         % otherwise the current date is printed 

\begin{document}
% \boldmath
\maketitle
\setcounter{section}{28}
\section{Bose-Einstein and Fermi-Dirac Distributions}
\subsection{}
\subsection{Wave Function of Identical Particles}
\subsection{The Statistics of Identical Particles}
suppose energy of each particle is $E$
\begin{equation}\label{key}
\mathcal{Z} = \sum_N \sum_\alpha \e^{\beta(\mu N - E_\alpha)} = \sum_N \e^{N\beta(\mu - E)}
\end{equation}
thus
\begin{equation}\label{key}
\langle n\rangle = \dfrac{\sum_N N\e^{N\beta(\mu - E)}}{\sum_N \e^{N\beta(\mu - E)}} = -\dfrac{1}{\beta \mathcal{Z}}\pdv{\mathcal{Z}}{E} = -\dfrac{1}{\beta}\pdv{\ln\mathcal{Z}}{E}
\end{equation}
for fermions
\begin{equation}\label{Z_fer}
\mathcal{Z} = \sum_{N=0}^1 \e^{N\beta(\mu - E)} = 1 + \e^{\beta(\mu - E)}
\end{equation}
Fermi-Dirac distribution fxn
\begin{equation}\label{key}
f_D(E) \equiv \langle n\rangle = \dfrac{\e^{\beta(\mu - E)}}{1 + \e^{\beta(\mu - E)}} = \dfrac{1}{\e^{\beta(E-\mu) + 1}}
\end{equation}
for bosons
\begin{equation}\label{Z_bos}
\mathcal{Z} = \sum_{N=0}^\infty \e^{N\beta(\mu - E)} = \dfrac{1}{1 - \e^{\beta(\mu - E)}}
\end{equation}
Bose-Einstein distribution fxn
\begin{equation}\label{key}
f_B(E) \equiv \langle n\rangle = \dfrac{\e^{\beta(\mu - E)}}{1 - \e^{\beta(\mu - E)}} = \dfrac{1}{\e^{\beta(E-\mu)} - 1}
\end{equation}

\section{Quantum Gases and Condensates}
\subsection{non-interacting q fluid}
\begin{equation}\label{key}
\mathcal{Z} = \prod_k \mathcal{Z}_k^{2S+1}
\end{equation}
where $\mathcal{Z}_k$ refers to $\eqref{Z_fer}\eqref{Z_bos}$\\
i.e.
\begin{equation}\label{key}
\mathcal{Z}_k = (1\pm\e^{(\mu-E)})^{\pm 1}
\end{equation}
grand potential
\begin{equation}\label{key}
\begin{aligned}
\Phi &= -\dfrac{1}{\beta}\ln\mathcal{Z}\\
&= \mp \dfrac{2S + 1}{\beta}\sum_k\ln(1\pm\e^{(\mu-E)})\\
&= \mp \dfrac{1}{\beta} \intinf \ln(1\pm\e^{(\mu-E)})g(E)\dd E
\end{aligned}
\end{equation}
\begin{equation}\label{key}
g(k)\dd k = \dfrac{4\pi k^2\dd k}{(2\pi/L)^3}(2S + 1) = \dfrac{(2S+1)Vk^2\dd k}{2\pi^2}
\end{equation}
\begin{equation}\label{key}
g(E)\dd E = \dfrac{(2S+1)VE^{1/2}\dd E}{4\pi^2}\qty(\dfrac{2m}{\hbar^2})^{3/2}
\end{equation}
thus
\begin{equation}\label{key}
\begin{aligned}
\Phi &= \mp\dfrac{(2S+1)V}{4\pi^2\beta} \qty(\dfrac{2m}{\hbar^2})^{3/2} \intinf \ln(1\pm\e^{\beta(\mu-E)})E^{1/2}\dd E\\
&= \mp\dfrac{(2S+1)V}{4\pi^2\beta} \qty(\dfrac{2m}{\hbar^2})^{3/2} \intinf \ln(1\pm\e^{\beta(\mu-E)})\dfrac{2}{3}\dd(E^{3/2})\\
&= \mp\dfrac{2}{3}\dfrac{(2S+1)V}{4\pi^2\beta} \qty(\dfrac{2m}{\hbar^2})^{3/2} \qty{\qty[\ln(1\pm\e^{\beta(\mu-E)})E^{3/2}]\Big|_0^\infty - \intinf E^{3/2}\dfrac{\mp\beta\e^{\beta(\mu - E)}}{1\pm\e^{\beta(\mu-E)}}\dd E}\\
&= \mp\dfrac{2}{3}\dfrac{(2S+1)V}{4\pi^2\beta} \qty(\dfrac{2m}{\hbar^2})^{3/2} \qty{ 0 \pm \beta\intinf E^{3/2}\dfrac{1}{\e^{\beta(E-\mu)} \pm 1}\dd E }\\
&= \dfrac{2}{3}\dfrac{(2S+1)V}{4\pi^2} \qty(\dfrac{2m}{\hbar^2})^{3/2} \intinf\dfrac{E^{3/2}}{\e^{\beta(E-\mu)} \pm 1}\dd E 
\end{aligned}
\end{equation}



\end{document}
