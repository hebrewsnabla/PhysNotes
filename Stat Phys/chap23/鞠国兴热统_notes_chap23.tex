\documentclass[UTF8]{ctexart} % use larger type; default would be 10pt

\usepackage[utf8]{inputenc} % set input encoding (not needed with XeLaTeX)

%%% Examples of Article customizations
% These packages are optional, depending whether you want the features they provide.
% See the LaTeX Companion or other references for full information.

%%% PAGE DIMENSIONS
\usepackage{geometry} % to change the page dimensions
\geometry{a4paper} % or letterpaper (US) or a5paper or....
% \geometry{margin=2in} % for example, change the margins to 2 inches all round
% \geometry{landscape} % set up the page for landscape
%   read geometry.pdf for detailed page layout information

\usepackage{graphicx} % support the \includegraphics command and options

% \usepackage[parfill]{parskip} % Activate to begin paragraphs with an empty line rather than an indent

%%% PACKAGES
\usepackage{booktabs} % for much better looking tables
\usepackage{array} % for better arrays (eg matrices) in maths
\usepackage{paralist} % very flexible & customisable lists (eg. enumerate/itemize, etc.)
\usepackage{verbatim} % adds environment for commenting out blocks of text & for better verbatim
\usepackage{subfig} % make it possible to include more than one captioned figure/table in a single float
% These packages are all incorporated in the memoir class to one degree or another...

%%% HEADERS & FOOTERS
\usepackage{fancyhdr} % This should be set AFTER setting up the page geometry
\pagestyle{fancy} % options: empty , plain , fancy
\renewcommand{\headrulewidth}{0pt} % customise the layout...
\lhead{}\chead{}\rhead{}
\lfoot{}\cfoot{\thepage}\rfoot{}

%%% SECTION TITLE APPEARANCE
\usepackage{sectsty}
\allsectionsfont{\sffamily\mdseries\upshape} % (See the fntguide.pdf for font help)
% (This matches ConTeXt defaults)

%%% ToC (table of contents) APPEARANCE
\usepackage[nottoc,notlof,notlot]{tocbibind} % Put the bibliography in the ToC
\usepackage[titles,subfigure]{tocloft} % Alter the style of the Table of Contents
\renewcommand{\cftsecfont}{\rmfamily\mdseries\upshape}
\renewcommand{\cftsecpagefont}{\rmfamily\mdseries\upshape} % No bold!

%%% END Article customizations

%%% The "real" document content comes below...

\setlength{\parindent}{0pt}
\usepackage{physics}
\usepackage{amsmath}
%\usepackage{symbols}
\usepackage{AMSFonts}
\usepackage{bm}
%\usepackage{eucal}
\usepackage{mathrsfs}
\usepackage{amssymb}
\usepackage{float}
\usepackage{multicol}
\usepackage{abstract}
\usepackage{empheq}
\usepackage{extarrows}
\usepackage{fontenc}
\usepackage{textcomp}


\DeclareMathOperator{\p}{\prime}
\DeclareMathOperator{\ti}{\times}
\DeclareMathOperator{\s}{\sum_{n=1}^{\infty}}
\DeclareMathOperator{\intinf}{\int_0^\infty}
\DeclareMathOperator{\intdinf}{\int_{-\infty}^\infty}
\DeclareMathOperator{\suminf}{\sum_{n=0}^\infty}
\DeclareMathOperator{\sumin}{\sum_{i=1}^n}
\DeclareMathOperator{\e}{\mathrm{e}}
\renewcommand{\I}{\mathrm{i}}
\DeclareMathOperator{\llra}{\longleftrightarrow}
\DeclareMathOperator{\lra}{\longrightarrow}
\DeclareMathOperator{\dlra}{\Leftrightarrow}
\DeclareMathOperator{\dra}{\Rightarrow}
\newcommand{\txra}{\textrightarrow}
\newcommand{\dis}{\displaystyle}
\newcommand{\dbar}{{\mathchar'26\mkern-11mu\mathrm{d}}}
\numberwithin{equation}{subsection}

\title{Notes of JU Guoxing TD\&SP}
\author{hebrewsnabla}
%\date{} % Activate to display a given date or no date (if empty),
         % otherwise the current date is printed 

\begin{document}
% \boldmath
\maketitle
\setcounter{section}{22}
\section{Photon}
\subsection{Classic Thermodynamics of Electromagnetic Radiation}
Energy density
\begin{equation}\label{key}
u = \dfrac{U}{V} = n\hbar\omega
\end{equation}
pressure
\begin{equation}\label{key}
pV = \dfrac{1}{3}Nm\langle v^2\rangle
\end{equation}
for photon
\begin{equation}\label{key}
p = \dfrac{1}{3}nmc^2
\end{equation}

\section{Phonon}
\subsection{Einstein Model}

\subsection{Debye Model}

\section{Relativistic Gas}
\subsection{}
\subsection{Ultrarelativistic Gas}
\begin{equation}\label{key}
E = pc = \hbar kc
\end{equation}
single-particle partition function
\begin{equation}\label{key}
Z_1 = \intinf\e^{-\beta\hbar kc}g(k)\dd k
\end{equation}
where
\begin{equation}\label{key}
g(k)\dd k = \dfrac{Vk^2\dd k}{2\pi^2}
\end{equation}
Let $x = \beta\hbar kc$
\begin{equation}\label{key}
Z_1 = \dfrac{V}{2\pi^2}\qty(\dfrac{1}{\beta\hbar c})^3\intinf\e^{-x}x^2\dd x = \dfrac{V}{\pi^2}\qty(\dfrac{1}{\beta\hbar c})^3
\end{equation}
or $Z_1 = \dfrac{V}{\Lambda^3}$\\
~\\
N-particle partition fxn
\begin{equation}\label{key}
Z_N = \dfrac{Z_1^N}{N!}
\end{equation}

\subsection{Adiabatic Expansion}
\begin{equation}\label{key}
VT^3 = Cons.
\end{equation}


\section{Real Gases}
\subsection{vdW Gas}
\begin{equation}\label{key}
\qty(p + \dfrac{a}{V_m^2})(V_m - b) = RT
\end{equation}
critical point (临界点)
\begin{equation}\label{key}
\qty(\pdv{p}{V}) = -\dfrac{RT}{(V - b)^2} + \dfrac{2a}{V^3} = 0
\end{equation}
\begin{equation}\label{key}
\qty(\pdv[2]{p}{V}) = ...
\end{equation}
...\\
critical vol
\begin{equation}\label{key}
V_c = 3b
\end{equation}
critical temp
\begin{equation}\label{key}
T_c = \dfrac{8a}{27Rb}
\end{equation}
critical pres
\begin{equation}\label{key}
p_c = \dfrac{a}{27b^2}
\end{equation}
thus
\begin{equation}\label{key}
\dfrac{p_c V_c}{RT_c} = \dfrac{3}{8}
\end{equation}




\subsection{Dieterici Equation}
vdW Eq.
\begin{equation}\label{key}
\begin{aligned}
p &= \dfrac{RT}{V -b} - \dfrac{a}{V_m^2}\\
&\equiv p_{rep} + p_{attr}
\end{aligned}
\end{equation}
Dieterici Eq.
\begin{equation}\label{key}
p = p_{rep}\e^{-\dfrac{a}{RTV}}
\end{equation}

\subsection{Virial Expansion}
\begin{equation}\label{key}
\dfrac{pV_m}{RT} = 1 + \dfrac{B}{V_m} + \dfrac{C}{V_m^2} + \cdots
\end{equation}
B, C, etc., are called virial coefficients.\\
Try to drive B:
\begin{equation}\label{key}
U = \sum_{i=1}^N\dfrac{p_i^2}{2m} + \sum_{i<j}V(\abs{\vb{r}_i - \vb{r}_j})
\end{equation}
\begin{equation}\label{key}
Z = \dfrac{1}{N!h^{3N}}
\end{equation}
...
\begin{equation}\label{key}
p = ...
\end{equation}

Using hard sphere model
\begin{equation}\label{key}
V(r) = \left\{
\begin{aligned}
&\infty & r<r_0\\
&-\varepsilon\qty(\dfrac{r_0}{r})^6 & r>r_0
\end{aligned}\right.
\end{equation}
we have
\begin{equation}\label{key}
B = b - \dfrac{a\beta}{N}
\end{equation}
where
\begin{equation}\label{key}
a = ... \quad b = ...
\end{equation}

\subsection{The Law of Corresponding States}
Def
\begin{equation}\label{key}
\tilde{p} = \dfrac{p}{p_c},\quad \tilde{V} = \dfrac{V}{V_c},\quad \tilde{T} = \dfrac{T}{T_c}
\end{equation}
thus
\begin{equation}\label{key}
\qty(\tilde{p} + \dfrac{3}{\tilde{V}^2}) = \dfrac{8\tilde{T}}{3\tilde{V} - 1}
\end{equation}


\section{Cooling Real Gas}
\subsection{The Joule Expansion}
U unchanged. Irreversible.\\
Joule coefficient
\begin{equation}\label{key}
\mu_J = \qty(\pdv{T}{V})_U
\end{equation}
thus
\begin{equation}\label{key}
\mu_J = -\qty(\pdv{T}{U})_V\qty(\pdv{U}{V})_T = -\dfrac{1}{C_V}\qty[T\qty(\pdv{p}{T})_V - p]
\end{equation}
For ideal gas
\begin{equation}\label{key}
\mu_J = 0
\end{equation}
For vdW gas
\begin{equation}\label{key}
\mu_J = -\dfrac{a}{C_V V^2}
\end{equation}
\begin{equation}\label{key}
\Delta T = ...
\end{equation}

\subsection{Isothermal Expansion}
\begin{equation}\label{key}
\qty(\pdv{U}{V})_T = T\qty(\pdv{p}{T})_V - p
\end{equation}
\begin{equation}\label{key}
\Delta U = \int_{V_1}^{V_2}\qty[T\qty(\pdv{p}{T})_V - p]\dd V
\end{equation}
For ideal gas
\begin{equation}\label{key}
\Delta U = 0
\end{equation}
For vdW gas
\begin{equation}\label{key}
...
\end{equation}

\subsection{Joule-Kelvin Expansion}
aka Joule-Thomson expansion.


\section{Phase Transitions}
\subsection{Latent Heat}

\subsection{Chemical Potential and Phase Changes}
Phase equilibrium
\begin{equation}\label{key}
\mu_1 = \mu_2
\end{equation}

\subsection{Clausius-Clapeyron Equation}
\begin{equation}\label{key}
\mu_1(p,T) = \mu_2(p,T)
\end{equation}
\begin{equation}\label{key}
\dd\mu_1 = \dd\mu_2
\end{equation}
with Gibbs-Duhem Eq.:
\begin{equation}\label{key}
\dd\mu = - s\dd T + v\dd p
\end{equation}
where $s = \dfrac{S}{N},\; v = \dfrac{V}{N}$\\
thus
\begin{equation}\label{key}
-s_1\dd T + v_1\dd p = -s_2\dd T + v_2\dd p
\end{equation}
\begin{equation}\label{key}
\dv{p}{T} = \dfrac{s_2 - s_1}{v_2 - v_1}
\end{equation}
Def latent per particle
\begin{equation}\label{key}
\ell = T\Delta s
\end{equation}
thus
\begin{equation}\label{key}
\dv{p}{T} = \dfrac{\ell}{T(v_2 - v_1)}
\end{equation}
or
\begin{equation}\label{key}
\dv{p}{T} = \dfrac{L}{T(V_2 - V_1)}
\end{equation}





\end{document}
