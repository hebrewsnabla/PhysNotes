\documentclass[UTF8]{ctexart} % use larger type; default would be 10pt

\usepackage[utf8]{inputenc} % set input encoding (not needed with XeLaTeX)

%%% Examples of Article customizations
% These packages are optional, depending whether you want the features they provide.
% See the LaTeX Companion or other references for full information.

%%% PAGE DIMENSIONS
\usepackage{geometry} % to change the page dimensions
\geometry{a4paper} % or letterpaper (US) or a5paper or....
% \geometry{margin=2in} % for example, change the margins to 2 inches all round
% \geometry{landscape} % set up the page for landscape
%   read geometry.pdf for detailed page layout information

\usepackage{graphicx} % support the \includegraphics command and options

% \usepackage[parfill]{parskip} % Activate to begin paragraphs with an empty line rather than an indent

%%% PACKAGES
\usepackage{booktabs} % for much better looking tables
\usepackage{array} % for better arrays (eg matrices) in maths
\usepackage{paralist} % very flexible & customisable lists (eg. enumerate/itemize, etc.)
\usepackage{verbatim} % adds environment for commenting out blocks of text & for better verbatim
\usepackage{subfig} % make it possible to include more than one captioned figure/table in a single float
% These packages are all incorporated in the memoir class to one degree or another...

%%% HEADERS & FOOTERS
\usepackage{fancyhdr} % This should be set AFTER setting up the page geometry
\pagestyle{fancy} % options: empty , plain , fancy
\renewcommand{\headrulewidth}{0pt} % customise the layout...
\lhead{}\chead{}\rhead{}
\lfoot{}\cfoot{\thepage}\rfoot{}

%%% SECTION TITLE APPEARANCE
\usepackage{sectsty}
\allsectionsfont{\sffamily\mdseries\upshape} % (See the fntguide.pdf for font help)
% (This matches ConTeXt defaults)

%%% ToC (table of contents) APPEARANCE
\usepackage[nottoc,notlof,notlot]{tocbibind} % Put the bibliography in the ToC
\usepackage[titles,subfigure]{tocloft} % Alter the style of the Table of Contents
\renewcommand{\cftsecfont}{\rmfamily\mdseries\upshape}
\renewcommand{\cftsecpagefont}{\rmfamily\mdseries\upshape} % No bold!

%%% END Article customizations

%%% The "real" document content comes below...

\setlength{\parindent}{0pt}
\usepackage{physics}
\usepackage{amsmath}
%\usepackage{symbols}
\usepackage{AMSFonts}
\usepackage{bm}
%\usepackage{eucal}
\usepackage{mathrsfs}
\usepackage{amssymb}
\usepackage{float}
\usepackage{multicol}
\usepackage{abstract}
\usepackage{empheq}
\usepackage{extarrows}
\usepackage{fontenc}


\DeclareMathOperator{\p}{\prime}
\DeclareMathOperator{\ti}{\times}
\DeclareMathOperator{\s}{\sum_{n=1}^{\infty}}
\DeclareMathOperator{\intinf}{\int_0^\infty}
\DeclareMathOperator{\intdinf}{\int_{-\infty}^\infty}
\DeclareMathOperator{\suminf}{\sum_{n=0}^\infty}
\DeclareMathOperator{\sumin}{\sum_{i=1}^n}
\DeclareMathOperator{\e}{\mathrm{e}}
\renewcommand{\I}{\mathrm{i}}
\DeclareMathOperator{\llra}{\longleftrightarrow}
\DeclareMathOperator{\lra}{\longrightarrow}
\DeclareMathOperator{\dlra}{\Leftrightarrow}
\DeclareMathOperator{\dra}{\Rightarrow}
\newcommand{\dis}{\displaystyle}
\numberwithin{equation}{subsection}

\title{Notes of JU Guoxing TD\&SP}
\author{hebrewsnabla}
%\date{} % Activate to display a given date or no date (if empty),
         % otherwise the current date is printed 

\begin{document}
% \boldmath
\maketitle
\section{Introduction}
热力学量:formal additivity (形式可加性) + 均匀性\\
与物质量(m,n,N)的关系:\\
extensive quantity (广延量)\\
intensive quantity (强度量)\\
与过程的关系:\\
过程量  $\dd W,\;-p\dd V,\;\dd Q,\;T\dd S$\\
状态量

\section{Heat}

\setcounter{subsection}{1}
\subsection{Heat Capacity}
def:
\begin{equation}\label{key}
C=\dv{Q}{T}
\end{equation}
Heat capacity is a kind of response function (响应函数,体现系统对外界作用的响应情况). 与过程有关.\\

Other response fxn involving state fxn ($f(T,V,p)=0,\;V=V(T,p),\;p=p(T,V)$)\\
1) isobaric expansivity (定压膨胀系数)
\begin{equation}\label{key}
\alpha=\dfrac{1}{V}\qty(\pdv{V}{T})_p
\end{equation}
2) isochoric pressure coefficient\\
3) isothermal compressibility\\
\begin{equation}\label{key}
\kappa_T=-\dfrac{1}{V}\qty(\pdv{V}{p})_T
\end{equation}
and there exists a connection between them
\begin{equation}\label{key}
\alpha=\kappa_T\beta p
\end{equation}
e.g.\\

~\\
Homework\\
1.3, 1.5, 2.2, 2.5\\
Proof\\
For $f(x,y,z)=0$\\
(i)
\begin{equation}\label{key}
\qty(\pdv{x}{y})_z\qty(\pdv{y}{z})_x\qty(\pdv{z}{x})_y=-1
\end{equation}
(ii)
\begin{equation}\label{key}
\qty(\pdv{x}{y})_z=\dfrac{1}{\qty(\pdv{y}{x})_z}
\end{equation}
赵凯华《定性与半定量物理学》\\
Heat capacity per mass unit (specific heat capacity)
\begin{equation}\label{key}
c=\dfrac{1}{m}\dv{Q}{T}
\end{equation}
\begin{equation}\label{key}
C_V=\qty(\pdv{Q}{T})_V
\end{equation}
\begin{equation}\label{key}
C_p=\qty(\pdv{Q}{T})_p
\end{equation}
\section{Probability}
\setcounter{subsection}{3}
\subsection{variance}
def: variance
\begin{equation}\label{key}
\sigma_x^2=\langle(x-\langle x\rangle)^2\rangle
\end{equation}
standard deviation
\begin{equation}\label{key}
\sigma_x=\sqrt{\langle(x-\langle x\rangle)^2\rangle}
\end{equation}
Discussion:
1) 
\begin{equation}\label{key}
\begin{aligned}
\sigma_x^2&=\langle x^2-2x\langle x\rangle+\langle x\rangle^2\rangle\\
&=\langle x^2\rangle-\langle x\rangle^2
\end{aligned}
\end{equation}
2) k-degree moment: $\langle(x-\langle x\rangle)^k\rangle$\\
k=1		average deviation\\
k=2		variance\\
k=3		skewness(偏斜度)\\
k=4		kurtosis(峭度)\\
\subsection{Linear Transform and Variance}
suppose
\begin{equation}\label{key}
y=ax+b
\end{equation}
where $a$ and $b$ are constants\\
we have
\begin{equation}\label{key}
\langle y\rangle=a\langle x\rangle+b
\end{equation}
thus
\begin{equation}\label{key}
\langle y^2\rangle=\langle a^2 x^2 + 2abx + b^2\rangle=a^2\langle x^2\rangle+2ab\langle x\rangle+b^2
\end{equation}
\begin{equation}\label{key}
\langle y\rangle^2=a^2\langle x\rangle^2+2ab\langle x\rangle+b^2
\end{equation}
\begin{equation}\label{key}
\sigma_y^2=a^2\langle x^2\rangle-a^2\langle x\rangle^2=a^2\sigma_x^2
\end{equation}
\begin{equation}\label{key}
\sigma_y=a\sigma_x
\end{equation}
\subsection{independent variable}
Suppose $u$ and $v$ are independent random variables, the probability of $u\sim u+\dd{u}$ and $v\sim v+\dd{v}$ is
\begin{equation}\label{key}
P_u(u)\dd{u}P_v(v)\dd{v}
\end{equation}
\begin{equation}\label{key}
\begin{aligned}
\langle uv\rangle&=\iint uv P_u(u)P_v(v)\dd{u}\dd{v}\\
&=\int u P_u(u)\dd{u}\int vP_v(v)\dd{v}\\
&=\langle u\rangle\langle v\rangle
\end{aligned}
\end{equation}
Suppose we have $n$ independent random variables $X_i$, all with average $\langle X\rangle$ and variance $\sigma_X^2$, and $Y=X_1+\cdots+X_n$, show the average and variance of $Y$.
\begin{equation}\label{key}
\langle Y\rangle=n\langle X\rangle
\end{equation}
\begin{equation}\label{key}
\sigma_Y^2=\langle Y^2\rangle-\langle Y\rangle^2
\end{equation}
where
\begin{equation}\label{key}
\begin{aligned}
\langle Y^2\rangle &=\langle X_1^2+\cdots+X_n^2+2X_1X_2+\cdots\rangle\\
&=n\langle X^2\rangle+n(n-1)\langle X\rangle^2
\end{aligned}
\end{equation}
thus
\begin{equation}\label{key}
\sigma_Y^2=n\langle X^2\rangle-n\langle X\rangle^2=n\sigma_X^2
\end{equation}
\begin{equation}\label{key}
\sigma_Y=\sqrt{n}\sigma_X
\end{equation}
which means, suppose $average(x)=\dfrac{\sumin X_i}{n}$
\begin{equation}\label{key}
\sigma_{average(x)}=\dfrac{\sqrt{n}\sigma_X}{n}=\dfrac{\sigma_X}{\sqrt{n}}
\end{equation}

\subsection{Binomial Distribution}
\paragraph{Bernoulli Trial}
\begin{equation}\label{key}
P(x)=\left\{
	\mqty{p &"success"\\1-p &"fail"}\right.
\end{equation}
\paragraph{Binomial Distribution}~\\
prob of $k$ successes in $n$ independent trials:
\begin{equation}\label{key}
P(n,k)=C_n^k p^k (1-p)^{n-k}
\end{equation}
\begin{equation}\label{key}
\langle k\rangle=np
\end{equation}
\begin{equation}\label{key}
\sigma_k^2=np(1-p)
\end{equation}
fractional width (相对宽度) of the distribution:
\begin{equation}\label{key}
\dfrac{\sigma_k}{\langle k\rangle}=\sqrt{\dfrac{1-p}{np}}\propto\dfrac{1}{\sqrt{n}}
\end{equation}
\paragraph{Poisson Distribution}

\paragraph{Exponential Distribution}

\paragraph{Moment Generating Function}


\end{document}
