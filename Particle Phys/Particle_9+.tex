%\documentclass[UTF8]{ctexart} % use larger type; default would be 10pt
\documentclass[a4paper]{article}
%\usepackage{xeCJK}
%\usepackage{ctex}
\usepackage{luatexja-fontspec}
\setmainjfont{FandolSong}
%\usepackage[utf8]{inputenc} % set input encoding (not needed with XeLaTeX)

%%% Examples of Article customizations
% These packages are optional, depending whether you want the features they provide.
% See the LaTeX Companion or other references for full information.

%%% PAGE DIMENSIONS
\usepackage{geometry} % to change the page dimensions
\geometry{a4paper} % or letterpaper (US) or a5paper or....
% \geometry{margin=2in} % for example, change the margins to 2 inches all round
% \geometry{landscape} % set up the page for landscape
%   read geometry.pdf for detailed page layout information

\usepackage{graphicx} % support the \includegraphics command and options

% \usepackage[parfill]{parskip} % Activate to begin paragraphs with an empty line rather than an indent

%%% PACKAGES
\usepackage{booktabs} % for much better looking tables
\usepackage{array} % for better arrays (eg matrices) in maths
\usepackage{paralist} % very flexible & customisable lists (eg. enumerate/itemize, etc.)
\usepackage{verbatim} % adds environment for commenting out blocks of text & for better verbatim
\usepackage{subfig} % make it possible to include more than one captioned figure/table in a single float
% These packages are all incorporated in the memoir class to one degree or another...

%%% HEADERS & FOOTERS
\usepackage{fancyhdr} % This should be set AFTER setting up the page geometry
\pagestyle{fancy} % options: empty , plain , fancy
\renewcommand{\headrulewidth}{0pt} % customise the layout...
\lhead{}\chead{}\rhead{}
\lfoot{}\cfoot{\thepage}\rfoot{}

%%% SECTION TITLE APPEARANCE
\usepackage{sectsty}
\allsectionsfont{\sffamily\mdseries\upshape} % (See the fntguide.pdf for font help)
% (This matches ConTeXt defaults)

%%% ToC (table of contents) APPEARANCE
\usepackage[nottoc,notlof,notlot]{tocbibind} % Put the bibliography in the ToC
\usepackage[titles,subfigure]{tocloft} % Alter the style of the Table of Contents
\renewcommand{\cftsecfont}{\rmfamily\mdseries\upshape}
\renewcommand{\cftsecpagefont}{\rmfamily\mdseries\upshape} % No bold!

%%% END Article customizations

%%% The "real" document content comes below...

\setlength{\parindent}{0pt}
\usepackage{physics}
\usepackage{amsmath}
%\usepackage{symbols}
\usepackage{AMSFonts}
\usepackage{bm}
%\usepackage{eucal}
\usepackage{mathrsfs}
\usepackage{amssymb}
\usepackage{float}
\usepackage{multicol}
\usepackage{abstract}
\usepackage{empheq}
\usepackage{extarrows}
\usepackage{textcomp}
\usepackage{mhchem}
\usepackage{braket}
\usepackage{siunitx}
\usepackage[utf8]{inputenc}
\usepackage{tikz-feynman}
\usepackage{feynmp}


\DeclareMathOperator{\p}{\prime}
\DeclareMathOperator{\ti}{\times}

\DeclareMathOperator{\e}{\mathrm{e}}
\DeclareMathOperator{\I}{\mathrm{i}}
\DeclareMathOperator{\Arg}{\mathrm{Arg}}
\newcommand{\NA}{N_\mathrm{A}}
\newcommand{\kB}{k_\mathrm{B}}

\DeclareMathOperator{\ra}{\rightarrow}
\DeclareMathOperator{\llra}{\longleftrightarrow}
\DeclareMathOperator{\lra}{\longrightarrow}
\DeclareMathOperator{\dlra}{\;\Leftrightarrow\;}
\DeclareMathOperator{\dra}{\;\Rightarrow\;}

%%%%%%%%%%%% QUANTUM MECHANICS %%%%%%%%%%%%%%%%%%%%%%%%
\newcommand{\bkk}[1]{\Braket{#1|#1}}
\newcommand{\bk}[2]{\Braket{#1|#2}}
\newcommand{\bkev}[2]{\Braket{#2|#1|#2}}

\DeclareMathOperator{\na}{\bm{\nabla}}
\DeclareMathOperator{\nna}{\nabla^2}
\DeclareMathOperator{\drrr}{\dd[3]\vb{r}}

\DeclareMathOperator{\psis}{\psi^\ast}
\DeclareMathOperator{\Psis}{\Psi^\ast}
\DeclareMathOperator{\hi}{\hat{\vb{i}}}
\DeclareMathOperator{\hj}{\hat{\vb{j}}}
\DeclareMathOperator{\hk}{\hat{\vb{k}}}
\DeclareMathOperator{\hr}{\hat{\vb{r}}}
\DeclareMathOperator{\hT}{\hat{\vb{T}}}
\DeclareMathOperator{\hH}{\hat{H}}

\DeclareMathOperator{\hL}{\hat{\vb{L}}}
\DeclareMathOperator{\hp}{\hat{\vb{p}}}
\DeclareMathOperator{\hx}{\hat{\vb{x}}}
\DeclareMathOperator{\ha}{\hat{\vb{a}}}
\DeclareMathOperator{\hS}{\hat{\vb{S}}}
\DeclareMathOperator{\hSigma}{\hat{\bm\Sigma}}
\DeclareMathOperator{\hJ}{\hat{\vb{J}}}

\DeclareMathOperator{\Tdv}{-\dfrac{\hbar^2}{2m}\dv[2]{x}}
\DeclareMathOperator{\Tna}{-\dfrac{\hbar^2}{2m}\nabla^2}

%\DeclareMathOperator{\s}{\sum_{n=1}^{\infty}}
\DeclareMathOperator{\intinf}{\int_0^\infty}
\DeclareMathOperator{\intdinf}{\int_{-\infty}^\infty}
%\DeclareMathOperator{\suminf}{\sum_{n=0}^\infty}
\DeclareMathOperator{\sumnzinf}{\sum_{n=0}^\infty}
\DeclareMathOperator{\sumnoinf}{\sum_{n=1}^\infty}
\DeclareMathOperator{\sumndinf}{\sum_{n=-\infty}^\infty}
\DeclareMathOperator{\sumizinf}{\sum_{i=0}^\infty}

%%%%%%%%%%%%%%%%% PARTICLE PHYSICS %%%%%%%%%%%%%%%%
\DeclareMathOperator{\hh}{\hat{h}}               % helicity
\DeclareMathOperator{\hP}{\hat{\vb{P}}}          % Parity
\DeclareMathOperator{\hU}{\hat{U}}
\DeclareMathOperator{\hG}{\hat{G}}

\DeclareMathOperator{\GeV}{\si{GeV}}
\DeclareMathOperator{\LI}{\mathscr{L}.I.}
%\DeclareMathOperator{\g5}{\gamma^5}
\DeclareMathOperator{\gmuu}{\gamma^\mu}
\DeclareMathOperator{\gmud}{\gamma_\mu}
\DeclareMathOperator{\gnuu}{\gamma^\nu}
\DeclareMathOperator{\gnud}{\gamma_\nu}

\renewcommand{\u}{\mathrm{u}}
\renewcommand{\d}{\mathrm{d}}
\DeclareMathOperator{\s}{\mathrm{s}}

\DeclareMathOperator{\q}{\mathrm{q}}
\DeclareMathOperator{\bq}{\bar{\mathrm{q}}}

\DeclareMathOperator{\g}{\mathrm{g}}
\DeclareMathOperator{\W}{\mathrm{W}}
\DeclareMathOperator{\Z}{\mathrm{Z}}

%%% Feynman Diagram
\newcommand{\pa}{particle}
\newcommand{\mo}{momentum}
\newcommand{\el}{edge label}

\newcommand{\dis}{\displaystyle}
\numberwithin{equation}{section}

\title{Notes of Particle Physics, LIU Zuowei}
\author{hebrewsnabla}
%\date{} % Activate to display a given date or no date (if empty),
         % otherwise the current date is printed 

\begin{document}
% \boldmath
\maketitle

\tableofcontents

\newpage

\setcounter{section}{8}
\section{Symmetries and the Quark Model}
$ \bigstar $
Standard Model can be written in group theory\\
EM -- U(1)\\
weak -- \\
strong -- SU(3)\\
EM, weak -- SU(2)
\subsection{Symm. in QM}
\begin{equation}\label{key}
\psi \ra \psi' = \hat{U}\psi
\end{equation}
\begin{equation}\label{key}
\bk{\psi}{\psi} = \bk{\psi'}{\psi'} = \Braket{\psi | \hat{U}^\dagger \hat{U} | \psi}
\end{equation}
thus $ \hat{U} $ must be unitary
\begin{equation}\label{key}
\hat{U}^\dagger \hat{U} = I
\end{equation}
...
\begin{equation}\label{key}
[\hH, \hat{U}] = 0
\end{equation}

A finite continuous symm operation can be built up from a series of infinitesimal transformations like
\begin{equation}\label{key}
\hU = I + \I\epsilon\hG
\end{equation}
$ \hG $: generator of the transformation 产生算符
\begin{equation}\label{key}
\hat{U}^\dagger \hat{U} = I \dra \hG^\dagger = \hG \;(\text{Hermitian})
\end{equation}
\begin{equation}\label{key}
[\hH, \hG] = 0
\end{equation}
thus
\begin{equation}\label{key}
\dv{t}\ev{\hG} = \I\ev{[\hH, \hG]} = 0 \;(\text{conserved})
\end{equation}
Noether's Theorem: Symmetry $ \dra $ Conserved quantity\\
E.g.: translational symm. -- momentum conserved\\

\subsubsection{Finite Transformation}
\begin{equation}\label{key}
\hU(\bm\alpha) = \e^{\I\bm\alpha\cdot\hG}
\end{equation}

\subsection{Flavor Symmetry $ \bigstar $}
Heisenberg: proton and neutron are 2 states of nucleon.\\
In isospin space
\begin{equation}\label{key}
p = \mqty(1\\0) \quad n = \mqty(0\\1)
\end{equation}
total isospin $ I = 1/2 $\\
3rd component of isospin $ I_3 = \pm 1/2 $\\

\subsubsection{Flavor Symm. of the Strong Interaction}
flavor space
\begin{equation}\label{key}
\u = \mqty(1\\0) \quad \d = \mqty(0\\1)
\end{equation}
\begin{equation}\label{key}
\mqty(\u'\\ \d') = \hU\mqty(\u\\ \d)
\end{equation}
\begin{equation}\label{key}
\hU = \mqty(U_1 & U_2\\U_3 & U_4)
\end{equation}
4 complex numbers -- 8 real parameters\\
$ \hU^\dagger \hU = I $ -- 4 constraints\\
thus, 4 real parameters needed.
\begin{equation}\label{key}
\hU = \e^{\I\alpha_i\hG_i} \quad i = 1,2,3,4
\end{equation}
one of these
\begin{equation}\label{key}
\hU = \mqty(1 & 0\\0 & 1)\e^{\I\phi}  \quad U(1) \text{ symm}
\end{equation}
the remaining three -- $ SU(2) $ (special unitary, traceless)\\
a suitable choice: Pauli matrices
\begin{equation}\label{key}
\sigma_1 = ...
\end{equation}
def: isospin
\begin{equation}\label{key}
\hT = \dfrac{1}{2}\bm\sigma
\end{equation}
\subsubsection{Isospin Algebra}
spin: $ \ket{l,m} $\\
isospin: $ \ket{I, I_3} $\\
\begin{equation}\label{key}
\u = \mqty(1\\0) = \Ket{\dfrac{1}{2},\dfrac{1}{2}} \quad \d = \mqty(0\\1) = \Ket{\dfrac{1}{2}, -\dfrac{1}{2}}
\end{equation}

\subsection{Combining Quarks into Baryons}
$ 2\otimes 2 = 3\oplus 1 $\\
singlet
\begin{equation}\label{key}
\ket{0,0} = \dfrac{1}{\sqrt{2}}(\u\d - \d\u)
\end{equation}
triplet
\begin{equation}\label{key}
\left\{
\mqty{\ket{1,1} = \u\u\\
	  \ket{1,0} = \tfrac{1}{\sqrt{2}}(\u\d + \d\u)\\
	  \ket{1,-1} = \d\d
	 }
\right.
\end{equation}

$ 2\otimes 2\otimes 2  = (3\oplus 1)\otimes 2 = 4\oplus 2\oplus 2$\\
(using spin multiplicity)\\
actually $ 4\oplus 2_S\oplus 2_A$ , symm and anti-symm\\
$ 3\otimes 2 $\\
$ 1 - \dfrac{1}{2} \leq I \leq 1 + \dfrac{1}{2}  \dra I = \dfrac{1}{2}, \dfrac{3}{2}$\\
\begin{equation}\label{key}
2I+1 \dra 4\oplus 2
\end{equation}
\begin{equation}\label{key}
\left\{
\mqty{\ket{3/2, 3/2} = \\
	  \ket{3/2, 1/2} = \\
	  \ket{3/2, -1/2} = \\
	  \ket{3/2, -3/2} = 
     }
\right.
\end{equation}
\begin{equation}\label{key}
\left\{
\mqty{\ket{1/2, 1/2}_S = \\
	\ket{1/2, -1/2}_S = \\
     }
\right.
\end{equation}
$ 1\otimes 2 $
\begin{equation}\label{key}
\left\{
\mqty{\ket{1/2, 1/2}_A = \\
	\ket{1/2, -1/2}_A = \\
}
\right.
\end{equation}

\subsection{Ground State Baryon Wfn.}
For proton, $ p = \u\u\d $, or neutron\\
\begin{equation}\label{key}
\Psi = \eta_{\text{space}} \chi_{\text{spin}} \phi_{\text{flavor}} \xi_{\text{color}}
\end{equation}
spin -- 8, flavor -- 8, color -- 27\\
But QCD states:\\
$ \xi $ -- anti-symm\\
$ \eta $ -- $ (-1)^l $, for ground state, symm\\
And $ \Psi $ is anti-symm\\
So, $ \chi\phi $ is symm
\begin{equation}\label{key}
\Psi = \dfrac{1}{\sqrt{2}} (\chi_S\phi_S + \phi_A\chi_A)
\end{equation}

\subsection{Isospin Representation of Anti-Quarks}
A general SU(2) transformation $ q \ra q' = Uq $
\begin{equation}\label{key}
\mqty(\u \\ \d) \ra \mqty(\u\\ \d) = \mqty(a & b\\ -b^* & a^*)\mqty(\u\\ \d)
\end{equation}
with $ aa^* + bb^* = 1 $\\
take complex conjugation
\begin{equation}\label{key}
\mqty(\bar\u\\ \bar\d) = \mqty(a^* & b^*\\ -b & a)\mqty(\bar\u\\ \bar\d)
\end{equation}



\subsection{SU(3) Flavor Symmetry}
\begin{equation}\label{key}
\mqty(\u' \\ \d' \\ \s') = \hU\mqty(\u\\ \d\\ \s) = 
\mqty(U_{11} & U_{12} & U_{13}\\
      ... &&\\
      ... &&)
\mqty(\u\\ \d\\ \s)
\end{equation}
\begin{equation}\label{key}
\hU = \e^{\I \bm\alpha\cdot\hT}
\end{equation}
\begin{equation}\label{key}
\hT_i = \dfrac{1}{2}\bm\lambda_i \quad i = 1,2,...,8
\end{equation}
\begin{equation}\label{key}
\u = \mqty(1\\ 0\\ 0) \quad \d = \mqty(0\\ 1\\ 0) \quad \s = \mqty(0 \\ 0\\ 1)
\end{equation}
expand SU(2) of $ \u,\d $ to SU(3)
\begin{equation}\label{key}
\lambda_1 = \mqty(1 & 0 & 0\\ 1 & 0 & 0\\ 0 & 0 & 0) \quad
\lambda_2 = ... \quad 
\lambda_3 = ...
\end{equation}
expand SU(2) of $ \u,\s $ and $ \d,\s $ to SU(3)
\begin{equation}\label{key}
\lambda_4 = \mqty(0 & 0 & 1\\ 0 & 0 & 0\\ 1 & 0 & 0) \quad
\lambda_5 = ... \quad 
\lambda_X = \mqty(1 & 0 & 0\\ 0 & 0 & 0\\ 0 & 0 & -1)
\end{equation}
\begin{equation}\label{key}
\lambda_6 = \mqty(0 & 0 & 0\\ 0 & 0 & 1\\ 0 & 1 & 0) \quad
\lambda_7 = ... \quad 
\lambda_Y = \mqty(0 & 0 & 0\\ 0 & 1 & 0\\ 0 & 0 & -1)
\end{equation}
However $ \lambda_3,\lambda_X,\lambda_Y $ are not linear independent, thus we take
\begin{equation}\label{key}
\lambda_8 = \dfrac{1}{\sqrt{3}}\lambda_X + \dfrac{1}{\sqrt{3}}\lambda_Y = \dfrac{1}{\sqrt{3}}\mqty(1 & 0 & 0\\ 0 & 1 & 0\\ 0 & 0 & -2)
\end{equation}
those eight aka Gell-Mann matrices

\subsubsection{SU(3) Flavor States}
\begin{equation}\label{key}
\hT^2 = \dfrac{1}{4} \sum_{i=1}^8 \lambda_i^2 = 
\end{equation}
observables: 3rd component of isospin and flavor hypercharge
\begin{equation}\label{key}
\hT_3 = \dfrac{1}{2}\lambda_3 \quad \hat{Y} = \dfrac{1}{\sqrt{3}}\lambda_8
\end{equation}
\begin{equation}\label{key}
I_3 = n_u - n_d \quad Y = \dfrac{1}{3}(n_u + n_d - 2n_s)
\end{equation}
states
\begin{equation}\label{key}
\u = (1/2, 1/3) \quad \bar\u = (-1/2, -1/3)
\end{equation}
\begin{equation}\label{key}
\d = (-1/2, 1/3) \quad \bar\d = (1/2, -1/3)
\end{equation}
\begin{equation}\label{key}
\s = (0, -2/3) \quad \bar\s = (0, 2/3)
\end{equation}
Meson ($ q\bar{q} $):
\begin{equation}\label{key}
\u\bar\u = (0, 0) \quad \u\bar\d = (1, 0) \u\bar\s = (1/2, 1)
\end{equation}
\begin{equation}\label{key}
\d\bar\u = ...
\end{equation}
\begin{equation}\label{key}
\s\bar\u = ...
\end{equation}

\subsubsection{The Light Mesons}

Flavor symm is approximate.



\section{Quantum Chromodynamics}
$ \bigstar $
What's the diff between SU(3)color and SU(3)flavor?\\
color is precise and gauge.
\subsection{The Local Gauge Principle}
\begin{equation}\label{key}
\psi(x) \ra \psi'(x) = \hU(x)\psi(x) = \e^{\I q\chi(x)}\psi(x)
\end{equation}
$ \chi(x) $ is dependent on $ x $. It's local U(1) transformation.
\begin{equation}\label{key}
\I\gamma^\mu\partial_\mu [\e^{\I q\chi(x)}\psi] = m[\e^{\I q\chi(x)}\psi]
\end{equation}
\begin{equation}\label{key}
\I\gamma^\mu\e^{\I q\chi(x)}[\partial_\mu \psi + \I q(\partial_\mu \chi) \psi] = \e^{\I q\chi(x)} m\psi
\end{equation}
\begin{equation}\label{key}
\I\gamma^\mu [\partial_\mu + \I q(\partial_\mu \chi)] \psi = m\psi
\end{equation}
def: $ D_\mu = \partial_\mu + \I q A_\mu $, gauge invariant derivative operator
\begin{equation}\label{key}
\I\gamma^\mu [\partial_\mu + \I q A_\mu] \psi = m\psi
\end{equation}
\begin{equation}\label{key}
A_\mu \ra A'_\mu = A_\mu - \partial_\mu \chi
\end{equation}

\subsubsection{From QED to QCD}
\begin{table}[H]
	\centering
	\begin{tabular}{|c|c|c|}
		\hline
		               & QED      &  QCD \\ \hline
		symm           & U(1)     &  SU(3) \\ \hline
		gauge boson & photon $ A_\mu $   & gluon $ G_\mu^a \quad a = 1,2,...,8$
		\\ \hline
		generator   & q           & $ T^a = \lambda^a / 2 $
		\\ \hline
		charge      & electric chg. $ +/- $ & color chg. $ r/g/b $ \\ \hline
	\end{tabular}
\end{table}
elec/color neutral particles do not interact with photon/gluon. But photon has no elec chg, while gluon has color chg.\\
SU(3) local transf.
\begin{equation}\label{key}
\psi(x) \ra \psi'(x) = \e^{\I g_S \bm\alpha(x)\cdot\hT} \psi(x)
\end{equation}
Dirac Eq. becomes
\begin{equation}\label{key}
\I\gamma^\mu [\partial_\mu + \I g_S(\partial_\mu \bm\alpha)\cdot\hT] \psi = m\psi
\end{equation}
\begin{equation}\label{key}
\I\gamma^\mu [\partial_\mu + \I g_S G_\mu^a T^a] \psi = m\psi
\end{equation}
\begin{equation}\label{key}
G_\mu^k \ra (G_\mu^k)' = G_\mu^k - \partial_\mu \alpha_k - g_S f_{ijk} \alpha_i G_\mu^j 
\end{equation}
$ f_{ijk} $: structure constant, def by
\begin{equation}\label{key}
[\lambda_i, \lambda_j] = 2\I f_{ijk}\lambda_k
\end{equation}

-----\\
U(1) -- Abelian\\
SU(3) -- Non-Abelian, or Yang-Mils

\subsection{Color and QCD}
\subsubsection{Quark-gluon Vertex}
\begin{equation}\label{key}
(-\I g_S\gamma^\mu) \qty(\dfrac{1}{2}\lambda^a) \quad\text{\# (spinor index)(color index)}
\end{equation}
\begin{equation}\label{key}
u(p) \ra c_i u(p) 
\end{equation}
where $ c_1, c_2, c_3 = r,g,b $\\
\begin{figure}[H]
	\centering
	\feynmandiagram [vertical = b to d]{
		a [particle=q] -- [fermion,edge label=$ p_1 $] b [label=$ {\mu,a} $] -- [fermion,edge label=$ p_3 $] c [particle=q],
		b -- [gluon] d [\pa=g],
	};
\end{figure}
\begin{equation}\label{key}
j_q^\mu = \bar{u}(p_3) c_j^\dagger \qty(-\I g_S\gamma^\mu \dfrac{1}{2}\lambda^a) c_i u(p_1) 
\end{equation}

gluon propagator
\begin{figure}[H]
	\centering
	\feynmandiagram [horizontal = a to b]{
		a [\pa=$ {\mu,a} $] -- [gluon,edge label=$ g $] 
		b [\pa=$ {\nu,b} $],
	};
\end{figure}
\begin{equation}\label{key}
-\I\dfrac{g_{\mu\nu}}{q^2}\delta^{ab}
\end{equation}

\subsection{Gluons}
gluon -- octet of colored states
\begin{equation}\label{key}
r\bar{g}, g\bar{r}, r\bar{b}, b\bar{r}, g\bar{b}, b\bar{g}, \dfrac{1}{\sqrt{2}}(r\bar{r} - g\bar{g}), \dfrac{1}{\sqrt{6}}(r\bar{r} + g\bar{g} - 2b\bar{b})
\end{equation}

\subsection{Color Confinement}
gluon field
\begin{equation}\label{key}
V(\vb{r}) \sim \kappa r
\end{equation}
colored objects are always confined to color singlet states.

\subsubsection{Color Confinement and Hadronic States}
$ 3\otimes \bar{3} = 8 \oplus 1 $, the 8 are gluon states and the 1 is meson state ($ \q\bq $).\\
$ 3\otimes 3 = 6 \oplus 3 $, and hadrons must be color singlet, thus qq meson does not exist.\\

For baryon ($ \q\q\q $)
\begin{equation}\label{key}
3 \otimes 3 \otimes 3 = 10 \oplus 8 \oplus 8 \oplus 1
\end{equation}
while $ \q\q\bq, \q\bq\bq $ do not exist.\\
Other: antibaryon $ (\bq\bq\bq)$, pentaquark $ (\q\q\q\q\bq) $.

\subsubsection{Hadronization and Jets}
...

\subsection{Running of $ \alpha_S $ and Asymptotic Freedom(渐进自由)}
... photo

\subsection{QCD in e-p Annihilation}
\begin{figure}[H]
	\centering
	\feynmandiagram [horizontal = b to e]{
		a [\pa=$ e^- $] -- [fermion] b -- [fermion] c [\pa=$ e^+ $],
		b -- [photon] e,
		d [\pa=$ \bq $] -- [fermion] e -- [fermion] f [\pa=$ \q $],
	};
\end{figure}
\begin{equation}\label{key}
R = \dfrac{\sigma(\ce{e^+ e^- \ra \text{hadrons}})}{\sigma(\ce{e^+ e^- \ra \mu^+ \mu^-})} = 3\sum_{\text{flavors}} Q_q^2
\end{equation}

agree w/ data by error 10\%(圈图修正?)\\

\subsection{}
\subsection{}
\subsection{Hadron-hadron Collisions}
elec collision -- $ \SI{10}{GeV} $\\
hadron coll. -- $ \SI{13}{TeV} $\\

\subsubsection{Kinematics}
Variables\\
$ e^- p $ elastic: $ \theta_e $\\
$ e^- p $ inelastic: $ \theta_e $, $ E_e $\\
$ pp $: $ x_1, x_2, Q^2 $\\
~\\
$ p\bar{p} \ra \text{2 jets} + X $\\
$ \{x_1, x_2, Q^2\} \ra \{\theta_1, \theta_2, P_T\} $\\
\begin{figure}[H]
	\centering
	\feynmandiagram [horizontal = a to b]{
		a [\pa=p] -- [fermion] b -- [fermion] c [\pa=$ j_1 $],
		d [\pa=p] -- [fermion] b -- [fermion] e [\pa=$ j_2 $],
	};
\end{figure}
\begin{equation}\label{key}
P_T = \sqrt{p_x^2 + p_y^2}
\end{equation}
net longitudinal momentum
\begin{equation}\label{key}
(x_1 - x_2)E_p
\end{equation}
where $ E_p $ is the energy og the proton.\\
def rapidity
\begin{equation}\label{key}
y = \dfrac{1}{2}\ln\dfrac{E + p_z}{E - p_z}
\end{equation}
high-energy jets: $ p_z =\approx E\cos\theta $
\begin{equation}\label{key}
y \approx \dfrac{1}{2}\ln(...) = -\ln\tan\dfrac{\theta}{2} \equiv \eta
\end{equation}

\subsubsection{The Drell-Yan process}
\begin{figure}[H]
	\centering
	\feynmandiagram [horizontal = c to d]{
		a [\pa=p] -- [fermion] b,
		b -- [fermion, \el=$x_1 p_1$] c --  [fermion, \el=$ x_2 p_2 $] b2,
		b2 -- [fermion] a2 [\pa=$ \bar{p} $],
		c -- [photon] d,
		e [\pa=$ \mu^+ $] -- [fermion,\el=$ p_4 $] d -- [fermion,\el=$ p_3 $] f[\pa=$ \mu^- $],
	};
\end{figure}

\begin{equation}\label{key}
\dd[2]\sigma = ...
\end{equation}

\begin{equation}\label{key}
\dd{\sigma}{y}{M} = \dfrac{8\pi\alpha^2}{9Ms} f\qty(\dfrac{M}{\sqrt{s}}\e^y,\dfrac{M}{\sqrt{s}}\e^{-y})
\end{equation}

\subsubsection{Jet Production at the LHC}
parton level diagram
\begin{figure}[H]
	\centering
	\feynmandiagram [vertical = b to e]{
		a [\pa=q] -- [fermion] b -- [fermion] c [\pa=q],
		b -- [gluon] e,
		d [\pa=q] -- [fermion] e -- [fermion] f [\pa=q],
	};
\end{figure}
\begin{equation}\label{key}
\dv{\hat\sigma}{Q^2} = \dfrac{4\pi\alpha_S^2}{9Q^4}\qty[1 + \qty(1 - \dfrac{Q^2}{\hat{s}})^2]
\end{equation}
hat means parton level
\begin{equation}\label{key}
\dv{\sigma}{Q^2} = \dv{\hat\sigma}{Q^2} g(x_1,x_2)\dd x_1 \dd x_2
\end{equation}

homework Jun 5\\
10.6 10.7\\

\section{The Weak Interaction}
\subsection{}
\begin{table}[H]
	\begin{tabular}{|c|c|}
		\hline
		QED/QCD & weak\\ \hline
		massless gauge boson & massive gauge boson\\ \hline
		vector intrxn & vector \& axial vector\\ \hline
	\end{tabular}
\end{table}
\subsection{Parity}
parity transf.
\begin{equation}\label{key}
\hP\psi(\vb{x},t) = \psi(-\vb{x},t)
\end{equation}
must be unitary and Hermitian.\\
and $ P = \pm 1 $.

\subsubsection{Intrinsic Parity}
(4.9) shows thr parity op. for Dirac spinors is $ \gamma^0 $.\\
For spin-half particles, $ P = +1 $; for anti-particles, $ P = -1 $.\\
For vector bosons responsible for the EM, strong, weak forces
\begin{equation}\label{key}
P(\gamma) = P(\g) = P(\W^\pm) = P(\Z) = -1
\end{equation}

\subsubsection{Parity Conservation in QED}
\begin{figure}[H]
	\centering
	\feynmandiagram [vertical = b to e]{
		a [\pa=$ e^- $] -- [fermion] b -- [fermion] c [\pa=$ e^- $],
		b -- [photon] e,
		d [\pa=q] -- [fermion] e -- [fermion] f [\pa=q],
	};
\end{figure}
\begin{equation}\label{key}
-\I \mathcal{M} = \bar{u}(p_3)[\I e Q \gamma^\mu] u(p_1)\dfrac{-\I g_{\mu\nu}}{q^2} \bar{u}(p_4)[\I e Q_q \gamma_\nu] u(p_2)
\end{equation}
\begin{equation}\label{key}
\mathcal{M} = \dfrac{Q_q e^2}{q^2} j_e \cdot j_q
\end{equation}
\begin{equation}\label{key}
j_e^\mu = ...
\end{equation}
Parity transf.
\begin{equation}\label{key}
\hP u = \gamma^0 u
\end{equation}
\begin{equation}\label{key}
\hP \bar{u} = \hP u^\dagger \gamma^0 = ... = \bar{u}\gamma^0  \text{???}
\end{equation}
thus
\begin{equation}\label{key}
\hP j_e^0 = \bar{u}\gamma^0\gamma^0\gamma^0 u = j_e^0
\end{equation}
\begin{equation}\label{key}
\hP j_e^k = ... = -j_e^k
\end{equation}
\begin{equation}\label{key}
\hP j_e\cdot j_q = \hP(j_e^0 j_q^0 - j_e^k j_q^k) = j_e \cdot j_q
\end{equation}
thus, parity is conserved in QED.\\
And also in QCD.\\
e.g.
\begin{equation}\label{key}
\ce{\rho^0(1^-) -> \pi^+(0^-) + \pi^-(0^-)}
\end{equation}
where $ 1^- $ denotes $ J^P $.\\
ang momentum conservation $ 1 = 0 + 0 + \ell \dra \ell = 1$\\
parity consv. $ (-1) = (-1)(-1)(-1)^\ell $  $ \checkmark $
\begin{equation}\label{key}
\ce{\eta^0(0^-) -> \pi^+(0^-) + \pi^-(0^-)}
\end{equation}
where $ 1^- $ denotes $ J^P $.\\
ang momentum conservation $ 0 = 0 + 0 + \ell \dra \ell = 0$\\
parity consv. $ (-1) = (-1)(-1)(-1)^\ell $  $ \cross $\\
~\\
\begin{table}[H]
	\begin{tabular}{|c|c|c|}
		\hline
		       & e.g. & $ P $ \\ \hline
		scalar &  &  +1 \\ \hline
		vector & $ \vb{x}, \vb{p} $ & -1 \\ \hline
		pseudoscalar & $  h = \vb{S}\cdot\vb{p} $ & -1 \\ \hline
		axial vector & $ \vb{L}, \vb{S}, \vb{B}, \bm\mu $ & +1 \\ \hline
	\end{tabular}
\end{table}

\subsubsection{Parity Violation in Nuclear $ \beta $-decay}

\subsection{V-A Structure of the Weak Interaction $ \bigstar $}
\begin{equation}\label{key}
j^\mu \propto \bar{u}(g_V\gamma^\mu + g_A\gamma^\mu \gamma^5)u = g_V j_V^\mu + g_A j_A^\mu
\end{equation}
...
parity violation $ \propto \dfrac{g_V g_A}{g_V^2 + g_A^2} $\\
Experiment shows maximal parity violation: $ \abs{g_V} = \abs{g_A} $\\
and it looks like $ V - A $(V minus A)
\begin{equation}\label{key}
j^\mu = g\bar\psi\gamma^\mu(1 - \gamma^5)\psi
\end{equation}
Feynman rule
\begin{equation}\label{key}
...
\end{equation}

\subsection{Chiral Structure of the Weak Interaction}
\begin{equation}\label{key}
P_R = ...
\end{equation}


\begin{equation}\label{key}
\bar\psi_{1L}\gamma^\mu \psi_{2R} = 0 = \bar\psi_{1R}\gamma^\mu \psi_{2L}
\end{equation}

Only L particles participate in charged weak current.\\
...  R anti-particles ...\\


$ E >> m \;\dra$ chiral states $ \approx $ helicity states

\subsection{The W-boson Propagator}
propagator of photon
\begin{equation}\label{key}
\dfrac{-\I g_{\mu\nu}}{q^2}
\end{equation}
of W-boson
\begin{equation}\label{key}
\dfrac{-\I}{q^2 - m_W^2}\qty(g_{\mu\nu} - \dfrac{q_\mu q_\nu}{m_W^2})
\end{equation}

\subsubsection{Fermi Theory}
for low energy
\begin{equation}\label{key}
\dfrac{\I g_{\mu\nu}}{m_W^2}
\end{equation}

\subsection{Helicity in Pion Decay}
$ \pi^\pm $, $ J^P = 0^- $, formed from $ \u\bar\d / \d\bar\u $.\\
decay:
\begin{eqnarray}
&\pi^- \ra e^- \bar{\nu}_e\\
&\pi^- \ra \mu^- \bar{\nu}_\mu\\
\end{eqnarray}

\section{Review}
helicity 对重粒子是不是LI的。\\
Jacobi\\

\end{document}