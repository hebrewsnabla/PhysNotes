%\documentclass[UTF8]{ctexart} % use larger type; default would be 10pt
\documentclass[a4paper]{article}
\usepackage{xeCJK}
%\usepackage{ctex}
%\usepackage[utf8]{inputenc} % set input encoding (not needed with XeLaTeX)

%%% Examples of Article customizations
% These packages are optional, depending whether you want the features they provide.
% See the LaTeX Companion or other references for full information.

%%% PAGE DIMENSIONS
\usepackage{geometry} % to change the page dimensions
\geometry{a4paper} % or letterpaper (US) or a5paper or....
% \geometry{margin=2in} % for example, change the margins to 2 inches all round
% \geometry{landscape} % set up the page for landscape
%   read geometry.pdf for detailed page layout information

\usepackage{graphicx} % support the \includegraphics command and options

% \usepackage[parfill]{parskip} % Activate to begin paragraphs with an empty line rather than an indent

%%% PACKAGES
\usepackage{booktabs} % for much better looking tables
\usepackage{array} % for better arrays (eg matrices) in maths
\usepackage{paralist} % very flexible & customisable lists (eg. enumerate/itemize, etc.)
\usepackage{verbatim} % adds environment for commenting out blocks of text & for better verbatim
\usepackage{subfig} % make it possible to include more than one captioned figure/table in a single float
% These packages are all incorporated in the memoir class to one degree or another...

%%% HEADERS & FOOTERS
\usepackage{fancyhdr} % This should be set AFTER setting up the page geometry
\pagestyle{fancy} % options: empty , plain , fancy
\renewcommand{\headrulewidth}{0pt} % customise the layout...
\lhead{}\chead{}\rhead{}
\lfoot{}\cfoot{\thepage}\rfoot{}

%%% SECTION TITLE APPEARANCE
\usepackage{sectsty}
\allsectionsfont{\sffamily\mdseries\upshape} % (See the fntguide.pdf for font help)
% (This matches ConTeXt defaults)

%%% ToC (table of contents) APPEARANCE
\usepackage[nottoc,notlof,notlot]{tocbibind} % Put the bibliography in the ToC
\usepackage[titles,subfigure]{tocloft} % Alter the style of the Table of Contents
\renewcommand{\cftsecfont}{\rmfamily\mdseries\upshape}
\renewcommand{\cftsecpagefont}{\rmfamily\mdseries\upshape} % No bold!

%%% END Article customizations

%%% The "real" document content comes below...

\setlength{\parindent}{0pt}
\usepackage{physics}
\usepackage{amsmath}
%\usepackage{symbols}
\usepackage{AMSFonts}
\usepackage{bm}
%\usepackage{eucal}
\usepackage{mathrsfs}
\usepackage{amssymb}
\usepackage{float}
\usepackage{multicol}
\usepackage{abstract}
\usepackage{empheq}
\usepackage{extarrows}
\usepackage{textcomp}
\usepackage{mhchem}
\usepackage{braket}
\usepackage{siunitx}
\usepackage[utf8]{inputenc}
\usepackage{tikz-feynman}
\usepackage{feynmp}


\DeclareMathOperator{\p}{\prime}
\DeclareMathOperator{\ti}{\times}

\DeclareMathOperator{\e}{\mathrm{e}}
\DeclareMathOperator{\I}{\mathrm{i}}
\DeclareMathOperator{\Arg}{\mathrm{Arg}}
\newcommand{\NA}{N_\mathrm{A}}
\newcommand{\kB}{k_\mathrm{B}}

\DeclareMathOperator{\ra}{\rightarrow}
\DeclareMathOperator{\llra}{\longleftrightarrow}
\DeclareMathOperator{\lra}{\longrightarrow}
\DeclareMathOperator{\dlra}{\Leftrightarrow}
\DeclareMathOperator{\dra}{\Rightarrow}

%%%%%%%%%%%% QUANTUM MECHANICS %%%%%%%%%%%%%%%%%%%%%%%%
\newcommand{\bkk}[1]{\Braket{#1|#1}}
\newcommand{\bk}[2]{\Braket{#1|#2}}
\newcommand{\bkev}[2]{\Braket{#2|#1|#2}}

\DeclareMathOperator{\na}{\bm{\nabla}}
\DeclareMathOperator{\nna}{\nabla^2}
\DeclareMathOperator{\drrr}{\dd[3]\vb{r}}

\DeclareMathOperator{\psis}{\psi^\ast}
\DeclareMathOperator{\Psis}{\Psi^\ast}
\DeclareMathOperator{\hi}{\hat{\vb{i}}}
\DeclareMathOperator{\hj}{\hat{\vb{j}}}
\DeclareMathOperator{\hk}{\hat{\vb{k}}}
\DeclareMathOperator{\hr}{\hat{\vb{r}}}
\DeclareMathOperator{\hT}{\hat{\vb{T}}}
\DeclareMathOperator{\hH}{\hat{H}}

\DeclareMathOperator{\hL}{\hat{\vb{L}}}
\DeclareMathOperator{\hp}{\hat{\vb{p}}}
\DeclareMathOperator{\hx}{\hat{\vb{x}}}
\DeclareMathOperator{\ha}{\hat{\vb{a}}}
\DeclareMathOperator{\hS}{\hat{\vb{S}}}
\DeclareMathOperator{\hSigma}{\hat{\bm\Sigma}}
\DeclareMathOperator{\hJ}{\hat{\vb{J}}}

\DeclareMathOperator{\Tdv}{-\dfrac{\hbar^2}{2m}\dv[2]{x}}
\DeclareMathOperator{\Tna}{-\dfrac{\hbar^2}{2m}\nabla^2}

%\DeclareMathOperator{\s}{\sum_{n=1}^{\infty}}
\DeclareMathOperator{\intinf}{\int_0^\infty}
\DeclareMathOperator{\intdinf}{\int_{-\infty}^\infty}
%\DeclareMathOperator{\suminf}{\sum_{n=0}^\infty}
\DeclareMathOperator{\sumnzinf}{\sum_{n=0}^\infty}
\DeclareMathOperator{\sumnoinf}{\sum_{n=1}^\infty}
\DeclareMathOperator{\sumndinf}{\sum_{n=-\infty}^\infty}
\DeclareMathOperator{\sumizinf}{\sum_{i=0}^\infty}

%%%%%%%%%%%%%%%%% PARTICLE PHYSICS %%%%%%%%%%%%%%%%
\DeclareMathOperator{\hh}{\hat{h}}               % helicity
\DeclareMathOperator{\hP}{\hat{\vb{P}}}          % Parity

\DeclareMathOperator{\GeV}{\si{GeV}}
\DeclareMathOperator{\LI}{\mathscr{L}.I.}
\DeclareMathOperator{\g5}{\gamma^5}
\DeclareMathOperator{\gmuu}{\gamma^\mu}
\DeclareMathOperator{\gmud}{\gamma_\mu}
\DeclareMathOperator{\gnuu}{\gamma^\nu}
\DeclareMathOperator{\gnud}{\gamma_\nu}

\newcommand{\dis}{\displaystyle}
\numberwithin{equation}{section}

\title{Notes of Particle Physics, LIU Zuowei}
\author{hebrewsnabla}
%\date{} % Activate to display a given date or no date (if empty),
         % otherwise the current date is printed 

\begin{document}
% \boldmath
\maketitle

\tableofcontents

\newpage

\setcounter{section}{0}
\section{Introduction}
物理楼318, zuoweiliu@nju.edu.cn\\
平时20 作业30 期末50\\

References\\
-- Mark Thomson "Modern Particle Physics"\\
-- David Griffiths "Introduction to Elementary Particles"\\
-- Francis Halzen \& Alan D. Martin "Quarks \& Leptons"\\

\subsection{Elementary Particles}~\\
p -- uud\\
n -- ddu\\
u -- upper quark, d -- down quark\\
c -- s --\\
t -- top b -- \\
~\\
Gen1 -- $ e^-/\nu_e/u/d $\\
Gen2 -- $ \mu^-/\nu_\mu/c/s $\\
Gen3 -- $ \tau^-/\nu_\tau/t/b $\\
all have spin 1/2\\
charge: u,c,t 2/3; d,s,b -1/3\\
mass: $ m_t \approx 170 m_p $\\
~\\
Electromagnetic Interaction, EM -- proton\\
Strong -- gluon\\
Weak -- $ W^\pm/Z^0 $\\
Gravity -- Higgs boson\\
$ m_g = 125GeV $, $ m_Z = 91GeV $, $ m_W = 80GeV $ -- EW scale\\
~\\

\subsection{Feynman Diagrams}
EM $ \alpha_{EM} = 1/137 $\\
strong $ \alpha_s = 1 $\\
weak $ \alpha_w = 1/30 $\\
~\\
small $ \alpha $ enables perturbative calculation\\
Feynman rules\\

\section{Underlying Concepts}
\subsection{Natural Units}
$ [kg, m, s] \ra [\si{GeV}, \hbar, c] $\\
$ \hbar = c = 1 $
mass: $ \si{GeV}/c^2 $ -- $ \si{GeV} $\\
length: $ \hbar c/\GeV $ - $ \GeV^{-1} $\\
time: $ \hbar/\si{GeV} $ -- $ \GeV^{-1} $\\

$ E, \vb{p}, m \lra \GeV $\\
$ t, l \lra \GeV^{-1} $\\
$ \sigma \lra \GeV^{-2} $\\

e.g. $ \ev{r^2}^{1/2} = 4.1\GeV^{-1} = ? $\\
$ [L] = [E]^{-1}[\hbar][c] $\\
$ \GeV^{-1} = \SI{1.6e-10}{J^{-1}}\cdot\SI{3e8}{m.s^{-1}}\cdot\SI{1.055e-34}{J.s} \approx \SI{0.8}{fm} $\\
\paragraph{Heaviside-Lorentz Units}
\begin{equation}\label{key}
\epsilon_0 = \mu_0 = 1
\end{equation}
\begin{equation}\label{key}
\alpha = \dfrac{e^2}{4\pi\epsilon_0\hbar c} \dra \dfrac{e^2}{4\pi} = \dfrac{1}{137}
\end{equation}

\subsection{Special Relativity}
\subsubsection{The Lorentz Transformation}
\begin{equation}\label{key}
\vb{X}' = \bm{\Lambda}\vb{X}
\end{equation}
\begin{equation}\label{key}
\mqty(t'\\x'\\y'\\z') 
= \mqty(\gamma & 0 & 0 & -\gamma\beta\\
        0 & 1 & 0 & 0\\
        0 & 0 & 1 & 0\\
        -\gamma\beta & 0 & 0 & \gamma)
  \mqty(t\\x\\y\\z)
\end{equation}
where $ \gamma = (1 - \beta^2)^{-1/2} $.
\begin{equation}\label{key}
\vb{X} = \bm{\Lambda}^{-1}\vb{X}'
\end{equation}
\begin{equation}\label{key}
\mqty(t\\x\\y\\z) 
= \mqty(\gamma & 0 & 0 & \gamma\beta\\
0 & 1 & 0 & 0\\
0 & 0 & 1 & 0\\
\gamma\beta & 0 & 0 & \gamma)
\mqty(t'\\x'\\y'\\z')
\end{equation}
\begin{equation}\label{key}
\bm{\Lambda}\bm{\Lambda}^{-1} = \vb{I}
\end{equation}

\subsubsection{4-Vectors \& Lorentz Invariant}
contravariant 4-vector
\begin{equation}\label{key}
X^\mu = (t,x,y,z)
\end{equation}
when $ \mu \ra \mu' $, 
\begin{equation}\label{key}
X^{\mu'} = \Lambda^\mu_\nu X^\nu
\end{equation}
\begin{equation}\label{key}
t^2 - x^2 - y^2 - z^2 = t'^2 - x'^2 - y'^2 - z'^2
\end{equation}
covariant 4-vector $ X_\mu = (t,-x,-y,-z) $\\
$ X^\mu X_\mu = X^{\mu'} X_{\mu'} $ \# Ein sum convention\\
thus $ X^\mu X_\mu = \mathscr{L}.I. $ \# Lorentz invariant\\
\begin{equation}\label{key}
X_\mu = G_\mu^\nu X^\nu
\end{equation}
\begin{equation}\label{key}
G_\mu^\nu \equiv \mqty( 1 & 0 & 0 & 0\\
                        0 & -1 & 0 & 0\\
                        0 & 0 & -1 & 0\\
                        0 & 0 & 0 & -1)
\end{equation}
~\\
\paragraph{4-momentum}
\begin{equation}\label{key}
P^\mu = (E, p_x, p_y, p_z)
\end{equation}
\begin{equation}\label{key}
P^\mu P_\mu = E^2 - \vb{p}^2 = m^2
\end{equation}

\paragraph{4-derivative}
\begin{equation}\label{key}
\mqty(\partial_{t'}\\\partial_{x'}\\\partial_{y'}\\\partial_{z'})
= \mqty(\gamma & 0 & 0 & \gamma\beta\\
0 & 1 & 0 & 0\\
0 & 0 & 1 & 0\\
\gamma\beta & 0 & 0 & \gamma)
\mqty(\partial_{t}\\\partial_{x}\\\partial_{y}\\\partial_{z})
\end{equation}
i.e.
\begin{equation}\label{key}
\partial_{\mu'} = \Lambda_\mu^\nu \partial_\nu
\end{equation}
\begin{equation}\label{key}
\partial^\mu\partial_\mu = \pdv[2]{t} - \nabla^2 \equiv \Box
\end{equation}

\subsubsection{Mandelstam Variables}

\subsection{Non-relativistic QM}
\subsubsection{Wave Mechanics \& Schr\"odinger Equation}
\subsubsection{Prob. Dens. \& Prob. Current}

\subsubsection{TD \& Conserved Quantities}

\subsubsection{Commutation relations \& Compatible Observables}

\subsubsection{Angular Momentum}

\section{Decay Rates and Cross Sections}

\subsection{Fermi's Golden Rule $ \bigstar $}
Transition rate
\begin{equation}\label{key}
\Gamma_{fi} = 2\pi \abs{T_{fi}}^2\rho(E_i)
\end{equation}

\subsection{Phase Space and Wavefxn Normalization $ \bigstar $}
$ \alpha \ra 1 + 2 $ \\

1st order:
\begin{equation}\label{key}
T_{fi} = \Braket{f | H' | i} = \Braket{\psi_1^*\psi_2^* | H' | \psi_a}
\end{equation}
Born approx.:
\begin{equation}\label{key}
\psi(\vb{x}, t) = A\e^{\I(\vb{p}\cdot\vb{x} - Et)}
\end{equation}
Normalization in a box of side $ a $:
\begin{equation}\label{key}
\Braket{\psi | \psi} = 1 \dra A^2 = 1/a^3
\end{equation}
PBC gives
\begin{equation}\label{key}
\vb{p} = \vb{n}\dfrac{2\pi}{a}
\end{equation}
thus
\begin{equation}\label{key}
\dd[3]{\vb{p}} = \dd[3]{\vb{n}}\dfrac{(2\pi)^3}{V}
\end{equation}
\begin{equation}\label{key}
\dd n \equiv \dd[3]\vb{n} \dd[3] p \dfrac{V}{(2\pi)^3} = 4\pi p^2\dd p \dfrac{V}{(2\pi)^3}
\end{equation}
\begin{equation}\label{key}
\rho(E) = \dv{n}{E} = \dv{n}{p}\abs{\dv{p}{E}}
\end{equation}
Set $ V = 1 $, and there are $ N $ particles, thus $ N - 1 $ indep momenta
\begin{equation}\label{key}
\dd n = \prod_{i=1}^{N-1}\dd n_i = \prod_{i=1}^{N-1} \dfrac{\dd[3]\vb{p}_i}{(2\pi)^3}
\end{equation}
\begin{equation}\label{key}
~
\end{equation}

\subsubsection{LI Phase Space}
\begin{equation}\label{key}
\Gamma_{fi} = \dfrac{1}{2E_a} \int \abs{\mathcal{M}_{fi}}^2\qty[(2\pi)^4\delta^4(\vb{p}_a - \vb{p}_1 - \vb{p}_2)] \dfrac{\dd[3]\vb{p}_1}{(2\pi)^3 2E_1} \dfrac{\dd[3]\vb{p}_2}{(2\pi)^3 2E_2}
\end{equation}

\subsubsection{Fermi G R Revisited}
\begin{equation}\label{key}
\Gamma_{fi} = 
\end{equation}
In CoM, $ E_a \ra m_a $\\
next subsec.

\subsection{Particle Decays $ \bigstar $}


静止参考系中
\begin{equation}\label{key}
p_a = (m_a, 0)
\end{equation}
$ \therefore $
\begin{equation}\label{key}
\delta^3(\vb{p}_a - \vb{p}_1 - \vb{p}_2)\dd[3]\vb{p}_2 = 
\end{equation}
\begin{equation}\label{key}
\Gamma = \dfrac{1}{2m_a} \int \abs{\mathcal{M}}^2 (2\pi)^4 \delta(m_a - E_1 - E_2)\dfrac{\dd[3]\vb{p}_1}{(2\pi)^6 4E_1E_2} 
\end{equation}
where $ \vb{p}_2 = -\vb{p}_1 $. thus $ E_2 = \sqrt{m_2^2 + p_1^2} $.\\
and $ \dd[3]\vb{p}_1 = p_1^2\dd p_1\sin\theta\dd\theta\dd\phi $
\begin{equation}\label{key}
\Gamma = \dfrac{1}{8\pi^2 m_a} \int \abs{\mathcal{M}}^2 \delta(m_a - \sqrt{m_1^2 + p_1^2} - \sqrt{m_2^2 + p_1^2})\dfrac{p_1^2\dd p_1\dd\Omega}{4E_1E_2} 
\end{equation}
Since
\begin{equation}\label{key}
\intdinf \delta[f(x)]g(x)\dd x = \sum_i\dfrac{g(x_i)}{\abs{f'(x_i)}}  \quad \text{w/} f(x_i)=0
\end{equation}
thus
\begin{equation}\label{key}
\Gamma_{fi} = \dfrac{p^*}{32\pi^2 m_a^2} \int \abs{\mathcal{M}_{fi}}^2 \dd\Omega
\end{equation}
where $ m_a - \sqrt{m_1^2 + {p^*}^2} - \sqrt{m_2^2 + {p^*}^2} = 0 $

\subsection{Interaction Cross Sections $ \bigstar $}
tot \# of target particles
\begin{equation}\label{key}
\delta N = n_b v\delta t A
\end{equation}
interaction prob.
\begin{equation}\label{key}
\delta P = \dfrac{\delta N \sigma}{A} = n_b v \sigma \delta t
\end{equation}

interaction rate per target particle
\begin{equation}\label{key}
\Gamma_b =  \dv{P}{t} = n v \sigma = \sigma \phi_a
\end{equation}
\begin{equation}\label{key}
\phi_a = \dfrac{\dd N_a}{\dd A \dd t} = \dfrac{n\dd A v\dd t}{\dd A\dd t} = n v
\end{equation}
total rate
\begin{equation}\label{key}
\Gamma_a n_a V = (n_a v) (n_b V) \sigma = \text{flux}\cross\text{\# target}\cross\sigma
\end{equation}

\subsubsection{$ \LI $ Flux}
$ \ce{a + b -> 1 + 2} $
\begin{equation}\label{key}
\Gamma_{fi} = \dfrac{v_a + v_b}{V}\sigma
\end{equation}
\begin{equation}\label{key}
\sigma = \dfrac{1}{4E_aE_b(v_a + v_b)} \int \abs{\mathcal{M}}^2 (2\pi)^4 \delta^4(p_a + p_b- p_1 - p_2) \dfrac{\dd[3]\vb{p}_1}{(2\pi)^3 2E_1} \dfrac{\dd[3]\vb{p}_2}{(2\pi)^3 2E_2}
\end{equation}
which is $ \LI $.\\
$ \LI $ flux factor $ F = 4E_aE_b(v_a + v_b) $.

\subsubsection{Scattering in the CoM frame}
photo

\begin{equation}\label{key}
\sigma = \dfrac{\abs{\vb{p}_f}}{64\pi^2 s\abs{\vb{p}_i}} \int \abs{\mathcal{M}}^2 \dd\Omega
\end{equation}

\subsection{Diff. Xsec. $ \bigstar $}
\begin{equation}\label{key}
\sigma = \int\dv{\sigma}{\Omega}\dd\Omega
\end{equation}
\subsubsection{CoM Frame}
\begin{equation}\label{key}
\dv{\sigma}{\Omega^*} = \dfrac{1}{64\pi^2}\dfrac{p_f^*}{p_i^*} \abs{\mathcal{M}_{fi}}^2
\end{equation}

\begin{equation}\label{key}
t = (p_i - p_3)^2   \quad \LI
\end{equation}
\begin{equation}\label{key}
\dv{\sigma}{t} = \dfrac{1}{64\pi s} \dfrac{\abs{\mathcal{M}_{fi}}^2}{{p_i^*}^2}  \quad \LI
\end{equation}

\subsubsection{Lab Frame}
$ \ce{e^- + p -> e^- + p} $
\begin{equation}\label{key}
\begin{aligned}
p_1 &\approx (E_1, 0, 0, E_1)\\
p_2 &= (m_p, 0, 0, 0)\\
... 
\end{aligned}
\end{equation}
\begin{equation}\label{key}
p_i^* \approx \dfrac{(s - m_p^2)^2}{4s}
\end{equation}
where $ s = (p_1 + p_2)^2 = m_p^2 + 2E_1m_p $\\

...


\section{The Dirac Equation}
\subsection{The Klein-Gordon Eq.}
\begin{equation}\label{key}
\hH^2 = \hp^2 + m^2
\end{equation}
where $ \hH = \dis\I\pdv{t} $, thus
\begin{equation}\label{key}
\pdv[2]{\psi}{t} = (\nabla^2 - m^2)\psi
\end{equation}
in $ \LI $ form
\begin{equation}\label{key}
(\partial^\mu\partial_\mu + m^2)\psi = 0
\end{equation}

\subsection{The Dirac Eq. $ \bigstar $}
\begin{equation}\label{key}
E\psi = (\bm\alpha\cdot\hp + \beta m)\psi
\end{equation}
\begin{equation}\label{key}
\begin{aligned}
-\pdv[2]{\psi}{t} &= (-\I\bm\alpha\cdot\na + \beta m)^2\psi\\
&= ...
\end{aligned}
\end{equation}
must satisfy
\begin{equation}\label{key}
\begin{aligned}
&\alpha_x^2 = \alpha_y^2 = \alpha_z^2 = \beta^2 = I\\
&\alpha_i\beta + \beta\alpha_i = 0\\
&\alpha_i\alpha_j + \alpha_j\alpha_i = 0
\end{aligned}
\end{equation}
------------
\begin{equation}\label{key}
\Tr(\alpha_i) = \Tr(\beta) = 0
\end{equation}
\begin{equation}\label{key}
\lambda_{\alpha_i} = \pm 1
\end{equation}
\begin{equation}\label{key}
len(\alpha_i) = len(\beta) = \mathrm{even number}
\end{equation}
\begin{equation}\label{key}
\alpha_i^\dagger = \alpha_i \quad \beta^\dagger = \beta
\end{equation}
-------------\\
Dirac spinor
\begin{equation}\label{key}
\psi = \mqty(\psi_1\\ \psi_2\\ \psi_3\\ \psi_4)
\end{equation}
There are infinite choices of $ \alpha $ and $ \beta $.\\
Dirac-Pauli representation:
\begin{equation}\label{key}
\beta = \mqty(I &0\\ 0 &-I) \quad \alpha_i = \mqty(0 &\sigma_i\\ \sigma_i &0)
\end{equation}
\begin{equation}\label{key}
\sigma_x = \mqty(0 & 1\\1 & 0) \quad \sigma_y = \mqty(0 & -\I\\ \I & 0) \quad \sigma_z = \mqty(1 & 0\\ 0 & -1)
\end{equation}

\subsection{Prob. Dens. \& Prob. Curr.}
\begin{equation}\label{key}
\I\pdv{\psi}{t} = -\I\bm\alpha\cdot\na \psi + m\beta\psi
\end{equation}
\begin{equation}\label{key}
-\I\pdv{\psi^\dagger}{t} = \I\na\psi^\dagger\cdot\bm\alpha^\dagger + m\psi^\dagger\beta^\dagger
\end{equation}
\begin{equation}\label{key}
\psi^\dagger\qty(-\I\bm\alpha\cdot\na \psi + m\beta\psi) - \qty(\I\na\psi^\dagger\cdot\bm\alpha^\dagger + m\psi^\dagger\beta^\dagger)\psi = \I\psi^\dagger\pdv{\psi}{t} + \I\pdv{\psi^\dagger}{t}\psi
\end{equation}
\begin{equation}\label{key}
\na\cdot(\psi^\dagger \bm\alpha \psi) + \pdv{t}(\psi^\dagger \psi) = 0
\end{equation}
\begin{equation}\label{key}
\na\cdot\vb{j} + \pdv{\rho}{t} = 0
\end{equation}

\subsection{Spin $ \bigstar $}
\begin{equation}\label{key}
[\hH_D, \hL] = [\bm\alpha\cdot\hp, \hr\cross\hp] = -\I \bm\alpha\cross\hp \neq = 0 
\end{equation}
\begin{equation}\label{key}
\hS = \dfrac{1}{2}\hSigma = \dfrac{1}{2} \mqty(\bm\sigma & 0\\ 0 & \bm\sigma)
\end{equation}
\begin{equation}\label{key}
[\alpha_i, \Sigma_j] = \mqty(0 & [\sigma_i, \sigma_j]\\ [\sigma_i, \sigma_j] & 0)
\end{equation}
\begin{equation}\label{key}
[\hH_D, \hSigma] = 2\I(\bm\alpha\cross\hp) \dra [\hH_D, \hS] = \I(\bm\alpha\cross\hp)
\end{equation}
def
\begin{equation}\label{key}
\hJ = \hL + \hS
\end{equation}
\begin{equation}\label{key}
[\hH_D, \hJ] = 0
\end{equation}
$ \hJ $ is a conserved quantity.\\
\begin{equation}\label{key}
\hS^2 = 
\end{equation}
\paragraph{EM -- magnetic moment $ \bigstar $}
\begin{equation}\label{mag}
E = -\bm\mu\cdot \vb{B}
\end{equation}
\begin{equation}\label{key}
E\psi = (\bm\alpha\cdot\hp + \beta m)\psi
\end{equation}
minimal substitution
\begin{equation}\label{key}
\mqty{E \ra E - q\phi\\
      \hp \ra \hp - q \vb{A}}
\end{equation}
where $ A = (\phi, \vb{A}) $
\begin{equation}\label{key}
(E - q\phi)\psi = [\bm\alpha\cdot(\hp - q \vb{A}) + \beta m]\psi
\end{equation}
\begin{equation}\label{key}
\mqty{(E - q\phi - m)\psi_A = \bm\sigma\cdot(\hp - q\vb{A})\psi_B\\
	  (E - q\phi + m)\psi_B = \bm\sigma\cdot(\hp - q\vb{A})\psi_A}
\end{equation}
Non-Rela limit: $ E \approx m >> q\phi $
\begin{equation}\label{key}
2m\psi_B = \bm\sigma\cdot(\hp - q\vb{A})\psi_A \quad \psi_B << \psi_A
\end{equation}
\begin{equation}\label{key}
\begin{aligned}
(E - q\phi - m)\psi_A &= \dfrac{1}{2m} [\bm\sigma\cdot(\hp - q\vb{A})]^2\psi_A\\
&= \dfrac{1}{2m} [(\hp - q\vb{A})^2 + \I\sigma\cdot(\hp - q\vb{A})\cross(\hp - q\vb{A})]\psi_A\\
&= \dfrac{1}{2m} [(\hp - q\vb{A})^2 + q\sigma\cdot(\na\cross\vb{A} + \vb{A}\cross\na)]\psi_A\\
\end{aligned}
\end{equation}
\begin{equation}\label{key}
(\na\cross\vb{A} + \vb{A}\cross\na)\psi_A = \vb{B}\psi_A
\end{equation}
\begin{equation}\label{key}
E\psi_A = \qty[m + \dfrac{1}{2m}(\hp - q\vb{A})^2 + q\phi - \dfrac{q}{2m}\bm\sigma\cdot\vb{B}]\psi_A
\end{equation}
From $ \eqref{mag} $,
\begin{equation}\label{key}
\bm\mu = \dfrac{q}{2m}\bm\sigma = \dfrac{q}{m}\hS
\end{equation}
which is spin magnetic moment.\\
Or,
\begin{equation}\label{key}
\bm\mu = g\dfrac{q}{2m}\hS
\end{equation}
Dirac Eq. explained $ g=2 $.\\
orbital magnetic moment
\begin{equation}\label{key}
\bm\mu = \dfrac{q}{2m}\hL
\end{equation}

In fact, Schwinger gave a correction on $ g $ in 1948
\begin{equation}\label{key}
a = \dfrac{g - 2}{2} = \dfrac{\alpha}{2\pi}
\end{equation}


\subsection{Covariant Form of the Dirac Eq.}
\begin{equation}\label{key}
\beta(\I\pdv{t} + \bm\alpha\cdot\hp - \beta m)\psi = 0
\end{equation}
def
\begin{equation}\label{key}
\gamma^0 = \beta \quad \gamma^1 = \beta\alpha_x \quad \gamma^2 = \beta\alpha_y \quad \gamma^3 = \beta\alpha_z
\end{equation}
and since $ \beta^2 = I $
\begin{equation}\label{cov}
(\I\gamma^\mu \partial_\mu - m)\psi = 0
\end{equation}
where $ \partial_\mu = \qty(\pdv{t}, \pdv{x}, \pdv{y}, \pdv{z}) $\\
Note: $ \gamma $ is not usual 4-vector, but $ \eqref{cov} $ is Lorentz covariant.

\subsubsection{}

\subsection{Sol. to the Dirac Eq. $ \bigstar $}
Suppose
\begin{equation}\label{key}
\psi(\vb{r}, t) = u(E, \vb{p})\e^{\I(\vb{p}\cdot\vb{r} - Et)}
\end{equation}
where $ u $ is a 4-component Dirac spinor\\
with $ \eqref{cov} $
\begin{equation}\label{key}
(\beta E - \beta\bm\alpha\cdot\vb{p} - m) u(E, \vb{p})\e^{\I(\vb{p}\cdot\vb{r} - Et)} = 0
\end{equation}
\begin{equation}\label{key}
(\gamma^\mu P_\mu - m)u = 0
\end{equation}
\subsubsection{Particles at Rest}
Since $ \vb{p} = 0 $
\begin{equation}\label{key}
(E\beta - m)u = 0
\end{equation}
\begin{equation}\label{key}
E \mqty(1 & 0 & 0 & 0\\
        0 & 1 & 0 & 0\\
        0 & 0 & -1 & 0\\
        0 & 0 & 0 & -1) \mqty(\phi_1\\\phi_2\\\phi_3\\\phi_4)
 = m \mqty(\phi_1\\\phi_2\\\phi_3\\\phi_4)
\end{equation}
the eigenvectors are
\begin{equation}\label{key}
u_1 = N\mqty(1\\0\\0\\0) \quad u_2 = N\mqty(0\\1\\0\\0)
\end{equation}
with $ E = m $
\begin{equation}\label{key}
u_3 = N\mqty(0\\0\\1\\0) \quad u_4 = 
\end{equation}
with $ E = -m $\\

How to explain negative energy?\\
Feynman-Stueckelberg:
\begin{equation}\label{key}
\e^{-\I Et} \equiv \e^{\I E (-t)}
\end{equation}
E.g. holes in solid state physics. Holes "move" oppositely as the electrons move, like propagating "backward" in time.\\
or anti-particles, like positron.


\subsubsection{General Free-particle Sol.}
\begin{equation}\label{key}
(\gamma^\mu P_\mu - m)u = 0
\end{equation}
\begin{equation}\label{key}
\mqty[\mqty(I& 0\\0 & -I) E 
      - \mqty(0 & \bm\sigma\cdot\vb{p}\\
              -\bm\sigma\cdot\vb{p} & 0)
      - m\mqty(I & 0\\0 & I)] u = 0
\end{equation}
\begin{equation}\label{key}
\mqty((E - m)I & -\bm\sigma\cdot\vb{p}\\
      \bm\sigma\cdot\vb{p} & -(E + m)I) \mqty(u_A\\u_B) = 0
\end{equation}
\begin{equation}\label{uaub}
u_A = \dfrac{\bm\sigma\cdot\vb{p}}{E - m}u_B \\
u_B = \dfrac{\bm\sigma\cdot\vb{p}}{E + m}u_A
\end{equation}
\begin{equation}\label{key}
u_A = .. = u_A
\end{equation}
thus we can choose $ u_A = \mqty(1\\0) $ or $ \mqty(0\\1) $


\subsection{Antiparticles $ \bigstar $}
\subsubsection{}
\subsubsection{the Feynman-St\"uckelberg Interpretation}
\begin{equation}\label{key}
(\gamma^\mu P_\mu + m)v = 0
\end{equation}
\begin{equation}\label{key}
\psi = v(p)\e^{\I Et - \I\vb{p}\cdot\vb{x}}
\end{equation}
...
\begin{equation}\label{key}
v_1 = N\mqty(...\\...\\0\\1) \quad v_2 = N\mqty(...\\...\\1\\0)
\end{equation}
kick out $ v_3 $, $ v_4 $, $ u_3 $, $ u_4 $ whose energy are negative.

\subsubsection{Normalization}
\begin{equation}\label{key}
\rho = \psi^\dagger \psi = u_1^\dagger u_1 = \abs{N}^2 \dfrac{2E}{E + m}
\end{equation}
def
\begin{equation}\label{key}
\rho = 2E \quad N = \sqrt{E + m}
\end{equation}

\subsubsection{Operators for Antiparticles}
\begin{equation}\label{key}
\hH^{(v)} = -\I\pdv{t} \quad \hp^{(v)} = +\I\na
\end{equation}
thus
\begin{equation}\label{key}
\hL^{(v)} = -\hL^{(u)} \quad \hS^{(v)} = -\hS^{(u)}
\end{equation}

\subsubsection{Charge Conjugation}
\begin{equation}\label{key}
E \ra E - q\phi \quad \vb{p} \ra \vb{p} - q\vb{A}
\end{equation}
\begin{equation}\label{key}
p_\mu \ra p_mu - qA_\mu \;\text{i.e.}\; \I\partial_\mu \ra \I\partial_\mu - qA_\mu
\end{equation}
for electron, $ q = -e $
\begin{equation}\label{key}
\gamma^\mu(\partial_\mu - \I e A_\mu)\psi + \I m\psi = 0
\end{equation}
take complex conj and pre-multiply $ -\I\gamma^2 $
\begin{equation}\label{key}
-\I\gamma^2 (\gamma^\mu)^* (\partial_\mu + \I e A_\mu)\psi^* - m\gamma^2\psi^* = 0
\end{equation}
since 
\begin{equation}\label{key}
\gamma^2\gamma^\mu = -\gamma^\mu\gamma^2
\end{equation}
for $ \mu \neq 2 $

\subsection{Spin and Helicity States $ \bigstar $}
$ \{u_1, u_2, v_1, v_2\} $ are not $ \hS_z $ eigenstates except $ \vb{p} = \pm p\hk $\\
and $ [\hH_D, \hS_z] \neq 0 $, thus we can't find a basis of simultaneous eigenstates.\\
def: helicity operator
\begin{equation}\label{key}
\hh \equiv \dfrac{\hS\cdot\vb{p}}{p}
\end{equation}
\begin{equation}\label{key}
\hh = \dfrac{1}{2p}\mqty(\bm\sigma\cdot\vb{p} & 0\\
                         0 & \bm\sigma\cdot\vb{p})
\end{equation}
\begin{equation}\label{key}
[\hh, \hH_D] = 0
\end{equation}
suppose $ \hh u = \lambda u $
\begin{equation}\label{key}
\dfrac{1}{2p}\mqty(\bm\sigma\cdot\vb{p} & 0\\
0 & \bm\sigma\cdot\vb{p}) u = \lambda u
\end{equation}
\begin{equation}\label{lamua}
(\bm\sigma\cdot\vb{p})u_A = 2p\lambda u_A
\end{equation}
since $ (\bm\sigma\cdot\vb{p})^2 = p^2 $
\begin{equation}\label{key}
\lambda = \pm\dfrac{1}{2}
\end{equation}
according to $ \eqref{uaub} $
\begin{equation}\label{key}
(\bm\sigma\cdot\vb{p})u_A = (E+ m) u_B
\end{equation}
$ \therefore $
\begin{equation}\label{key}
u_B = \dfrac{2\lambda p}{E + m}u_A
\end{equation}
To solve $ \eqref{lamua} $, let
\begin{equation}\label{key}
\vb{p} = ...
\end{equation}
\begin{equation}\label{key}
\hh = \dfrac{1}{2}\mqty(\cos\theta & \sin\theta \e^{-\I\phi}\\
                         \sin\theta \e^{\I\phi} & -\cos\theta)
\end{equation}
assume $ u_A = \mqty(a\\b) $
\begin{equation}\label{key}
\dfrac{b}{a} = \e^{\I\phi}\tan\dfrac{\theta}{2}
\end{equation}
thus
\begin{equation}\label{key}
u_\uparrow = N\mqty(\cos\dfrac{\theta}{2}\\
                    \e^{\I\phi}\sin\dfrac{\theta}{2}\\
                    ...\\
                    ...)
\end{equation}

\subsection{Intrinsic Parity of Dirac Fermions $ \bigstar $}
parity transformation
\begin{equation}\label{key}
t' = t \quad x' = -x
\end{equation}
parity operator
\begin{equation}\label{key}
\psi' = \hP \psi \hP^2 = I
\end{equation}
\begin{equation}\label{key}
~
\end{equation}



\begin{equation}\label{key}
\gamma^0 \hP \propto I
\end{equation}

\section{Interaction by Particle Exchange}
\subsection{1st and 2nd Order Perturbation Theory}

\subsubsection{Time-Ordered Perturbation Theory}

\subsection{Feynman Diagrams and Virtual Particles}

\subsection{Intro to QED}
Dirac Eq. + minimal substitution
\begin{equation}\label{key}
\partial_\mu \ra \partial_\mu + \I qA_\mu
\end{equation}
\begin{equation}\label{key}
\I\pdv{\psi}{t} + \I\gamma^0 \bm\gamma\cdot\na \psi - q\gamma^0 \gamma^\mu A_\mu \psi - m\gamma^0\psi = 0
\end{equation}
\begin{equation}\label{key}
\hH = (m\gamma^0 - \I\gamma^0\bm\gamma\cdot\na) + q\gamma^0 \gamma^\mu A_\mu
\end{equation}

这一节的例子可以不看。

\subsection{Feynman Rules for QED}

$ \ce{e^- e^- -> e^- e^-} $\\

$ \ce{e^- e^+ -> \mu^- \mu^+} $\\
\begin{equation}\label{key}
-\I \mathcal{M} = 
\end{equation}

\section{e-p Annihilation}
\begin{figure}[H]
	\centering
    \feynmandiagram [horizontal=a to b] {
    	i1  -- [fermion] a -- [fermion] i2,
    	a -- [photon, edge label=\(\gamma\), momentum'=\(k\)] b,
    	f1  -- [fermion] b -- [fermion] f2,
    };
\end{figure}

计算matrix element

\subsection{Perturbation Theory}

\subsection{e-p Annihilation}
\begin{equation}\label{key}
\mathcal{M} = ...
\end{equation}

\subsection{Spin Sums}




\begin{equation}\label{key}
\ev{\abs{\mathcal{M}}^2} = \dfrac{1}{4}\sum_{\text{spins}} \abs{\mathcal{M}}^2
\end{equation}

\subsection{Helicity Amplitudes}

\subsection{The $ \mu $ and e Currents}

\subsection{Cross Section}
\begin{equation}\label{key}
\mathcal{M}_{RL\ra RL} = e^2(1 + \cos\theta)
\end{equation}
\begin{equation}\label{key}
\ev{\abs{\mathcal{M}}^2} = e^4 (1 + \cos^2 \theta)
\end{equation}

\begin{equation}\label{key}
\dv{\sigma}{\Omega} = \dfrac{e^4 (1 + \cos^2 \theta)}{64\pi^2 s}
\end{equation}

\begin{equation}\label{key}
\sigma = \dfrac{4\pi\alpha^2}{3s}
\end{equation}

total spin = $ \pm 1 \; \dra$ contribution to helicity amplitude. (because photon has spin 1)\\

\subsection{}

\subsection{Spin in e-p Anni.}
\begin{equation}\label{key}
\ket{1,+1}_\theta = \dfrac{1}{2}(1 - \cos\theta)\ket{1, -1} + \dfrac{1}{\sqrt{2}}\sin\theta\ket{1,0} + \dfrac{1}{2}(1 +  \cos\theta)\ket{1,+1}
\end{equation}
thus
\begin{equation}\label{key}
\mathcal{M}_{RL\ra RL} \propto \Braket{1,+1 | 1,+1}_\theta = \dfrac{1}{2}(1 + \cos\theta)
\end{equation}


\subsection{Chirality $ \bigstar $}
def
\begin{equation}\label{key}
\gamma^5 = \I\gamma^0\gamma^1\gamma^2\gamma^3 = \mqty(0 & I \\ I & 0)
\end{equation}
in the limit $ E >> m $
\begin{equation}\label{key}
\g5 u_\uparrow = + u_\uparrow \quad \g5 u_\downarrow = -u_\downarrow
\end{equation}
def: left- and right-handed chiral states
\begin{equation}\label{key}
\g5 u_R = + u_R \quad \g5 u_L = -u_L
\end{equation}
\begin{equation}\label{key}
u_R = N\mqty(c\\ s\e^{\I\phi}\\ c\\ s\e^{\I\phi}) \quad u_L = 
\end{equation}
chiral states are identical to massless helicity states.

\subsubsection{Chiral Projection Operators}
\begin{equation}\label{key}
\begin{aligned}
P_R &= \dfrac{1}{2}(1 + \g5)\\
P_L &= \dfrac{1}{2}(1 - \g5)
\end{aligned}
\end{equation}
features
\begin{equation}\label{key}
P_R + P_L = 1 \quad P_R P_R = 1 \quad P_R P_L = 0
\end{equation}
\begin{equation}\label{key}
P_R u_R = u_R \quad ...
\end{equation}
\begin{equation}\label{key}
P_L u_R = 0 \quad P_L u_L = u_L
\end{equation}
thus any spinor can be decomposed with
\begin{equation}\label{key}
u = a_R u_R + a_L u_L = P_R u + P_L u
\end{equation}

\subsubsection{Chirality in QED}
\begin{equation}\label{key}
\bar\psi \gmuu \phi
\end{equation}

\section{E-p Elastic Scattering}
\subsection{Probing ...}

\subsection{Rutherford and Mott Scattering $ \bigstar $}
\begin{equation}\label{key}
\mathcal{M}_{fi} = \dfrac{e^2}{q^2} \bar{u}_3 \gmuu u_1 g_{\mu\nu} \bar{u}_4 \gnuu u_2
\end{equation}
Helicity states
\begin{equation}\label{key}
u_\uparrow = ... \qquad u_\downarrow = ...
\end{equation}
where
\begin{equation}\label{key}
\kappa = \dfrac{p}{E + m_e} = \dfrac{\beta_e \gamma_e}{\gamma_e + 1}
\end{equation}
NR: $ \kappa \ra 0 $, R: $ \kappa \approx 1 $\\
init $ \ce{e^-} $: $ (\theta, \phi)  = (0,0)$\\
fin $ \ce{e^-} $: $ (\theta, \phi)  = (\theta,0)$\\
\begin{equation}\label{key}
\begin{aligned}
j^e_{\uparrow\uparrow} &= (E + m_e)[(\kappa^2 + 1)c, 2\kappa s, 2\I\kappa s, 2\kappa c]\\
&=
\end{aligned}
\end{equation}
$ \kappa = 1 \dra j^e_{\uparrow\downarrow} = j^e_{\downarrow\uparrow} = 0 $, the same with last week.\\
Helicity is effectively conserved at vertices in high energy limits.\\

init p: $ \theta = 0, \phi = 0, \kappa = 0 $\\
fin p: $ \theta = \eta, \phi = 0, \kappa \approx 0 $
\begin{equation}\label{key}
\begin{aligned}
j^p_{\uparrow\uparrow} &= j^p_{\downarrow\downarrow} = 2m_p [c_\eta, 0, 0, 0]\\
j^p_{\uparrow\downarrow} &= j^p_{\downarrow\uparrow} = -2m_p [s_\eta, 0, 0, 0]
\end{aligned}
\end{equation}
all NR.\\
\begin{equation}\label{key}
\begin{aligned}
\ev{\abs{\mu}^2} &= \dfrac{1}{4}\sum \abs{\mathcal{M}_{fi}^2} = \dfrac{4m_p^2 m_e^2 e^4(\gamma_e+1)^2}{q^4} [(1-\kappa^2)^2 + 4\kappa^2c^2]\\
&= \dfrac{16m_p^2 m_e^2 e^4}{q^4} \qty[1 + \beta_e^2 \gamma_e^2 \cos^2\dfrac{\theta}{2}]\\
\end{aligned}
\end{equation}
\begin{equation}\label{key}
q^4 = ...
\end{equation}
i) $ \ce{e^-} $ is NR, $ 1 + \beta_e^2 \gamma_e^2 \cos^2\dfrac{\theta}{2} \approx 1 $ (Rutherford)\\
ii) $ \ce{e^-} $ is R, $ 1 + \beta_e^2 \gamma_e^2 \cos^2\dfrac{\theta}{2} \approx \beta_e^2 \gamma_e^2 \cos^2\dfrac{\theta}{2} $ (Mott)\\

\subsubsection{Rutherford Scattering}
\begin{equation}\label{key}
\ev{\abs{\mu}^2} = \dfrac{m_p^2 m_e^2 e^4}{p^4 \sin^4 (\theta/2)}
\end{equation}
\begin{equation}\label{key}
\dv{\sigma}{\Omega} = \dfrac{1}{64\pi^2} \qty(\dfrac{1}{m_p + E_1 - E_1\cos\theta}) \ev{\abs{\mu}^2}
\end{equation}
Since
\begin{equation}\label{key}
E_1 \sim m_e << m_p
\end{equation}
we have
\begin{equation}\label{key}
\dv{\sigma}{\Omega} = ...
\end{equation}

\subsubsection{Mott Scattering}
$ E \approx p_e $
\begin{equation}\label{key}
\ev{\abs{\mu}^2} = \dfrac{m_p^2 e^4}{E^2 \sin^4 (\theta/2)}\cos^2\dfrac{\theta}{2}
\end{equation}
\begin{equation}\label{key}
\dv{\sigma}{\Omega} = ...
\end{equation}

\subsection{Form Factors $ \bigstar $}
形状因子\\
charge density $ Q\rho(\vb{r}') $, where $ \int \rho(\vb{r}') = 1 $\\
potential
\begin{equation}\label{key}
V(\vb{r}) = \int \dfrac{Q\rho(\vb{r}')}{4\pi\abs{\vb{r} - \vb{r}'}} \dd[3]\vb{r}'
\end{equation}
In the Born approx., 
\begin{equation}\label{key}
\psi_i = \e^{\I(\vb{p}_1\cdot\vb{r} - Et)} \quad \psi_f = \e^{\I(\vb{p}_3\cdot\vb{r} - Et)}
\end{equation}
\begin{equation}\label{key}
\begin{aligned}
\mathcal{M}_{fi} &= \Braket{\psi_f | V(\vb{r}) | \psi_i} = \int \e^{\I \vb{q}\cdot\vb{r}} \int \dfrac{Q\rho(\vb{r}')}{4\pi\abs{\vb{r} - \vb{r}'}} \dd[3]\vb{r}' \dd[3]\vb{r}\\
&= \int \e^{\I \vb{q}\cdot\vb{R}}
 \int \dfrac{Q}{4\pi\abs{\vb{R}}}  \dd[3]\vb{R}
 \int \rho(\vb{r}')\e^{\I\vb{q}\cdot\vb{r}'}\dd[3]\vb{r}'\\
&\equiv \mathcal{M}_{fi}^{\text{pt}} F(\vb{q}^2)
\end{aligned}
\end{equation}
pt: point-like\\
Form factor
\begin{equation}\label{key}
F(\vb{q}^2) = \int \rho(\vb{r}')\e^{\I\vb{q}\cdot\vb{r}'}\dd[3]\vb{r}'
\end{equation}
i.e. Fourier transform of chg density distribution.
\begin{equation}\label{key}
F(0) = 1 \quad F(\infty) = 0
\end{equation}

Note 7.8 留数定理 $ \bigstar $
\subsection{Relativistic e-p Elastic Scattering}
\begin{equation}\label{key}
\begin{aligned}
&p_1 = (E_1, 0, 0 ,E_1)\\
&p_2
\end{aligned}
\end{equation}
\begin{equation}\label{key}
q^2 = -2E_1E_3(1 - \cos\theta)
\end{equation}
def $ Q^2 = -q^2 $
\begin{equation}\label{key}
\begin{aligned}
\dv{\sigma}{\Omega} &= \dfrac{1}{64\pi^2} \qty(\dfrac{E_3}{m_p E_1})^2 \ev{\abs{\mathcal{M}_{fi}}^2}\\
&= \dfrac{\alpha^2}{4E_1^2\sin^4(\theta/2)} \dfrac{E_3}{E_1} \qty(\cos^2\dfrac{\theta}{2} + \dfrac{Q^2}{2m_p^2}\sin^2\dfrac{\theta}{2})
\end{aligned}
\end{equation}
when $ Q^2 << m_p^2 $ and $ E_3 \approx E_1 $, reduced to Mott.

\subsection{the Rosenbluth Formula}



\subsubsection{Measuring G}
\begin{equation}\label{key}
\dv{\sigma}{\Omega} \Bigg/ \qty(\dv{\sigma}{\Omega})_{Mott} = \dfrac{G_E^2 + \tau G_M^2}{1 + \tau} + 2\tau G_M^2 \tan^2\dfrac{\theta}{2}
\end{equation}

G can be fit by a dipole function
\begin{equation}\label{key}
G_E^2(Q^2) = \dfrac{1}{(1 + Q^2/Q_0^2)^2}
\end{equation}
\begin{equation}\label{key}
G_M^2(Q^2) = 2.79G_E^2(Q^2)
\end{equation}
charge distr.
\begin{equation}\label{key}
\rho(r) = \rho_0 \e^{-r/a}
\end{equation}
$ a = 0.24 fm $



\subsubsection{Elastic Scattering at high $ Q^2 $}



homework 0508\\
chap7 2,3,4,5,8\\
7.8 $ Q_0 $数据错了

\section{Deep Inelastic Scattering (DIS)}
\subsection{E-p Inelastic Scattering}
$ \ce{e^- + p \ra e^- + X} $\\
invariant mass of the hadronic system $ X $: $ W^2 = p_4^2 $\\
elastic: $ W^2 = m_p^2 $\\
\subsubsection{Kinematic Variables}
\begin{equation}\label{key}
Q^2 = -(p_1 - p_3)^2 = -(2m_e^2 - 2p_1p_3) = -2m_e^2 + 2E_1E_3 - 2\vb{p}_1\cdot\vb{p}_3
\end{equation}
inelastic $ -> $ high E, thus
\begin{equation}\label{key}
Q^2 = 2E_1E_3(1 - \cos\theta) = 
\end{equation}
def
\begin{equation}\label{key}
x \equiv \dfrac{Q^2}{2p_2 q} \quad y \equiv \dfrac{p_2 q}{p_2 p_1} \quad \nu \equiv \dfrac{p_2 q}{m_p}
\end{equation}

\begin{equation}\label{key}
W^2 = (q + p_2)^2 = m_p^2 + q^2 + 2p_2 q
\end{equation}
\begin{equation}\label{key}
W^2 + Q^2 - m_p^2 = 2p_2 q
\end{equation}
\begin{equation}\label{key}
x \equiv \dfrac{Q^2}{Q^2 + W^2 - m_p^2} = \dfrac{Q^2}{2p_2 q}
\end{equation}
elastic: $ x = 1 $\\
inelastic: $ 0 \leq x \leq 1$, since ...\\
$ x $ expresses the elasticity.\\

\begin{equation}\label{key}
y = \dfrac{p_2 q}{p_2 p_1} = ... = 1 - \dfrac{E_3}{E_1}
\end{equation}
$ y $ expresses the inelasticity, or fractional energy loss of $ e^- $.

\begin{equation}\label{key}
\nu = ... = E_1 - E_3
\end{equation}
the kinematics of inelastic scattering can be described by any two of the Lorentz-invariant quantities $ x $, $ Q^2 $, $ y $ and $ \nu $, except $ \{y, \nu\} $.\\

\subsubsection{IS at Low $ Q^2 $}
\begin{equation}\label{key}
\begin{aligned}
W^2 &= m_p^2 + q^2 + 2p_2 q = m_p^2 + (p_1 - p_3)^2 + 2p_2(p_1 - p_3)\\
&= m_p^2 -2E_1E_3(1 - \cos\theta) + 2m_p(E_1 - E_3)\\
&= [m_p^2 + 2m_pE_1] - [2m_p + 2E_1(1 - \cos\theta)] E_3
\end{aligned}
\end{equation}

...\\

\subsection{Deep Inelastic Scattering $ \bigstar $}
Elastic Rosenbluth formula
\begin{equation}\label{key}
\dv{\sigma}{\Omega} = ...
\end{equation}
Using def of $ Q^2 $ and $ y $
\begin{equation}\label{key}
\dv{\sigma}{\Omega} = \dfrac{4\pi\alpha^2}{Q^4} \qty[\dfrac{G_E^2 + \tau G_M^2}{1 + \tau} \qty(1 - y - \dfrac{m_p^2 y^2}{Q^2}) + \dfrac{1}{2}y^2 G_M^2]
\end{equation}
def: $ f_1(Q^2) = ... $, $ f_2(Q^2) = ... $
\begin{equation}\label{key}
\dv{\sigma}{Q^2} = \dfrac{4\pi\alpha^2}{Q^4} \qty[\qty(1 - y - \dfrac{m_p^2 y^2}{Q^2}) f_2(Q^2) + \dfrac{1}{2}y^2 f_1(Q^2)]
\end{equation}

\subsubsection{Structure Func.}
\begin{equation}\label{key}
\dv{\sigma}{x}{Q^2} = \dfrac{4\pi\alpha^2}{Q^4} \qty[\qty(1 - y - \dfrac{m_p^2 y^2}{Q^2}) \dfrac{F_2(x,Q^2)}{x} + y^2 F_1(x,Q^2)]
\end{equation}
deep IS: $ Q^2 >> m_p^2 y^2 $
\begin{equation}\label{key}
\dv{\sigma}{x}{Q^2} = \dfrac{4\pi\alpha^2}{Q^4} \qty[\qty(1 - y) \dfrac{F_2(x,Q^2)}{x} + y^2 F_1(x,Q^2)]
\end{equation}
\paragraph{SLAC's 2 striking features}
Bjorken scaling
\begin{equation}\label{key}
F_1(x,Q^2) \ra F_1(x) \qquad F_2(x,Q^2) \ra F_2(x)
\end{equation}
almost independent of $ Q^2 $. point-like constituents within the proton.\\
Callan-Gross relation
\begin{equation}\label{key}
F_2(x) = 2x F_1(x)
\end{equation}

\subsection{Elec-quark Scattering}
[e-p scat. cross sect.] = [pdf(slow)] $ \cross $ [e-q scat. cross sect.(fast)]\\
pdf: parton distribution function $ \bigstar $\\

e-q scat. in CoM frame
\begin{equation}\label{key}
\dv{\sigma}{\Omega} = \dfrac{Q_q^2e^4}{8\pi^2 s} \dfrac{1 + \tfrac{1}{4}(1 + \cos\theta)^2}{1 - \cos\theta)^2}
\end{equation}
$ \LI $ form
\begin{equation}\label{key}
\dv{\sigma}{q^2} = \dv{\sigma}{\Omega}\abs{\dv{\Omega}{q^2}} = ...
\end{equation}

\subsection{The Quark Model}
\paragraph{Infinite Momentum Frame}
$\quad  E_p >> m_p $\\
the struck quark
\begin{equation}\label{key}
p_q = \xi p_2 = (\xi E_2, 0, 0, \xi E_2)
\end{equation}

homework May 15\\
8.2 8.3 8.6 8.8\\

\begin{equation}\label{key}
(\xi p_2 + q)^2 = m_q^2  \quad (\xi p_2)^2 = m_q^2
\end{equation}
\begin{equation}\label{key}
\xi = \dfrac{-q^2}{2p_2 q} = x
\end{equation}

\begin{equation}\label{key}
\dv{\sigma}{q^2} = \dfrac{2\pi\alpha^2 Q_q^2}{q^4}\qty[1 + \qty(1 + \dfrac{q^2}{s_q})^2]
\end{equation}
where $ s_q = (p_1 + xp_2)^2 = xs $\\
And 
\begin{equation}\label{key}
y_q = \dfrac{xp_2 q}{xp_2 p_1} = y \quad x_q = 1 (\text{e-q is elastic})
\end{equation}
\begin{equation}\label{key}
\dfrac{q^2}{s_q} = \dfrac{-(s_q - m_q^2)x_q y_q}{s_q} = -y
\end{equation}
\begin{equation}\label{key}
\dv{\sigma}{q^2} = \dfrac{2\pi\alpha^2 Q_q^2}{q^4}\qty[1 + \qty(1 - y)^2]
\end{equation}
\begin{equation}\label{key}
\dv{\sigma}{Q^2} = \dfrac{4\pi\alpha^2 Q_q^2}{Q^4}\qty[1 - y + \dfrac{y^2}{2}]
\end{equation}

\subsubsection{PDF}
\begin{equation}\label{key}
\dv{\sigma}{x}{Q^2} = \dfrac{4\pi\alpha^2}{Q^4}\qty[(1-y) + \dfrac{y^2}{2}]\sum_i Q_i^2 q_i(x)
\end{equation}
which predicts Bjorken scaling and CG relation.\\

\subsubsection{Determination of the PDFs}
But PDFs cannot be computed in QFT.
\begin{equation}\label{key}
F_2^{ep} = x\qty(\dfrac{4}{9}u^p(x) + \dfrac{1}{9}d^p(x) + \dfrac{4}{9}\bar{u}^p(x) + \dfrac{1}{9}\bar{d}^p(x))
\end{equation}
valence quark: $ p = uud, n = udd $\\
sea quark: any flavor can be produced via gluon, so anti-quark be taken into the eq above. $ \bigstar $\\
\begin{equation}\label{key}
F_2^{en} = x\qty(\dfrac{4}{9}u^n(x) + \dfrac{1}{9}d^n(x) + \dfrac{4}{9}\bar{u}^n(x) + \dfrac{1}{9}\bar{d}^n(x))
\end{equation}
From symm
\begin{equation}\label{key}
u^n(x) = d^p(x)
\end{equation}
thus, def
\begin{equation}\label{key}
F_2^{ep} = x\qty(\dfrac{4}{9}u(x) + \dfrac{1}{9}d(x) + \dfrac{4}{9}\bar{u}(x) + \dfrac{1}{9}\bar{d}(x))
\end{equation}
\begin{equation}\label{key}
F_2^{en} = x\qty(\dfrac{4}{9}d(x) + \dfrac{1}{9}u(x) + \dfrac{4}{9}\bar{d}(x) + \dfrac{1}{9}\bar{u}(x))
\end{equation}
\begin{equation}\label{key}
\int_0^1 F_2^{ep}(x)\dd x = ... \equiv \dfrac{4}{9}f_u + \dfrac{1}{9}f_d
\end{equation}
\begin{equation}\label{key}
\int_0^1 F_2^{en}(x)\dd x \equiv \dfrac{4}{9}f_d + \dfrac{1}{9}f_n
\end{equation}
By experiment
\begin{equation}\label{key}
f_u = 0.36 \quad f_d = 0.18
\end{equation}

\subsubsection{Valence and Sea Quarks}

$ x\ra 0 $, $ R \ra 1 $\\
$ x \ra 1 $, $ R \ra 1/4 $, while $ R_{naive} = 2/3 $.\\
why?









\end{document}