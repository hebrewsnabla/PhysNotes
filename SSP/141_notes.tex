\documentclass[UTF8]{ctexart} % use larger type; default would be 10pt

\usepackage[utf8]{inputenc} % set input encoding (not needed with XeLaTeX)

%%% Examples of Article customizations
% These packages are optional, depending whether you want the features they provide.
% See the LaTeX Companion or other references for full information.

%%% PAGE DIMENSIONS
\usepackage{geometry} % to change the page dimensions
\geometry{a4paper} % or letterpaper (US) or a5paper or....
% \geometry{margin=2in} % for example, change the margins to 2 inches all round
% \geometry{landscape} % set up the page for landscape
%   read geometry.pdf for detailed page layout information

\usepackage{graphicx} % support the \includegraphics command and options

% \usepackage[parfill]{parskip} % Activate to begin paragraphs with an empty line rather than an indent

%%% PACKAGES
\usepackage{booktabs} % for much better looking tables
\usepackage{array} % for better arrays (eg matrices) in maths
\usepackage{paralist} % very flexible & customisable lists (eg. enumerate/itemize, etc.)
\usepackage{verbatim} % adds environment for commenting out blocks of text & for better verbatim
\usepackage{subfig} % make it possible to include more than one captioned figure/table in a single float
% These packages are all incorporated in the memoir class to one degree or another...

%%% HEADERS & FOOTERS
\usepackage{fancyhdr} % This should be set AFTER setting up the page geometry
\pagestyle{fancy} % options: empty , plain , fancy
\renewcommand{\headrulewidth}{0pt} % customise the layout...
\lhead{}\chead{}\rhead{}
\lfoot{}\cfoot{\thepage}\rfoot{}

%%% SECTION TITLE APPEARANCE
\usepackage{sectsty}
\allsectionsfont{\sffamily\mdseries\upshape} % (See the fntguide.pdf for font help)
% (This matches ConTeXt defaults)

%%% ToC (table of contents) APPEARANCE
\usepackage[nottoc,notlof,notlot]{tocbibind} % Put the bibliography in the ToC
\usepackage[titles,subfigure]{tocloft} % Alter the style of the Table of Contents
\renewcommand{\cftsecfont}{\rmfamily\mdseries\upshape}
\renewcommand{\cftsecpagefont}{\rmfamily\mdseries\upshape} % No bold!

%%% END Article customizations

%%% The "real" document content comes below...

\setlength{\parindent}{0pt}
\usepackage{physics}
\usepackage{amsmath}
%\usepackage{symbols}
\usepackage{AMSFonts}
\usepackage{bm}
%\usepackage{eucal}
\usepackage{mathrsfs}
\usepackage{amssymb}
\usepackage{float}
\usepackage{multicol}
\usepackage{abstract}
\usepackage{empheq}
\usepackage{extarrows}
\usepackage{textcomp}
\usepackage{mhchem}
\usepackage{braket}


\DeclareMathOperator{\p}{\prime}
\DeclareMathOperator{\ti}{\times}

\DeclareMathOperator{\e}{\mathrm{e}}
\renewcommand{\I}{\mathrm{i}}
\DeclareMathOperator{\kb}{k_{\mathrm{B}}}
\DeclareMathOperator{\Arg}{\mathrm{Arg}}

\DeclareMathOperator{\ra}{\rightarrow}
\DeclareMathOperator{\llra}{\longleftrightarrow}
\DeclareMathOperator{\lra}{\longrightarrow}
\DeclareMathOperator{\dlra}{\Leftrightarrow}
\DeclareMathOperator{\dra}{\Rightarrow}

\DeclareMathOperator{\na}{\bm{\nabla}}
\DeclareMathOperator{\nna}{\nabla^2}
\DeclareMathOperator{\drrr}{\dd[3]\vb{r}}

\DeclareMathOperator{\psis}{\psi^\ast}
\DeclareMathOperator{\Psis}{\Psi^\ast}
\DeclareMathOperator{\hi}{\hat{\vb{i}}}
\DeclareMathOperator{\hj}{\hat{\vb{j}}}
\DeclareMathOperator{\hk}{\hat{\vb{k}}}
\DeclareMathOperator{\hr}{\hat{\vb{r}}}
\DeclareMathOperator{\hT}{\hat{\vb{T}}}
\DeclareMathOperator{\hH}{\hat{\vb{H}}}
\DeclareMathOperator{\hp}{\hat{\vb{p}}}
\DeclareMathOperator{\hx}{\hat{\vb{x}}}
\DeclareMathOperator{\ha}{\hat{\vb{a}}}
\DeclareMathOperator{\Tdv}{-\dfrac{\hbar^2}{2m}\dv[2]{x}}
\DeclareMathOperator{\Tna}{-\dfrac{\hbar^2}{2m}\nabla^2}

%\DeclareMathOperator{\s}{\sum_{n=1}^{\infty}}
\DeclareMathOperator{\intinf}{\int_0^\infty}
\DeclareMathOperator{\intdinf}{\int_{-\infty}^\infty}
%\DeclareMathOperator{\suminf}{\sum_{n=0}^\infty}
\DeclareMathOperator{\sumnzinf}{\sum_{n=0}^\infty}
\DeclareMathOperator{\sumnoinf}{\sum_{n=1}^\infty}
\DeclareMathOperator{\sumndinf}{\sum_{n=-\infty}^\infty}
\DeclareMathOperator{\sumizinf}{\sum_{i=0}^\infty}


\newcommand{\dis}{\displaystyle}
\numberwithin{equation}{section}

\title{Notes of 141A\\
Solid State Physics}
\author{hebrewsnabla}
%\date{} % Activate to display a given date or no date (if empty),
         % otherwise the current date is printed 

\begin{document}
% \boldmath
\maketitle

\tableofcontents

\newpage

vibrations, phonon\\
conductivity\\
magnetism, spin\\
many-body physics, superconductivity\\

\setcounter{section}{-1}
\section{Review}
\subsection{Atoms}
\begin{equation}\label{key}
\hat{\vb{H}} = -\dfrac{1}{2}\nabla^2 - \dfrac{1}{r}
\end{equation}

\subsection{Bonding}
 --covalent\\
 --ionic\\
 --vdW\\
 --metallic\\
\subsubsection{Covalent}
\ce{H2}\\
\begin{equation}\label{key}
\hat{\vb{H}} = -\dfrac{1}{2}\nabla^2 - \dfrac{1}{r_A} - \dfrac{1}{r_B}
\end{equation}
\begin{equation}\label{key}
\hat{\vb{H}}\Psi = E\Psi
\end{equation}
use variational principle
\begin{equation}\label{key}
\Phi = C_A\Psi_A + C_B\Psi_B
\end{equation}
\begin{equation}\label{key}
\begin{aligned}
\langle H\rangle &= \bra{\Phi}\hat{\vb{H}}\ket{\Phi}\\
&= \bra{C_A\Psi_A + C_B\Psi_B}\hat{\vb{H}}\ket{C_A\Psi_A + C_B\Psi_B}\\
&= \cdots
\end{aligned}
\end{equation}
\begin{equation}\label{key}
\begin{aligned}
E &= \dfrac{\bra{\Phi}\hat{\vb{H}}\ket{\Phi}}{\braket{\Phi}{\Phi}}\\
&= \cdots
\end{aligned}
\end{equation}
\begin{equation}\label{key}
\pdv{\Phi}{C_A} = 0 \quad \pdv{\Phi}{C_B} = 0
\end{equation}
\begin{equation}\label{key}
\mqty(H_{AA}-E & H_{AB}-ES\\
      H_{AB}-ES & H_{BB}-E)
\mqty(C_A\\C_B) = 0
\end{equation}

\subsubsection{vdW Bonds}
dipole interaction
\begin{equation}\label{key}
\vb{E}_{dipole} \propto \dfrac{1}{r^3}
\end{equation}
\begin{equation}\label{key}
\vb{P}_{induced} \propto \vb{E}
\end{equation}
thus
\begin{equation}\label{key}
U_{dipole} \propto \vb{E}\cdot\vb{P} \propto -\dfrac{1}{r^6}
\end{equation}
~\\
repulsive $ \propto \dfrac{1}{r^12} $\\
L-J potential ...\\

\subsubsection{Ionic Bonding}
electropositive atom -- low ionization potential (电离能)\\
electronegative atom -- large electron affinity (电子亲和性)\\
E.g. \ce{NaCl}\\
Na -- $ E_I = 5.1eV $, Cl -- $ E_A = 3.6eV $\\
Charge transfer -- $ -7.9eV $\\
total $ 5.1 - 3.6 - 7.9 $\\
\begin{equation}\label{key}
U_{ij} = \pm\dfrac{e^2}{4\pi\epsilon+0 r_{ij}} + \dfrac{B}{r_{ij}^2}
\end{equation}

\subsubsection{Metallic Bonding}
Cu, Ag, Au -- extended covalent bonds\\
ion cohesion\\

\section{Crystal Structure}
"diamond" crystal\\
FCC lattice -- Cu\\

\subsection{definition}
Bravais Lattice\\
\begin{enumerate}
	\item infinite set of points so that everything looks the same no matter where you stand
	\item set of points defined by $ \vb{r}_n = n_1\vb{a}_1 + n_2\vb{a}_2 + n_3\vb{a}_3 $ 
\end{enumerate}

primitive cell: smallest periodic space filling volume\\

E.g. \ce{CuO2} layers

1 dot / unit cell\\
Wigner Seitz unit cell\\

symmetry of crystals: mapping that returns original lattice\\
-- translational\\
-- rotational\\
-- inversion\\
-- reflection\\
~\\
bravais 14 lattice\\
array of descrete points generated by translational operations
\begin{equation}\label{key}
\vb{R} = n_1\vb{a}_1 + n_2 ...
\end{equation}
where $ \vb{a}_i $ lattice translation vectors
\begin{enumerate}
	\item triclinic - 2 三方
	\item monoclinic - 2
	\item orthorhumbic - 4
	\item tetragonal - 2
	\item cubic - 3
		\subitem simple cubic 简单立方
		\subitem face-centered cubic / FCC / 面心立方\\
		 		-- Cu, Ag, Au, Ni, Al (soft metals); NaCl; Diamond, Si, Ge
		\subitem body-centered cubic / BCC / 体心立方\\
				-- Fe, V, Ta, Nb, W, Mo, Cr (brittle except Ta)
	\item trigonal - 1
	\item hexagonal - 1
\end{enumerate}

Index Crystal Planes: Miller Indices

crystals of inert atoms\\
L-J potential\\
\begin{equation}\label{key}
U(r) = ...
\end{equation}
first term: phenomenological repulsive potential to account for Pauli exclusion\\
second term: attractive potential due to spontaneous induced dipole.\\

Ionic crystals\\
\begin{equation}\label{key}
U_{tot} = N(z\lambda\e^{-\tfrac{q}{\rho}} - \dfrac{\alpha q^2}{R})
\end{equation}
$ z $ -- coordination number of the lattice\\
$ \alpha = \dis\sum_j \dfrac{(\pm)}{p_{ij}} $ -- Madelung Cons.


\newpage
8/29\\
\section{Wave Diffraction and Reciprocal Lattice}
Periodicity is important -- band gap, diffraction\\

\subsection{reciprocal lattice}
for 1D lattice, $ \{\vb{T}\} = \{ja\hat{\vb{x}}\} $.\\
Suppose $ n(x) $ is the elec density\\
thus $ n(x) = n(x+T) $\\
...
\begin{equation}\label{key}
n(x) = \sum_j n_j\e^{\I g_j x}
\end{equation}
where $ g_j = \dfrac{2\pi}{\lambda_j} $\\
since $ \lambda_j = \dfrac{a}{j} $
\begin{equation}\label{key}
g_j = \dfrac{2\pi}{a}j
\end{equation}
aka reciprocal lattice vectors.

\begin{equation}\label{key}
\begin{aligned}
\int_0^a n(x)\e^{-g_j' x}\dd x\\
&= n_j' a + \sum_{j\neq j'}\int_0^a n_j \e^{\tfrac{2\pi}{a}(j - j')x}\dd x\\
&= n_j' a
\end{aligned}
\end{equation}
\begin{equation}\label{key}
n_j = \dfrac{1}{a}\int_0^a n(x)\e^{-g_j x}\dd x
\end{equation}

Generalize to 3D\\
\begin{equation}\label{key}
n(\vb{r}) = \sum_{\vb{G}} n_{\vb{G}}\e^{\I\vb{G}\cdot\vb{r}}
\end{equation}
\begin{equation}\label{key}
\vb{G} = \nu_1 \vb{b}_1 + \nu_2 \vb{b}_2 + \nu_3 \vb{b}_3
\end{equation}
\begin{equation}\label{key}
n(\vb{r}) = n(\vb{r} + \vb{T})
\end{equation}
\begin{equation}\label{key}
\vb{T} = u_1\vb{a}_1 + u_2\vb{a}_2 + u_3\vb{a}_3
\end{equation}
\begin{equation}\label{key}
\vb{G}\cdot\vb{T} = 2\pi n
\end{equation}

\subsection{diffraction}
Bragg Law ...\\
diffraction\\
picture ...
\begin{equation}\label{key}
\vb{E}_T = \vb{E}_1 +\vb{E}_2
\end{equation}
$ \vb{k} = \dfrac{2\pi}{\lambda}\hat{\vb{n}} $, $ \vb{k}' = \dfrac{2\pi}{\lambda}\hat{\vb{n}'} $
\begin{equation}\label{key}
\Delta r = \vb{r}\cdot\hat{\vb{n}} - \vb{r}\cdot\hat{\vb{n}'}
\end{equation}
\begin{equation}\label{key}
\Delta\phi = \vb{r}\cdot k\hat{\vb{n}} - \vb{r}\cdot k\hat{\vb{n}'}
= \vb{r}\cdot(\vb{k} - \vb{k}')
\end{equation}
\begin{equation}\label{key}
\begin{aligned}
I &= \abs{\vb{E}_1 + \vb{E}_2}^2\\
&= \abs{\int\vb{E}\dd[3]\vb{r}}^2\\
&\propto \abs{\int n(\vb{r})\e^{\I\Delta\phi}\dd[3]\vb{r}}^2\\
&\propto \abs{\int n(\vb{r})\e^{\I\vb{r}\cdot(\vb{k} - \vb{k}')}\dd[3]\vb{r}}^2
\end{aligned}
\end{equation}
the amplitude
\begin{equation}\label{key}
\begin{aligned}
F_{k'k} &= \int n(\vb{r})\e^{\I\vb{r}\cdot(\vb{k} - \vb{k}')} \dd[3]\vb{r}\\
&= \int \e^{-\I\vb{r}\cdot\vb{k}'} n(\vb{r}) \e^{\I\vb{r}\cdot\vb{k}} \dd[3]\vb{r}\\
&= \bra{k'}n(r)\ket{k}
\end{aligned}
\end{equation}
aka Born approx.\\
~\\
\begin{equation}\label{key}
n(r) = \sum_G n_G \e^{\I\vb{G}\cdot\vb{r}}
\end{equation}

\begin{equation}\label{key}
\begin{aligned}
F_{k'k} &= \sum_G n_G \int \e^{\I(\vb{G} - (\vb{k} - \vb{k}'))\cdot\vb{r}} \dd[3]\vb{r}\\
&= \sum_G n_G \delta(\vb{G} - (\vb{k} - \vb{k}'))
\end{aligned}
\end{equation}
Xray spot -- $ \vb{G} = \vb{k} - \vb{k}' $

\newpage
9/7\\
\begin{equation}\label{key}
\vb{k}' - \vb{k} = \vb{G}
\end{equation}
\begin{equation}\label{key}
\vb{k}\cdot\qty(\dfrac{1}{2}\vb{G}) = \abs{\dfrac{1}{2}\vb{G}}^2
\end{equation}

\setcounter{section}{3}
\section{Phonons I. Crystal Vibrations}
Phonons: Quantized lattice vibrations\\

\begin{equation}\label{key}
\omega = 2\sqrt{\dfrac{C}{m}}\abs{\sin\dfrac{1}{2}ka}
\end{equation}
-- phonon dispersion relation\\
zeros: $ \dfrac{1}{2}ka = n\pi \;\dra\; k = \dfrac{2\pi n}{a}$

\newpage
9/10\\
photo ..
\begin{equation}\label{key}
u_s^k(t) = u_0\e^{\I(ksa-\omega t)}
\end{equation}
\begin{equation}\label{key}
\omega(k) = 2\sqrt{\dfrac{c}{m}}\abs{\sin\dfrac{1}{2}ka}
\end{equation}
\begin{equation}\label{key}
G = \dfrac{2\pi}{a}P, \quad P\in\mathbb{Z}
\end{equation}
\begin{equation}\label{key}
u_s^{k+G} = 
\end{equation}

$ \omega $ vs $ k $ -- $ E=\hbar\omega $ vs $ p=\hbar k $\\

limit\\
1. long wavelength limit\\
$ \lambda \ra\infty $, $ k\ra 0 $
\begin{equation}\label{key}
\omega(k) \ra 2\sqrt{\dfrac{c}{m}}\abs{\dfrac{1}{2}ka} \propto k
\end{equation}
acoustic \\
\begin{equation}\label{key}
v_p = \dfrac{\omega}{k} = 
\end{equation}
\begin{equation}\label{key}
v_g = \dv{\omega}{k} = 
\end{equation}
2. short wl limit\\
$ k\ra\dfrac{\pi}{a} $\\
\begin{equation}\label{key}
v_g = 0
\end{equation}

\newpage
9/12\\
Phonon scattering\\

9/19\\
\section{Phonons II. Thermal Properties}
\subsection{Phonon Heat Capacity}
%\begin{equation}\label{key}
%\begin{aligned}
%U(r) &= \sum_k \hbar\omega_k n_k(T)\\
%&= \int_0^{\omega_D}\hbar\omega n(\omega)D(\omega)\dd\omega
%\end{aligned}
%\end{equation}
%\begin{equation}\label{key}
%C(T) = \pdv{U(T)}{T} = ...
%\end{equation}
\begin{equation}\label{key}
U(T) = \sum_{k,p} \expval{n_{k,p}}_T \hbar\omega_{k,p}
\end{equation}
\begin{equation}\label{key}
\begin{aligned}
\expval{n} &= \dfrac{\dis\sumnzinf n \e^{-n\hbar\omega/\kb T}}{\dis\sumnzinf \e^{-n\hbar\omega/\kb T}} \quad \text{(Bose-Einstein Distr.)}\\
&= ...\\
&= \dfrac{1}{\e^{\hbar\omega/\kb T} - 1}
\end{aligned}
\end{equation}
\begin{equation}\label{key}
\begin{aligned}
U(T) = \int_0^{\omega_{max}} \hbar\omega f(\omega)
\end{aligned}
\end{equation}
High temp, Debye model, $ T >> \theta_D $
\begin{equation}\label{key}
C = 3Nk_B
\end{equation}
Low temp, Einstein model, $ T << \theta_D $
\begin{equation}\label{key}
C(T) = ... = \dfrac{12\pi^4 Nk_B}{5}\qty(\dfrac{T}{\theta_D})^3
\end{equation}
$ \hbar\omega_{max} = k_B T $, $ k_{max} = \dfrac{k_B T}{\hbar v} $\\
fraction of modes excited = $ \qty(\dfrac{k_{max}}{k_D})^3 $\\
\# of modes excited = $ 3N \qty(\dfrac{k_{max}}{k_D})^3 $
\begin{equation}\label{key}
\begin{aligned}
U(T) &= (\text{\# of modes excited})\langle E\rangle_{mode}\\
&= 3N\qty(\dfrac{k_{max}}{k_D})^3k_B T\\
&= 3N\qty(\dfrac{T}{\theta_D})^3k_B T\\
&= 
\end{aligned}
\end{equation}
\begin{equation}\label{key}
C(T) = ...
\end{equation}

\subsection{Anharmonic Crystal Interactions / Thermal Expansion}
true potential of phonons are not hyperbolic but L-J-like\\
\begin{equation}\label{key}
U = Cx^2 - gx^3 + ...
\end{equation}
\begin{equation}\label{key}
\langle x\rangle_T = \dfrac{\int_x x P_T(x)\dd x}{\int_x P_T(x)\dd x}
\end{equation}
\begin{equation}\label{key}
P_T(x) \propto \e^{-U(x)/k_B T}
\end{equation}
\begin{equation}\label{key}
\begin{aligned}
\langle x\rangle_T &= \dfrac{\int_x x \e^{-U(x)/k_B T}\dd x}{\int_x \e^{-U(x)/k_B T}\dd x}\\
&= ...\\
&= \dfrac{3g}{4C^2}k_B T
\end{aligned}
\end{equation}

\subsection{Thermal Conductivity}
energy flux
\begin{equation}\label{key}
j = \dfrac{\text{enrgy}}{\text{time}\cdot\text{area}} = \dfrac{P}{A} = -K\dv{T}{x}
\end{equation}
K -- thermal conductivity\\
similar to $ j = \sigma E $\\
\begin{equation}\label{key}
j = \rho v
\end{equation}

\newpage
10/2 ankit\\
\subsection{Sommerfeld Model}
3D
\begin{equation}\label{key}
N = \dfrac{4}{3}\pi k^3\qty(\dfrac{L}{2\pi})^3 = ...
\end{equation}
\begin{equation}\label{key}
D(\varepsilon) = ... \propto \varepsilon^{1/2}
\end{equation}
2D
\begin{equation}\label{key}
N = \pi k^2\qty(\dfrac{L}{2\pi})^2 = ...
\end{equation}
\begin{equation}\label{key}
D(\varepsilon) = ... \propto 1
\end{equation}\begin{equation}\label{key}
1D
N = 2k\qty(\dfrac{L}{2\pi}) = ...
\end{equation}
\begin{equation}\label{key}
D(\varepsilon) = ... \propto \varepsilon^{-1/2}
\end{equation}

\subsection{Heat Capacity}
\begin{equation}\label{key}
U = \int_0^\infty \dd\varepsilon D(\varepsilon)f(\varepsilon)\varepsilon
\end{equation}
\begin{equation}\label{key}
N = \int_0^\infty \dd\varepsilon D(\varepsilon)f(\varepsilon)
\end{equation}
In metals, $ T << T_F $\\
Use Sommerfeld expansion\\
Consider integral of the form
\begin{equation}\label{key}
\intdinf H(\varepsilon)f(\varepsilon)\dd\varepsilon
\end{equation}
\begin{equation}\label{key}
K(\varepsilon) = \int_{-\infty}^\varepsilon H(\varepsilon')\dd\varepsilon'
\end{equation}
\begin{equation}\label{key}
H = \dv{K}{\varepsilon}
\end{equation}
\begin{equation}\label{key}
\intdinf H(\varepsilon)f(\varepsilon)\dd\varepsilon = \intdinf K(\varepsilon)(-\dv{f}{\varepsilon})\dd\varepsilon
\end{equation}
\begin{equation}\label{key}
K(\varepsilon) = K(\mu) + \sum_{n=1}^\infty \dfrac{1}{n!}\dv[n]{K}{\varepsilon} (\varepsilon - \mu)^n
\end{equation}
\begin{equation}\label{key}
\begin{aligned}
\intdinf H(\varepsilon)f(\varepsilon)\dd\varepsilon &= \intdinf K(\mu)(-\dv{f}{\varepsilon})\dd\varepsilon + \sum_{n=1}^\infty \intdinf \dfrac{1}{n!}\dv[n]{K}{\varepsilon} (\varepsilon - \mu)^n \qty(-\dv{f}{\varepsilon})\dd\varepsilon\\
&= K(\mu) + ...\\
&= \int_{-\infty}^\mu H(\varepsilon)\dd\varepsilon + \dfrac{\pi^2}{6}(k_B T)^2 H(\mu) + \mathcal{O}((K_B T)^4)
\end{aligned}
\end{equation}
thus
\begin{equation}\label{key}
U = ...
\end{equation}
\begin{equation}\label{key}
N = ...
\end{equation}

\section{Free Electron Fermi Gas}
\subsection{Electrical Conductivity and Ohm's Law}
\begin{equation}\label{key}
\hbar\qty(\dv{t} + \dfrac{1}{\tau})\vb{k} = -e\vb{E}
\end{equation}
\subsection{Motion in Magnetic Fields}
$ \vb{B} = B\hk $, $ \vb{v} = \dfrac{\hbar\vb{k}}{m} $
\begin{equation}\label{key}
m\qty(\dv{t} + \dfrac{1}{\tau})\vb{v} = -e\qty(\vb{E} + \dfrac{\vb{v}\ti\vb{B}}{c})
\end{equation}
...
\begin{equation}\label{key}
\mqty(J_x\\ J_y) = \bm{\sigma}\mqty(E_x\\ E_y)
\end{equation}
$ J_x = n(-e)v_x $
\begin{equation}\label{key}
\bm{\sigma} = \dfrac{\sigma_{DC}}{1 + \omega_0^2\tau^2}\mqty(1 & -\omega_0\tau\\ \omega_0\tau & 1)
\end{equation}
$ \sigma_{DC} = \dfrac{n e^2\tau}{m} $

\begin{equation}\label{key}
\bm{\rho} = \bm{\sigma}^{-1} = \dfrac{1}{\sigma
_{DC}}\mqty(1 & \omega_0\tau\\ \omega_0\tau & 1)
\end{equation}
\subsubsection{Hall Effect}

\subsection{AC Conductivity}
$ \vb{E} \sim \vb{E}_0 \e^{\-I\omega t} $, $ \omega >> 1/\tau $, $ \omega_c = \dfrac{e B}{m c} $, $ \vb{B} = B\hk $
\begin{equation}\label{key}
m\qty(\dv{t} + \dfrac{1}{\tau})\vb{v} = -e\qty(\vb{E} + \dfrac{\vb{v}\ti\vb{B}}{c})
\end{equation}
assume $ \vb{v} = \vb{v}_0\e^{\I\omega t} $\\
...
\begin{equation}\label{key}
\mqty(J_x\\ J_y) = \bm{\sigma}\mqty(E_x\\ E_y)
\end{equation}
\begin{equation}\label{key}
\begin{aligned}
\sigma_{xx} = \sigma_{yy} &= \dfrac{\omega_p^2\tau(1 - \I\omega\tau)}{4\pi[(1 - \I\omega\tau)^2 + (\omega_c\tau)^2]}\\
\sigma_{xy} = -\sigma_{yx} &= \dfrac{\omega_p^2\omega_c\tau^2}{4\pi[(1 - \I\omega\tau)^2 + (\omega_c\tau)^2]}
\end{aligned}
\end{equation}
\begin{equation}\label{key}
\omega_p = \dfrac{4\pi ne^2}{m}
\end{equation}
\begin{equation}\label{key}
\bm{\sigma} = \dfrac{\omega_p^2}{4\pi\omega}\mqty(\I & \omega_c/\omega\\ -\omega_c/\omega & \I) 
\end{equation}

\subsection{Complex Dielectric Function}
\begin{equation}\label{key}
\begin{aligned}
\na\cdot\vb{D} = 4\pi\rho_f\\
\na\cdot\vb{B} = 0\\
\na\ti\vb{E} + \dfrac{1}{c}\pdv{\vb{H}}{t} = 0\\
\na\ti\vb{H} - \dfrac{1}{c}\pdv{\vb{D}}{t} = \dfrac{4\pi}{c}\vb{J}_f
\end{aligned}
\end{equation}
\begin{equation}\label{key}
\na\cdot\vb{J} = -\pdv{\rho}{t}
\end{equation}





\end{document}