\documentclass[UTF8]{ctexart} % use larger type; default would be 10pt

\usepackage[utf8]{inputenc} % set input encoding (not needed with XeLaTeX)

%%% Examples of Article customizations
% These packages are optional, depending whether you want the features they provide.
% See the LaTeX Companion or other references for full information.

%%% PAGE DIMENSIONS
\usepackage{geometry} % to change the page dimensions
\geometry{a4paper} % or letterpaper (US) or a5paper or....
% \geometry{margin=2in} % for example, change the margins to 2 inches all round
% \geometry{landscape} % set up the page for landscape
%   read geometry.pdf for detailed page layout information

\usepackage{graphicx} % support the \includegraphics command and options

% \usepackage[parfill]{parskip} % Activate to begin paragraphs with an empty line rather than an indent

%%% PACKAGES
\usepackage{booktabs} % for much better looking tables
\usepackage{array} % for better arrays (eg matrices) in maths
\usepackage{paralist} % very flexible & customisable lists (eg. enumerate/itemize, etc.)
\usepackage{verbatim} % adds environment for commenting out blocks of text & for better verbatim
\usepackage{subfig} % make it possible to include more than one captioned figure/table in a single float
% These packages are all incorporated in the memoir class to one degree or another...

%%% HEADERS & FOOTERS
\usepackage{fancyhdr} % This should be set AFTER setting up the page geometry
\pagestyle{fancy} % options: empty , plain , fancy
\renewcommand{\headrulewidth}{0pt} % customise the layout...
\lhead{}\chead{}\rhead{}
\lfoot{}\cfoot{\thepage}\rfoot{}

%%% SECTION TITLE APPEARANCE
\usepackage{sectsty}
\allsectionsfont{\sffamily\mdseries\upshape} % (See the fntguide.pdf for font help)
% (This matches ConTeXt defaults)

%%% ToC (table of contents) APPEARANCE
\usepackage[nottoc,notlof,notlot]{tocbibind} % Put the bibliography in the ToC
\usepackage[titles,subfigure]{tocloft} % Alter the style of the Table of Contents
\renewcommand{\cftsecfont}{\rmfamily\mdseries\upshape}
\renewcommand{\cftsecpagefont}{\rmfamily\mdseries\upshape} % No bold!

%%% END Article customizations

%%% The "real" document content comes below...

\setlength{\parindent}{0pt}
\usepackage{physics}
\usepackage{amsmath}
%\usepackage{symbols}
\usepackage{AMSFonts}
\usepackage{bm}
%\usepackage{eucal}
\usepackage{mathrsfs}
\usepackage{amssymb}
\usepackage{float}
\usepackage{multicol}
\usepackage{abstract}
\usepackage{empheq}
\usepackage{extarrows}
\usepackage{textcomp}
\usepackage{mhchem}
\usepackage{braket}


\DeclareMathOperator{\p}{\prime}
\DeclareMathOperator{\ti}{\times}

\DeclareMathOperator{\e}{\mathrm{e}}
\renewcommand{\I}{\mathrm{i}}
\DeclareMathOperator{\kb}{k_{\mathrm{B}}}
\DeclareMathOperator{\Arg}{\mathrm{Arg}}

\DeclareMathOperator{\ra}{\rightarrow}
\DeclareMathOperator{\llra}{\longleftrightarrow}
\DeclareMathOperator{\lra}{\longrightarrow}
\DeclareMathOperator{\dlra}{\Leftrightarrow}
\DeclareMathOperator{\dra}{\Rightarrow}

\newcommand{\bkk}[1]{\Braket{#1|#1}}
\newcommand{\bk}[2]{\Braket{#1|#2}}
\newcommand{\bkev}[2]{\Braket{#2|#1|#2}}

\DeclareMathOperator{\na}{\bm{\nabla}}
\DeclareMathOperator{\nna}{\nabla^2}
\DeclareMathOperator{\drrr}{\dd[3]\vb{r}}

\DeclareMathOperator{\psis}{\psi^\ast}
\DeclareMathOperator{\Psis}{\Psi^\ast}
\DeclareMathOperator{\hi}{\hat{\vb{i}}}
\DeclareMathOperator{\hj}{\hat{\vb{j}}}
\DeclareMathOperator{\hk}{\hat{\vb{k}}}
\DeclareMathOperator{\hr}{\hat{\vb{r}}}
\DeclareMathOperator{\hT}{\hat{\vb{T}}}
\DeclareMathOperator{\hH}{\hat{\vb{H}}}
\DeclareMathOperator{\hp}{\hat{\vb{p}}}
\DeclareMathOperator{\hx}{\hat{\vb{x}}}
\DeclareMathOperator{\ha}{\hat{\vb{a}}}
\DeclareMathOperator{\Tdv}{-\dfrac{\hbar^2}{2m}\dv[2]{x}}
\DeclareMathOperator{\Tna}{-\dfrac{\hbar^2}{2m}\nabla^2}

%\DeclareMathOperator{\s}{\sum_{n=1}^{\infty}}
\DeclareMathOperator{\intinf}{\int_0^\infty}
\DeclareMathOperator{\intdinf}{\int_{-\infty}^\infty}
%\DeclareMathOperator{\suminf}{\sum_{n=0}^\infty}
\DeclareMathOperator{\sumnzinf}{\sum_{n=0}^\infty}
\DeclareMathOperator{\sumnoinf}{\sum_{n=1}^\infty}
\DeclareMathOperator{\sumndinf}{\sum_{n=-\infty}^\infty}
\DeclareMathOperator{\sumizinf}{\sum_{i=0}^\infty}


\newcommand{\dis}{\displaystyle}
\numberwithin{equation}{section}

\title{Notes of 141A\\
Solid State Physics}
\author{hebrewsnabla}
%\date{} % Activate to display a given date or no date (if empty),
         % otherwise the current date is printed 

\begin{document}
% \boldmath
\maketitle

\tableofcontents

\newpage

\setcounter{section}{6}
\section{Energy Bands}
\subsection{Nearly Free Elec Model}
\begin{equation}\label{key}
\begin{aligned}
U(x) &= U_0 \qty[\cos^4\dfrac{\pi x}{a} - \dfrac{3}{8}]\\
&= ...\\
&= U_0 \qty(\dfrac{1}{2}\cos\dfrac{2\pi x}{a} + \dfrac{1}{8}\cos\dfrac{4\pi x}{a})
\end{aligned}
\end{equation}
\begin{equation}\label{key}
(\lambda_k - \epsilon)C_k + \sum_G U_G C_{k-G} = 0
\end{equation}
solve reduced problem near degenerate points
\begin{equation}\label{key}
\mqty(\lambda_k - \epsilon & U_g\\
      U_g^\ast & \lambda_{k-g} - \epsilon)
\mqty(C_k \\ C_{k-g}) = 0  \quad (g = \dfrac{2\pi}{a})
\end{equation}
\begin{equation}\label{key}
(\lambda_k - \epsilon)^2 - \abs{U_g}^2 = 0
\end{equation}
\begin{equation}\label{key}
\epsilon = \dfrac{\hbar^2 k^2}{2m} \pm \dfrac{U_0}{4}
\end{equation}
or
\begin{equation}\label{key}
\epsilon = \dfrac{\hbar^2 (k+G_1)^2}{2m} \pm \dfrac{U_0}{16}
\end{equation}

\subsection{Bloch Oscillators}
Semi-classical eq. of motion
\begin{equation}\label{key}
\hbar\dv{\vb{k}}{t} = \vb{F} = -e\vb{E}
\end{equation}
\begin{equation}\label{key}
v_{grp} = \dfrac{1}{\hbar}\dv{\varepsilon}{k}
\end{equation}
consider
\begin{equation}\label{key}
\varepsilon(k) = \varepsilon_0(1 - \cos ka)
\end{equation}
\begin{equation}\label{key}
v = \dfrac{a\varepsilon_0}{\hbar}\sin ka
\end{equation}
\begin{equation}\label{key}
\begin{aligned}
x &= \int \dfrac{a\varepsilon_0}{\hbar}\sin ka \dd t\\
&= \dfrac{a\varepsilon_0}{\hbar} \int \sin ka \dd k \dv{t}{k}\\
&= ...\\
&= -\dfrac{\varepsilon_0}{eE}\qty(\cos\dfrac{-eEa}{\hbar} - 1)
\end{aligned}
\end{equation}

\begin{equation}\label{key}
T = \dfrac{\Delta k}{\dd k/\dd t} = \dfrac{2\pi/a}{eE/\hbar} = ...
\end{equation}
scattering time $ \tau >> T $. \\

suppose $ \vb{F} = m^\ast \vb{a} $
\begin{equation}\label{key}
m^\ast = \hbar\dfrac{\dd k/\dd t}{\dd v_g/\dd t} = \hbar \qty(\dv{v_g}{k})^{-1} = \hbar^2 \qty(\pdv[2]{\varepsilon}{k})^{-1}
\end{equation}


\subsection{Consequence of Band Structure}
even \# of elec's per unit cell -- insulator -- \ce{C, Si, Ge}\\
odd \# of elec's per unit cell -- metal -- \ce{Cu, Ag, Au}\\
Ins -- $ E_g > 2\; eV $\\
Semiconductor -- $ E_g < 2; eV $\\

\subsubsection{Density of States}
Van Hove Singularities of DoS

\section{Semiconductors}

\subsection{Photoconductivity}
\begin{equation}\label{key}
\vb{J}_n = \dfrac{1}{V}\int_{unocc} e D(\vb{k}_e) \vb{v}_e(\vb{k}_e)\dd[3]\vb{k}_e
\end{equation}
\begin{equation}\label{key}
\dv{\vb{k}_e}{t} = \dfrac{-e}{\hbar}(\vb{E} + \dfrac{1}{c}\vb{v}_e \cross\vb{B})
\end{equation}
\begin{equation}\label{key}
\vb{v}_e(\vb{k}_e) = \dfrac{1}{\hbar}\pdv{E_v(\vb{k}_e)}{\vb{k}_e}
\end{equation}
Holes: Def 
\begin{equation}\label{key}
\vb{k}_n = -\vb{k}_e \quad E_n = -E_v
\end{equation}
\begin{equation}\label{key}
\vb{v}_n(\vb{k}_n) = \dfrac{1}{\hbar}\pdv{E_n(\vb{k}_n)}{\vb{k}_n} = \dfrac{1}{\hbar}\pdv{(-E_v(-\vb{k}_e))}{(-\vb{k}_e)} = \vb{v}_e(\vb{k}_e)
\end{equation}
Holes: + charge, evolves like particle, $ m^\ast > 0 $

\subsection{Intrinsic Mobility}
\begin{equation}\label{key}
\vb{J}_{tot} = \vb{J}_e + \vb{J}_h
\end{equation}
\begin{equation}\label{key}
\sigma_{tot} = \sigma_e + \sigma_h
\end{equation}
\begin{equation}\label{key}
\sigma_e = \dfrac{n_e e^2 \tau_e}{\abs{m_e^\ast}}
\end{equation}
\begin{equation}\label{key}
\sigma_h = \dfrac{n_h e^2 \tau_h}{\abs{m_h^\ast}}
\end{equation}
def: mobility
\begin{equation}\label{key}
\mu = \dfrac{e\tau}{m^\ast}
\end{equation}
\begin{equation}\label{key}
\sigma_{tot} = n_e e \mu_e + n_h e \mu_h
\end{equation}


\subsection{Impurity Conductivity / Doping}
\begin{equation}\label{key}
n_e(T) \neq n_h(T) \quad\dra\quad \text{doping}
\end{equation}
Impurity atoms that can give up an electron are called donors.
\subsubsection{Donor States}
the donated elec moves in the coulomb potential $ -e/\epsilon r $

\begin{equation}\label{key}
N_d = \dfrac{\#donors}{Vol.} \quad N_a = \dfrac{\#acceptors}{Vol.}
\end{equation}
\begin{equation}\label{key}
n_e(T) = 2\qty(\dfrac{m_e k_B T}{2\pi\hbar^2})^{3/2}\e^{-(E_c - \mu)/k_B T}
\end{equation}
\begin{equation}\label{key}
n_p(T) = 2\qty(\dfrac{m_h k_B T}{2\pi\hbar^2})^{3/2}\e^{-(\mu - E_v)/k_B T}
\end{equation}
Intrinsic: 
\begin{equation}\label{key}
n_i(T) = \sqrt{n_e(T)n_p(T)}
\end{equation}
\begin{equation}\label{key}
\mu_i(T) = E_v + \dfrac{1}{2}E_g + \dfrac{3}{4}k_B T\ln\dfrac{m_h}{m_e}
\end{equation}
\begin{equation}\label{key}
\dfrac{N_d - N_a}{n_i(T)} = 2\sinh\dfrac{\mu - \mu_i}{k_B T}
\end{equation}

\subsection{Hall Effect w/ 2 Carrier Types}
Kittel 8.3\\
mobilities:
\begin{equation}\label{key}
\mu_e = \dfrac{e\tau_e}{m_e} \quad \mu_h = \dfrac{e\tau_h}{m_h}
\end{equation}
($ m_e $ is eff mass)\\
Recall conductivity tensor w/ B-field
\begin{equation}\label{key}
\sigma = \dfrac{\sigma_0}{(1 + \omega_c^2\tau^2)^2} 
		\mqty(1 & -\omega_c\tau\\
			  -\omega_c\tau & 1)
\end{equation}
where $ \sigma_0 = \dfrac{n e^2 \tau_e}{m_e}$, $ \omega_c = \dfrac{eB}{m_e c} $.\\
transverse current ($ \omega_c\tau << 1 $)
\begin{equation}\label{key}
j_y(e) = \sigma_0(\omega_c\tau E_x + E_y) = n e \mu_e (\dfrac{\mu_e B}{c}E_x + E_y)
\end{equation}
\begin{equation}\label{key}
j_y(h) = p e \mu_h (\dfrac{-\mu_h B}{c}E_x + E_y)
\end{equation}
longitudinal curr
\begin{equation}\label{key}
j_x = j_x(e) + j_x(h) = (n e \mu_e + p e \mu_h)E_x
\end{equation}
tot transverse curr = 0, thus
\begin{equation}\label{key}
(n\mu_e^2 - p\mu_h^2)\dfrac{eB}{c} E_x + (n\mu_e + p\mu_h)e E_y = 0
\end{equation}
\begin{equation}\label{key}
E_y = -E_x \dfrac{B}{c}\dfrac{n\mu_e^2 - p\mu_h^2}{n\mu_e + p\mu_h}
\end{equation}
\begin{equation}\label{key}
R_H = \dfrac{E_y}{j_x B} = -\dfrac{1}{e c}\dfrac{n\mu_e^2 - p\mu_h^2}{(n\mu_e + p\mu_h)^2}
\end{equation}

\subsection{Tight-binding}
start w/ AOs (1-D)
\begin{equation}\label{key}
\psi_k(r) = \sum_j c_j\phi(r - r_j)
\end{equation}
($ \phi $ is s orb)\\
Bloch Th.
\begin{equation}\label{key}
\psi_k(r) = \e^{\I k r} u(r)
\end{equation}
if $ c_j = \dfrac{1}{\sqrt{N}}\e^{\I kr} $
\begin{equation}\label{key}
\Braket{\psi_k | \hH | \psi_k} = \dfrac{1}{N}\sum_j\sum_m \e^{\I k(r_j - r_m)}\Braket{\phi_m | \hH | \phi_j}
\end{equation}
let $ \rho_m = r_m - r_j $, suppose only nearest neighbor interacts
\begin{equation}\label{key}
\Braket{\psi_k | \hH | \psi_k} = \dfrac{1}{N}\sum_m\e^{\I k\rho_m}\int\dd V \phi^*(r - \rho_m)\hH\phi(r)
\end{equation}
($ r - r_j \ra r $)\\
Let
\begin{equation}\label{key}
-\varepsilon_0 = \int\dd V \phi^*(r)\hH\phi(r) \quad -t = \int\dd V \phi^*(r - \rho)\hH\phi(r)
\end{equation}
\begin{equation}\label{key}
E = -\varepsilon_0 - t\sum_{n.n.}\e^{-\I k\rho_{n.n.}}
\end{equation}
for cubic lattice (1-D)
\begin{equation}\label{key}
E = -\varepsilon_0 -t (1 + 2\cos ka)
\end{equation}
consider a lattice w/ 2-atom basis
\begin{equation}\label{key}
\psi_k(r) = \alpha_k\psi_k^A(r) + \beta_k\psi_k^B(r)
\end{equation}
\begin{equation}\label{key}
H_{AB} = \Braket{\psi_k^A | \hH | \psi_k^B}
= \mqty( AHA & AHB\\
		BHA & BHB)
= \mqty(-\varepsilon_1 & \sum_{n.n.}\e^{-\I k\rho_{n.n.}}\\
		\sum_{n.n.}\e^{\I k\rho_{n.n.}} & -\varepsilon_2)
\end{equation}
\begin{equation}\label{key}
H_{AB} \mqty(\alpha\\ \beta) = E_k \mqty(\alpha\\ \beta)
\end{equation}

\begin{equation}\label{key}
H = \mqty(-\varepsilon_1 & t(1 + \e^{-\I ka})\\
		t(1 + \e^{\I ka}) & -\varepsilon_2)
\end{equation}
\begin{equation}\label{key}
\det(H - E_k I) = 0
\end{equation}
\begin{equation}\label{key}
E_k = \dfrac{1}{2}(\varepsilon_1 + \varepsilon_2) \pm \sqrt{...}
\end{equation}

\subsection{p-n Junctions}
p-n jcn\\
-- diode\\
-- half-transistor\\
-- solar cell\\
-- LED\\
-- laser\\
~\\

\begin{equation}\label{key}
I = I_{CB} + I_{VB}
\end{equation}
conduction band, valence band
\begin{equation}\label{key}
I_{CB} = I_{nr} - I_{ng}
\end{equation}
recomb, generation
\begin{equation}\label{key}
I_{nr} \sim \e^{-e(\Delta\phi - V)\beta}
\end{equation}
\begin{equation}\label{key}
I_{ng} \sim \e^{-e\Delta\phi\beta}
\end{equation}
\begin{equation}\label{key}
ICB = Ing(\e^{eV\beta} - 1)
\end{equation}
\begin{equation}\label{key}
IVB = Ihr - Ihg
\end{equation}
\begin{equation}\label{key}
IVB = Ihg(\e^{eV\beta} - 1)
\end{equation}
\begin{equation}\label{key}
I = (Ing + Ihg)(\e^{eV\beta} - 1)
\end{equation}
LED\\
Solar cell\\
Schottkey barrier\\
p-n-p jcn

\section{Tight Binding}
Start: Localized picture\\
\begin{equation}\label{key}
\begin{aligned}
\Psi_{\vb{k}} &= \dfrac{1}{\sqrt{N}} \sum_{j=1}^N C_{\vb{k},j} \phi(\vb{r} - \vb{r}_j)\\
&= \dfrac{1}{\sqrt{N}} \sum_{j=1}^N \e^{\I\vb{k}\cdot\vb{r}_j} \phi(\vb{r} - \vb{r}_j) \quad \text{(Bloch Th.)}
\end{aligned}
\end{equation}
\begin{equation}\label{key}
\begin{aligned}
\varepsilon(\vb{k}) &= \bkev{\hH}{\Psi_{\vb{k}}} = \dfrac{1}{N} \bkev{\hH}{\sum_{j=1}^N \e^{\I\vb{k}\cdot\vb{r}_j} \phi(\vb{r} - \vb{r}_j)}\\
&= \dfrac{1}{N} \sum_j \sum_n \e^{\I\vb{k}\cdot(\vb{r}_j - \vb{r}_n)} \Braket{\phi(\vb{r} - \vb{r}_n) | \hH | \phi(\vb{r} - \vb{r}_j)}
\end{aligned}
\end{equation}
Let
\begin{equation}\label{key}
\bm{\rho}_m =  \vb{r}_j - \vb{r}_n \quad \bm{\xi}_n = \vb{r} - \vb{r}_n
\end{equation}
\begin{equation}\label{key}
\begin{aligned}
\varepsilon(\vb{k}) &= \dfrac{1}{N} \sum_m \sum_n \e^{\I\vb{k}\cdot\bm{\rho}_m} \Braket{\phi(\bm{\xi}_n) | \hH | \phi(\bm{\xi}_n - \bm{\rho}_m)}\\
&= \sum_m \e^{\I\vb{k}\cdot\bm{\rho}_m} \Braket{\phi(\bm{\xi}) | \hH | \phi(\bm{\xi} - \bm{\rho}_m)}\\
&\equiv \sum_m \e^{\I\vb{k}\cdot\bm{\rho}_m} I(\rho_m)
\end{aligned}
\end{equation}
Overlap integral
\begin{equation}\label{key}
I(\rho_m) = \Braket{\phi(\vb{r}) | \hH | \phi(\vb{r} - \bm{\rho}_m)}
\end{equation}
Now neglect all non-neighbor interaction.\\
when $ \rho_m = \rho $($ t $: hopping amplitude.)
\begin{equation}\label{key}
-t = I(\rho) = \Braket{\phi(\vb{r}) | \hH | \phi(\vb{r} - \bm{\rho})}
\end{equation}
when $\rho_m = 0 $
\begin{equation}\label{key}
-\alpha = \Braket{\phi(\vb{r}) | \hH | \phi(\vb{r})}
\end{equation}
thus
\begin{equation}\label{key}
\varepsilon(\vb{k}) = -\alpha -t \sum_m \e^{\I\vb{k}\cdot\bm{\rho}_m}
\end{equation}

\subsection{1-D Crystal}
\begin{equation}\label{key}
\begin{aligned}
\varepsilon(k) &= ... = -\alpha -2t\cos ka\\
&= -\alpha - 2t\qty(1 - \dfrac{k^2 a^2}{2})\\
&= -\alpha - 2t + t k^2 a^2
\end{aligned}
\end{equation}
let
\begin{equation}\label{key}
m^* = \dfrac{\hbar^2}{2ta^2}
\end{equation}
\begin{equation}\label{key}
\varepsilon(k) = -\alpha - 2t + \dfrac{\hbar^2 k^2}{2m^*}
\end{equation}

\subsection{Graphene Tight-binding}
General tight-binding for 2 atom-basis
\begin{equation}\label{key}
\psi_k = a_k \psi_k^{(A)}(r) + b_k \psi_k^{(B)}(r)
\end{equation}
\begin{equation}\label{key}
\psi_k^{(B)}(r) = \sum_j c_j \phi^{(B)}(r - r_j)
\end{equation}


\section{Screening}
\begin{equation}\label{key}
D(k,\omega) = \varepsilon(k,\omega) E(k,\omega)
\end{equation}
\subsection{Static Screening}
$\omega = 0 $
\begin{equation}\label{key}
E(r) = \sum_k \e^{\I k\cdot r} E(k)
\end{equation}
\begin{equation}\label{key}
\rho(r) = \sum_k \e^{\I k\cdot r} \rho(k)
\end{equation}
\begin{equation}\label{key}
D(r) = \sum_k \e^{\I k\cdot r} D(k)
\end{equation}

\begin{equation}\label{key}
\na \cdot E = \na \cdot \qty(\sum_k \e^{\I k\cdot r} E(k)) =  4\pi \sum_k \e^{\I k\cdot r} \rho(k)
\end{equation}
\begin{equation}\label{key}
\na\cdot D = \na\cdot \qty(\sum_k \e^{\I k\cdot r} \varepsilon(k) E(k)) = 4\pi \sum_k \e^{\I k\cdot r} \rho_{ext}(k)
\end{equation}
$ \therefore $
\begin{equation}\label{key}
\varepsilon(k) = \dfrac{\rho_{ext}(k)}{\rho(k)} = 1 - \dfrac{\rho_{ind}(k)}{\rho(k)}
\end{equation}
In terms of potential, ...\\

\subsection{Calculating $ \rho_{ind} $ Using Thomas-Fermi Theory of Screening}
\begin{equation}\label{key}
\Tdv \psi_i(r) = E_i \psi_i(r)
\end{equation}
\begin{equation}\label{key}
V = -e\phi(r)
\end{equation}
$ \phi(r) $ varies slowly enough that
\begin{equation}\label{key}
E_i = \dfrac{\hbar^2 k^2}{2m} - e\phi(r)
\end{equation}








\end{document}